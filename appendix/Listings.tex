\chapter{Additional Listings}
\label{app:Listings}
  This appendix contains additional listings of the source code used in this
  thesis that are not essential to the understanding of the thesis.
  The listings are included here for completeness.
  
  \begin{code}{
    Calculation of \(|\mathbb{T}_{\leq 5}(\mathcal{T},\, \mathcal{F})|\) for 
    \(\mathcal{T} = \{x,\, c\}\) and the set \(\mathcal{F} = \{+,\, -,\, 
    \times,\, /,\, \sin,\, \cos,\, \mathrm{pow}\}\) using the \textit{Julia}
    programming language.
  }{label={lst:cardinality_of_T_leq_5}}{julia}
    arities = [2, 2, 2, 2, 1, 1, 2] # $A\left(\{+,\,-,\,\times,\,/,\,\sin,\,\cos,\,\mathrm{pow}\}\right) = \{2,\,2,\,2,\,2,\,1,\,1,\,2\}$
    terminals_size = 8  # $|\mathcal{T}| = |\{x,\,1,\,2,\,3,\,4,\,5,\,6,\,7\}| = 8$
    # $|\mathbb{T}_{\leq \mathtt{h}}|$
    t_leq(h::Int)::Int128 = if h == 0 # $|\mathcal{T}|$ if $\mathtt{h} = 0$
      terminals_size
    else  # $\left(\sum_{h = 0}^{\mathtt{h} - 1}\,\sum_{f \in \mathcal{F}} |\mathbb{T}_h|^{A(f)}\right)+ |\mathcal{T}|$ if $\mathtt{h} > 0$
      c_sum = terminals_size
      for i = 0:h - 1
        c_sum = c_sum + sum(t(i) .^ arities)
      end
      c_sum
    end
    # $|\mathbb{T}_\mathbf{h}|$
    t(h::Int)::Int128 = if h == 0 # $|\mathcal{T}|$ if $\mathtt{h} = 0$
      terminals_size
    else  # $\sum_{f \in \mathcal{F}} |\mathbb{T}_{\mathtt{h} - 1}|^{A(f)}$ if $\mathtt{h} > 0$
      sum(t(h - 1) .^ arities) 
    end
    res = t_leq(5)  # $|\mathbb{T}_{\leq 5}(\mathcal{T},\, \mathcal{F})|$
  \end{code}
  
  \begin{code}{
    Minimal implementation of the \textit{One Max} problem using the 
    \textit{Keen} framework.
  }{label=lst:app:keen_omp}{kotlin}
    fun main() {
        Core.random = Random(11)
        val result = engine(
            { gt: Genotype<Boolean, BoolGene> -> gt.flatten().count { it }.toDouble() },
            genotype { chromosome { booleans { size = 20; truesProbability = 0.5 } } }) {
            populationSize = 20
            selector = TournamentSelector(sampleSize = 3)
            alterers = listOf(
              BitFlipMutator(probability = 0.03),
              SinglePointCrossover(probability = 0.2)
            )
            limits = listOf(TargetFitness(20.0))
            }.evolve()
        println("Target fitness reached at generation ${result.generation}")
        println("Best individual is ${result.best.genotype}")
        println("with fitness ${result.best.fitness}")
    }
  \end{code}


  \begin{code}{
    \textit{Room Scheduling} problem using the \textit{Keen} framework.
  }{label=lst:app:keen_room_scheduling}{kotlin}
    data class Meeting(val start: Int, val end: Int)

    private val meetings =
        listOf(
            Meeting(1, 3),
            Meeting(2, 3),
            Meeting(5, 6),
            Meeting(7, 9),
            Meeting(4, 7),
            Meeting(8, 10),
            Meeting(2, 7),
            Meeting(3, 4),
            Meeting(1, 5),
            Meeting(3, 6),
            Meeting(4, 5)
        )

    private fun fitnessFn(genotype: Genotype<Int, IntGene>): Double {
        // Create a map to represent the rooms, where the key is the room number and the value 
        // is a list of meetings in that room.
        val rooms = meetings.groupBy { 
            genotype.chromosomes[meetings.indexOf(it)].genes[0].dna 
        }
        // Calculate the number of conflicts in each room.
        val conflicts = rooms.values.sumOf { meetingList ->
            val table = IntArray(10) // Create an array to represent the time slots in a day.
            meetingList.forEach { meeting ->
                // Increment the time slots for each meeting.
                for (i in meeting.start until meeting.end) {
                    table[i]++
                }
            }
            // Count the number of time slots with more than one meeting.
            table.count { it > 1 }
        }
        // The fitness score is the number of rooms used plus the number of conflicts.
        // We add 1 to the number of rooms used to avoid a fitness score of 0.
        return rooms.size.toDouble() + conflicts
    }

    fun main() {
        // Create a genetic algorithm engine with the fitness function and genotype for the 
        // problem.
        val engine = engine(::fitnessFn, genotype {
            repeat(meetings.size) {
                chromosome {
                    ints { size = 1; range = meetings.indices.first to meetings.indices.last }
                }
            }
        }) {
            // Set the parameters for the genetic algorithm.
            populationSize = 100
            optimizer = FitnessMinimizer()
            alterers = listOf(Mutator(0.06), SinglePointCrossover(0.2))
            limits = listOf(SteadyGenerations(20), GenerationCount(100))
            listeners = listOf(EvolutionSummary(), EvolutionPlotter())
        }
        // Evolve the population and get the best result.
        val result = engine.evolve()
        // Print the statistics of the genetic algorithm.
        println(engine.listeners.first())
        // Create a schedule based on the best genotype.
        val schedule = MutableList(result.best.genotype.size) { mutableListOf<Meeting>() }
        meetings.forEachIndexed { index, meeting ->
            val room = result.best.genotype.chromosomes[index].genes[0].dna
            schedule[room].add(meeting)
        }
        // Print the schedule for each room.
        schedule.forEachIndexed { index, room ->
            println("Room $index: $room")
        }
        // Display a plot of the fitness values over time.
        (engine.listeners.last() as EvolutionPlotter).displayFitness()
    }
  \end{code}