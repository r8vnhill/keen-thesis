\chapter{Additional Listings}
\label{app:Listings}
  This appendix contains additional listings of the source code used in this
  thesis that are not essential to the understanding of the thesis.
  The listings are included here for completeness.
  
  \begin{code}{
    Calculation of \(|\mathbb{T}_{\leq 5}(\mathcal{T},\, \mathcal{F})|\) for 
    \(\mathcal{T} = \{x,\, c\}\) and the set \(\mathcal{F} = \{+,\, -,\, 
    \times,\, /,\, \sin,\, \cos,\, \mathrm{pow}\}\) using the \textit{Julia}
    programming language.
  }{label={lst:cardinality_of_T_leq_5}}{julia}
    arities = [2, 2, 2, 2, 1, 1, 2] # $A\left(\{+,\,-,\,\times,\,/,\,\sin,\,\cos,\,\mathrm{pow}\}\right) = \{2,\,2,\,2,\,2,\,1,\,1,\,2\}$
    terminals_size = 8  # $|\mathcal{T}| = |\{x,\,1,\,2,\,3,\,4,\,5,\,6,\,7\}| = 8$
    # $|\mathbb{T}_{\leq \mathtt{h}}|$
    t_leq(h::Int)::Int128 = if h == 0 # $|\mathcal{T}|$ if $\mathtt{h} = 0$
      terminals_size
    else  # $\left(\sum_{h = 0}^{\mathtt{h} - 1}\,\sum_{f \in \mathcal{F}} |\mathbb{T}_h|^{A(f)}\right)+ |\mathcal{T}|$ if $\mathtt{h} > 0$
      c_sum = terminals_size
      for i = 0:h - 1
        c_sum = c_sum + sum(t(i) .^ arities)
      end
      c_sum
    end
    # $|\mathbb{T}_\mathbf{h}|$
    t(h::Int)::Int128 = if h == 0 # $|\mathcal{T}|$ if $\mathtt{h} = 0$
      terminals_size
    else  # $\sum_{f \in \mathcal{F}} |\mathbb{T}_{\mathtt{h} - 1}|^{A(f)}$ if $\mathtt{h} > 0$
      sum(t(h - 1) .^ arities) 
    end
    res = t_leq(5)  # $|\mathbb{T}_{\leq 5}(\mathcal{T},\, \mathcal{F})|$
  \end{code}
  
  \begin{code}{
    Minimal implementation of the \textit{One Max} problem using the 
    \textit{Keen} framework.
  }{label=lst:app:keen_omp}{kotlin}
    fun main() {
        Core.random = Random(11)
        val result = engine(
            { gt: Genotype<Boolean, BoolGene> -> gt.flatten().count { it }.toDouble() },
            genotype { chromosome { booleans { size = 20; truesProbability = 0.5 } } }) {
            populationSize = 20
            selector = TournamentSelector(sampleSize = 3)
            alterers = listOf(
              BitFlipMutator(probability = 0.03),
              SinglePointCrossover(probability = 0.2)
            )
            limits = listOf(TargetFitness(20.0))
            }.evolve()
        println("Target fitness reached at generation ${result.generation}")
        println("Best individual is ${result.best.genotype}")
        println("with fitness ${result.best.fitness}")
    }
  \end{code}


  \subimport{listings/}{RoomScheduling.tex}

  \begin{code}{
    Example of using the \texttt{Fun} class to implement the addition 
    operation
  }{
    label=lst:app:keen_fun_add
  }{kotlin}
    // Using the Fun class
    var add = Fun<Double>("+", 2) { it[0] + it[1] }
    // Extending the Fun class
    class Add : Fun<Double>("+", 2, { it[0] + it[1] })
    var add = Add()
    // As an anonymous object
    var add = object : Fun<Double>("+", 2, { it[0] + it[1] }) {}
  \end{code}

  \begin{code}{
    Demonstration of the use of \textit{Strait-Jakt}'s \texttt{constraint} DSL
    to validate preconditions inside a \textit{Probability Selector}.
  }{
    label=lst:app:strait-jakt
  }{kotlin}
    private fun selectByProbabilities(
      population: Population<T, G>,
      probabilities: List<Double>,
      count: Int,
    ): List<Individual<T, G>> {
      constraints {
        "The cumulative probabilities list must have the same size as the population" {
          probabilities must HaveSize(population.size)
        }
        "The last cumulative probability must be 1.0" {
          probabilities.last() must BeEqualTo(1.0)
        }
      }
      return List(count) {
        population[probabilities.indexOfFirst { Domain.random.nextDouble() <= it }]
      }
    }
  \end{code}

  \subimport{listings/}{Beacon.tex}