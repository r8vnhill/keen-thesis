\chapter{Additional Listings}
\label{app:Listings}
    This appendix contains additional listings of the source code used in this
    thesis that are not essential to the understanding of the thesis.
    The listings are included here for completeness.
    
    \section{Keen}
        \begin{code}{
            Minimal implementation of the \textit{One Max} problem using the \textit{Keen} framework.
        }{label=lst:app:keen_omp}{kotlin}
            Domain.random = Random(11)
            val engine =
                evolutionEngine(
                    { genotype: Genotype<Boolean, BooleanGene> -> genotype.flatten().count { it }.toDouble() },
                    genotypeOf {
                        chromosomeOf {
                            booleans {
                                size = 20
                                trueRate = 0.15
                            }
                        }
                    }) {
                    populationSize = 20
                    alterers += listOf(
                        BitFlipMutator(individualRate = 0.5), SinglePointCrossover(chromosomeRate = 0.6)
                    )
                    limits += TargetFitness(TARGET_FITNESS)
                }
            engine.evolve()
            engine.listeners.forEach { it.display() }
        \end{code}

        \subimport{listings/}{RoomScheduling.tex}

        \begin{code}{
            Example of using the \texttt{Fun} class to implement the addition 
            operation
        }{
            label=lst:app:keen_fun_add
        }{kotlin}
            // Using the Fun class
            var add = Fun<Double>("+", 2) { it[0] + it[1] }
            // Extending the Fun class
            class Add : Fun<Double>("+", 2, { it[0] + it[1] })
            var add = Add()
            // As an anonymous object
            var add = object : Fun<Double>("+", 2, { it[0] + it[1] }) {}
        \end{code}

    \section{Crash Reproduction}
        \subimport{listings/}{Beacon.tex}

    \section{Others}
        \begin{code}{
            Calculation of \(|\mathbb{T}_{\leq 5}(\mathcal{T},\, \mathcal{F})|\) for 
            \(\mathcal{T} = \{x,\, c\}\) and the set \(\mathcal{F} = \{+,\, -,\, 
            \times,\, /,\, \sin,\, \cos,\, \mathrm{pow}\}\) using the \textit{Julia}
            programming language.
        }{label={lst:cardinality_of_T_leq_5}}{julia}
            arities = [2, 2, 2, 2, 1, 1, 2] # $A\left(\{+,\,-,\,\times,\,/,\,\sin,\,\cos,\,\mathrm{pow}\}\right) = \{2,\,2,\,2,\,2,\,1,\,1,\,2\}$
            terminals_size = 8  # $|\mathcal{T}| = |\{x,\,1,\,2,\,3,\,4,\,5,\,6,\,7\}| = 8$
            # $|\mathbb{T}_{\leq \mathtt{h}}|$
            t_leq(h::Int)::Int128 = if h == 0 # $|\mathcal{T}|$ if $\mathtt{h} = 0$
            terminals_size
            else  # $\left(\sum_{h = 0}^{\mathtt{h} - 1}\,\sum_{f \in \mathcal{F}} |\mathbb{T}_h|^{A(f)}\right)+ |\mathcal{T}|$ if $\mathtt{h} > 0$
            c_sum = terminals_size
            for i = 0:h - 1
                c_sum = c_sum + sum(t(i) .^ arities)
            end
            c_sum
            end
            # $|\mathbb{T}_\mathbf{h}|$
            t(h::Int)::Int128 = if h == 0 # $|\mathcal{T}|$ if $\mathtt{h} = 0$
            terminals_size
            else  # $\sum_{f \in \mathcal{F}} |\mathbb{T}_{\mathtt{h} - 1}|^{A(f)}$ if $\mathtt{h} > 0$
            sum(t(h - 1) .^ arities) 
            end
            res = t_leq(5)  # $|\mathbb{T}_{\leq 5}(\mathcal{T},\, \mathcal{F})|$
        \end{code}

        \begin{code}{
            \enquote{Desugared} version of the code presented in \vref{lst:keen:strait-jakt-syntax}
        }{label=app:lst:strait-jakt-desugared}{kotlin}
            constraints(() -> {
                invoke("The value of x must be greater than or equal to 10", () -> {
                    must(x, BeAtLeast(10))
                })
                invoke("The value of y must not be negative", 
                    () -> CustomConstraintException("The value of y must not be negative")
                () -> {
                    mustNot(y, BeNegative)
                })
            })
        \end{code}
        
        \begin{code}{
            Demonstration of the use of \textit{Strait-Jakt}'s \texttt{constraint} DSL to validate preconditions inside the 
            \textit{Subtree Crossover} operator.
        }{
            label=lst:app:strait-jakt
        }{kotlin}
            private fun enforcePreconditions(chromosomes: List<Chromosome<T, G>>) = constraints {
                "The crossover operator requires two chromosomes." { chromosomes must HaveSize(2) }
                chromosomes.forEach { chromosome ->
                    "The parents must have the same size" { 
                        chromosome must HaveSize(chromosomes[0].size) 
                    }
                    chromosome.forEach { gene ->
                        ("The gene's arity (${gene.value.arity}) must match "
                            + "the gene's children (${gene.value.children.size})") {
                            gene.value.children must HaveSize(gene.value.arity)
                        }
                    }
                }
            }
        \end{code}

        In \vref{lst:app:strait-jakt} we can see an example of the use of the \textit{StraitJakt} library to validate 
        the preconditions of the \textit{Subtree Crossover} operator. The \texttt{constraints} DSL allows us to define a 
        set of constraints that must be satisfied for the operation to be executed. In this case, we are validating that 
        the operator receives two chromosomes with the same size and that each gene's arity matches the number of 
        children it has. If any of these constraints is not satisfied, the \textit{StraitJakt} library will throw a 
        \textit{composite exception} containing all the errors found. This is done by accumulating the exceptions
        instead of throwing them immediately. This approach allows us to validate all the constraints at once, instead
        of throwing an exception for each error found.
