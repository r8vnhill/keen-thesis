\chapter{Glossary}
\label{appendix:glossary}
  In this appendix we present a glossary of terms used throughout this document as a reference for
  the reader.

  \section*{A}
    \begin{definition}[Alteration]
    \label{def:alteration}
      See \vref{sec:genetic_algorithms:variation}
    \end{definition}

    \begin{definition}[Alterer]
    \label{def:alterer}
      See \vref{def:variation_operator}.
    \end{definition}

  \section*{C}
    \begin{definition}[Chromosome]
    \label{def:chromosome}
      Representation of a single column of genetic information of a candidate solution to a given
      optimization problem.

      Formally, a chromosome is a vector \(\textbf{c} = (g_1, g_2, \dots, g_n)\), where \(g_i\) is a
      gene (see \vref{def:gene}).
    \end{definition}

  \section*{E}
    \begin{definition}[Evolutionary computation]
    \label{def:evolutionary_computation}
      Family of algorithms for global optimization inspired by the process of natural selection.

      This typically involves processes mimicking natural selection, mutation, recombination, and 
      survival of the fittest. 
      The solutions to a problem are encoded as a set of \enquote{individuals} in a 
      \enquote{population}. 
      Over multiple generations, these individuals are selected and modified (via genetic operators 
      like crossover and mutation) in order to find better solutions.
    \end{definition}
    
  \section*{G}
    \begin{definition}[Gene]
    \label{def:gene}
      Representation of a single component of a candidate solution to a given optimization problem.

      Formally, for a multi-dimensional function \(f\), a gene is an element \(g\) in the domain of
      \(f\).
    \end{definition}

    \begin{definition}[Generation]
    \label{def:generation}
      Number of iterations performed by an evolutionary algorithm.
    \end{definition}

    \begin{definition}[Genotype]
    \label{def:genotype}
      Representation of the full genetic information of a candidate solution to a given optimization
      problem.

      Formally, a genotype is a matrix \(\mathbf{G} = (\textbf{c}_1, \textbf{c}_2, \dots, 
      \textbf{c}_n)\), where \(\textbf{c}_i\) is a chromosome (see \vref{def:chromosome}).
    \end{definition}

  \section*{I}
    \begin{definition}[Individual]
    \label{def:individual}
      A candidate solution to a given optimization problem.

      Formally, an individual is a pair \((\mathbf{G}, \mathbf{f})\), where \(\mathbf{G}\) is the
      genotype (see \vref{def:genotype}) and \(\mathbf{f}\) is the fitness value of the
      individual.
    \end{definition}

  \section*{M}
    \begin{definition}[Metaheuristics]
    \label{def:metaheuristic}
      Problem-independent algorithmic method that yields a sufficiently good solution within 
      reasonable time for an optimization problem, especially for complex problems where an exact 
      solution is not crucial.
    \end{definition}

    \begin{definition}[Mutator]
    \label{def:mutator}
      See \vref{def:mutation_operator}.
    \end{definition}

  \section*{P}
    \begin{definition}[Phenotype]
    \label{def:phenotype}
      Same as \vref{def:individual}.
    \end{definition}

    \begin{definition}[Population]
    \label{def:population}
      Set of candidate solutions to a given optimization problem.
    \end{definition}

  \section*{S}
    \begin{definition}[Search space]
    \label{def:search_space}
      Set of all candidate solutions to a given optimization problem.
    \end{definition}

    \begin{definition}[Selector]
      See \vref{def:selection_operator}.
    \end{definition}

  \section*{V}
    \begin{definition}[Variadic function]
    \label{def:variadic_function}
      Function that accepts a variable number of arguments.  
    \end{definition}
