\section*{G}
\begin{definition}[Gene]
  \label{def:gene}
    Representation of a single component of a candidate solution to a given 
    optimization problem.

    Formally, for a multi-dimensional function \(f\), a gene is an element 
    \(g\) in the domain of \(f\).
  \end{definition}

  \begin{definition}[Generation]
  \label{def:generation}
    Number of iterations performed by an evolutionary algorithm.
  \end{definition}
  
  \begin{definition}[Genetic Diversity]
  \label{def:diversity}
    In evolutionary computation, diversity refers to the degree of variation or 
    difference in the genetic representation of individuals within a 
    population.
    Maintaining diversity is essential for several reasons:

    \begin{enumerate}
      \item \textbf{Exploration vs. Exploitation:} A diverse population can 
        explore various regions of the solution space, ensuring that the 
        algorithm doesn't focus solely on one area.
        This balance between exploration (searching new areas) and exploitation 
        (optimizing known good areas) is fundamental in evolutionary 
        computation.
      \item \textbf{Preventing Premature Convergence:} Without adequate 
        diversity, the population may converge too quickly to a suboptimal 
        solution, known as a local optimum.
        By maintaining diversity, the population has a better chance of 
        discovering the global optimum.
      \item \textbf{Adaptability:} A diverse population can better adapt to 
        changing environments or requirements, making it more resilient against 
        dynamic problems.
    \end{enumerate}

    Diversity can be measured in several ways, including genetic distance 
    metrics, fitness-based measures, or phenotypic variance.
    To maintain or introduce diversity, techniques like mutation, crossover 
    variations, immigration, and diversity preservation strategies (like 
    fitness sharing or crowding) are employed.
  \end{definition}
  
  \begin{definition}[Genotype]
  \label{def:genotype}
    Representation of the full genetic information of a candidate solution to a given optimization
    problem.

    Formally, a genotype is a matrix \(\mathbf{G} = (\textbf{c}_1, \textbf{c}_2, \dots, 
    \textbf{c}_n)\), where \(\textbf{c}_i\) is a chromosome (see \vref{def:chromosome}).
  \end{definition}