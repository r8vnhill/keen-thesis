\chapter{Glossary}
\label{appendix:glossary}
  In this appendix we present a glossary of terms used throughout this document 
  that may be unfamiliar to the reader and are not defined in the main text.

  \subimport{./}{A.tex}
  \subimport{./}{C.tex}
  \subimport{./}{E.tex}
  \subimport{./}{G.tex}
    
  \subimport{./}{K.tex}
  \subimport{./}{M.tex}
  \subimport{./}{P.tex}
  \subimport{./}{S.tex}

  \section*{U}
    \begin{definition}[Unimodal function]
    \label{def:unimodal_function}
      A unimodal function is a function that, within a given domain or interval,
      has only one local maximum and one local minimum, which can also be the
      global maximum and minimum respectively.
      In other words, the function increases to a certain point and then 
      decreases, or vice versa.
      This single peak or trough is the \enquote{uni} in \enquote{unimodal}.

      Mathematically speaking, if \(f\) is a unimodal function in the interval 
      \([a,\, b]\), then there exists some point \(c\) in \([a,\, b]\) such 
      that:

      \begin{enumerate}
        \item \(f(x)\) is increasing on \([a,\, c]\) and decreasing on \([c,\, 
          b]\), or
        \item \(f(x)\) is decreasing on \([a,\, c]\) and increasing on \([c,\,
          b]\).
      \end{enumerate}
    \end{definition}

  \section*{V}
    \begin{definition}[Variadic function]
    \label{def:variadic_function}
      Function that accepts a variable number of arguments.
    \end{definition}
