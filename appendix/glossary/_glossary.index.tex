\chapter{Glossary}
\label{appendix:glossary}
  In this appendix we present a glossary of terms used throughout this document 
  that may be unfamiliar to the reader and are not defined in the main text.

  \subimport{./}{A.tex}

  \section*{C}
    \begin{definition}[Chromosome]
    \label{def:chromosome}
      Representation of a single column of genetic information of a candidate solution to a given
      optimization problem.

      Formally, a chromosome is a vector \(\textbf{c} = (g_1, g_2, \dots, g_n)\), where \(g_i\) is a
      gene (see \vref{def:gene}).
    \end{definition}

  \subimport{./}{E.tex}
    
  \section*{G}
    \begin{definition}[Gene]
    \label{def:gene}
      Representation of a single component of a candidate solution to a given optimization problem.

      Formally, for a multi-dimensional function \(f\), a gene is an element \(g\) in the domain of
      \(f\).
    \end{definition}

    \begin{definition}[Generation]
    \label{def:generation}
      Number of iterations performed by an evolutionary algorithm.
    \end{definition}

    \begin{definition}[Genotype]
    \label{def:genotype}
      Representation of the full genetic information of a candidate solution to a given optimization
      problem.

      Formally, a genotype is a matrix \(\mathbf{G} = (\textbf{c}_1, \textbf{c}_2, \dots, 
      \textbf{c}_n)\), where \(\textbf{c}_i\) is a chromosome (see \vref{def:chromosome}).
    \end{definition}

  \section*{I}
    \begin{definition}[Individual]
    \label{def:individual}
      A candidate solution to a given optimization problem.

      Formally, an individual is a pair \((\mathbf{G}, \mathbf{f})\), where \(\mathbf{G}\) is the
      genotype (see \vref{def:genotype}) and \(\mathbf{f}\) is the fitness value of the
      individual.
    \end{definition}
    
  \subimport{./}{K.tex}
  \subimport{./}{M.tex}
  \subimport{./}{P.tex}

  \section*{S}
    \begin{definition}[Search space]
    \label{def:search_space}
      Set of all candidate solutions to a given optimization problem.
    \end{definition}

    \begin{definition}[Selector]
      See \vref{def:selection_operator}.
    \end{definition}

  \section*{U}
    \begin{definition}[Unimodal function]
    \label{def:unimodal_function}
      A unimodal function is a function that, within a given domain or interval,
      has only one local maximum and one local minimum, which can also be the
      global maximum and minimum respectively.
      In other words, the function increases to a certain point and then 
      decreases, or vice versa.
      This single peak or trough is the \enquote{uni} in \enquote{unimodal}.

      Mathematically speaking, if \(f\) is a unimodal function in the interval 
      \([a,\, b]\), then there exists some point \(c\) in \([a,\, b]\) such 
      that:

      \begin{enumerate}
        \item \(f(x)\) is increasing on \([a,\, c]\) and decreasing on \([c,\, 
          b]\), or
        \item \(f(x)\) is decreasing on \([a,\, c]\) and increasing on \([c,\,
          b]\).
      \end{enumerate}
    \end{definition}

  \section*{V}
    \begin{definition}[Variadic function]
    \label{def:variadic_function}
      Function that accepts a variable number of arguments.
    \end{definition}
