\section*{E}
  \begin{definition}[Elitism]
    \label{def:elitism}
    In evolutionary computation, elitism refers to a strategy wherein the 
    top-performing individuals from a population are given a higher probability 
    of being selected for the next generation.
    While this doesn't guarantee their direct transfer, it increases the 
    likelihood of retaining the best solutions in subsequent generations, 
    ensuring that the quality of the population does not degrade.
  \end{definition}

  \begin{definition}[Ephemeral Constant]
  \label{def:ephemeral_constant}
    A constant that is randomly generated at the start of the program and 
    remains constant throughout the execution of the program.

    This is used to represent constant values in the program.
  \end{definition}

  \begin{definition}[Evolutionary computation]
  \label{def:evolutionary_computation}
    Family of algorithms for global optimization inspired by the process of natural selection.

    This typically involves processes mimicking natural selection, mutation, recombination, and 
    survival of the fittest.

    The solutions to a problem are encoded as a set of \enquote{individuals} in a 
    \enquote{population}.

    Over multiple generations, these individuals are selected and modified (via genetic operators 
    like crossover and mutation) in order to find better solutions.
  \end{definition}
