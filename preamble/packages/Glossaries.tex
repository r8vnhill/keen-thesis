% | glossaries - Package for creating glossaries and lists of acronyms in a document.
% |
% | The glossaries package allows for the easy creation and management of glossaries and lists of 
% | acronyms within a LaTeX document. 
% | It has the ability to automatically sort entries, create hyperlinks to glossary terms, 
% | and generate the glossary or acronym list automatically.
% |
% | The package is loaded in the preamble of the document with the command \usepackage{glossaries}.
% |
% | After it's loaded, new glossary entries can be defined using the \newglossaryentry command, 
% | and acronyms can be defined using the \newacronym command. These entries can then be used in the 
% | document with the \gls command.
% |
% | When the document is compiled, the glossaries package automatically creates a glossary section 
% | and an acronyms section, which list all the glossary entries and acronyms used in the document, 
% | along with their corresponding definitions or explanations.
% |
% | It is recommended to load this package after hyperref, to make sure all the links work properly.
\usepackage{glossaries}
