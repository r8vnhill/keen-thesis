% -------------------------
% NEW THEOREM ENVIRONMENTS
% -------------------------

% | newtheorem - Defines new theorem-like environments
% |
% | The 'newtheorem' command in LaTeX is used to define new
% | theorem-like environments such as lemmas, remarks, definitions, propositions,
% | theorems, and so on.
% |
% | The 'newtheorem' command produces a numbered environment that is automatically
% | numbered sequentially. The optional argument 'parent counter' allows you to
% | specify the scope of this numbering. For example, 'theorem' is numbered
% | within the scope of 'chapter'.
% |
% | Parameters for newtheorem:
% | #1 (environment name) - The name of the new environment, e.g., 'theorem'.
% | #2 (displayed name) - The name displayed when the environment is used.
% | #3 (parent counter) - Optional argument specifying the counter within which the
% | new environment should be numbered, e.g., 'chapter'.

\newtheorem{theorem}{Theorem}[chapter]
\newtheorem{lemma}{Lemma}[chapter]
\tcolorboxenvironment{lemma}{
  enhanced jigsaw,
  leftrule=1pt,
  rightrule=1pt,
  boxrule=0pt,
}
\newtheorem{corollary}{Corollary}[theorem]

% | newtheorem* - Defines unnumbered theorem-like environments
% |
% | The 'newtheorem*' command, on the other hand, produces an unnumbered
% | environment. For instance, 'remark' is an unnumbered environment.
% |
% | Parameters for newtheorem*:
% | #1 (environment name) - The name of the new environment, e.g., 'remark'.
% | #2 (displayed name) - The name displayed when the environment is used.

\newtheorem*{remark}{Remark}

% Using tcolorbox to style the 'remark' environment
% | tcolorboxenvironment - applies tcolorbox styling to a LaTeX environment
% | 
% | This command is used to apply specific styling attributes to a given
% | LaTeX environment. In this case, the 'remark' environment is being styled
% | with specific border rules using tcolorbox.

\tcolorboxenvironment{remark}{
  enhanced jigsaw,
  leftrule=1pt,
  rightrule=1pt,
  boxrule=0pt,
}

% Using declaretheorem to define the 'definition' environment with thmtools
% | declaretheorem - Defines new theorem-like environments with thmtools
% |
% | The 'declaretheorem' command allows for more flexibility in defining
% | theorem-like environments. Here, 'definition' is defined with numbering
% | within the 'chapter' scope.
% |
% | The resulting environment will be styled using tcolorbox in the following lines.

\declaretheorem[numberwithin=chapter, name=Definition]{definition}

% Using tcolorbox to style the 'definition' environment

\tcolorboxenvironment{definition}{
  enhanced jigsaw,
  leftrule=1pt,
  rightrule=1pt,
  boxrule=0pt,
}

% --------------------
% STYLING PROOF LAYOUT
% --------------------

% | RenewDocumentEnvironment - modifies an existing environment
% |
% | This command is used to redefine the behavior of an existing LaTeX
% | environment. In this case, the 'proof' environment is being modified
% | to display the word "Proof." in italics at the beginning, and
% | to end with the qed symbol.

\RenewDocumentEnvironment{proof}{}{\emph{Proof.}}{\qed}
