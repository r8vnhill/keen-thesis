% | newtheorem - Defines new theorem-like environments
% | newtheorem* - Defines unnumbered theorem-like environments
% |
% | The 'newtheorem' and 'newtheorem*' commands in LaTeX are used to define new
% | theorem-like environments such as lemmas, remarks, definitions, propositions,
% | theorems, and so on.
% |
% | The 'newtheorem' command produces a numbered environment that is automatically
% | numbered sequentially. The optional argument 'parent counter' allows you to
% | specify the scope of this numbering. In the example, 'definition' is numbered
% | within the scope of 'chapter'.
% |
% | The 'newtheorem*' command, on the other hand, produces an unnumbered
% | environment. For instance, 'remark' is an unnumbered environment.
% |
% | Parameters for newtheorem:
% | #1 (environment name) - The name of the new environment, e.g., 'definition'.
% | #2 (displayed name) - The name displayed when the environment is used.
% | #3 (parent counter) - Optional argument specifying the counter within which the
% | new environment should be numbered, e.g., 'chapter'.
% |
% | Parameters for newtheorem*:
% | #1 (environment name) - The name of the new environment, e.g., 'remark'.
% | #2 (displayed name) - The name displayed when the environment is used.
\newtheorem{theorem}{Theorem}[chapter]
\newtheorem{lemma}{Lemma}[chapter]
\newtheorem{corollary}{Corollary}[theorem]
\newtheorem*{remark}{Remark}
\newtheorem{definition}{Definition}[chapter]

\RenewDocumentEnvironment{proof}{}{\emph{Proof.}}{\qed}
