\subsubsection{Evaluation}
\label{sec:bg:gp:repr_ev:eval}
  Now, we need to define how to evaluate the fitness of a program.
  Again, there are many ways to do this, but a common way is to use the
  \emph{mean squared error} (MSE) between the points and the program.

  \begin{definition}[Mean Squared Error]
  \label{def:mse}
    If a vector of \(n\) predictions is generated from a sample of \(n\) data
    points on all variables, and \(\mathbf{y}_i\) is the \(i\)-th observed value
    and \(\hat{\mathbf{y}}_i\) is the \(i\)-th prediction, then the MSE of the 
    predictor is a function \(\mathrm{MSE}:\: \mathbb{R}^n \times \mathbb{R}^n
    \to \mathbb{R}\) defined as:

    \begin{equation}
      \label{eq:mse}
      \mathrm{MSE}(\mathbf{y}, \hat{\mathbf{y}}) 
        = \frac{1}{n} \sum_{i=1}^{n} (\mathbf{y}_i - \hat{\mathbf{y}}_i)^2
    \end{equation}
  \end{definition}

  The MSE is a common measure of the quality of an estimator, used in many
  \textit{machine learning} problems.

  In our case, we will use the MSE to evaluate the fitness of a program.
  Consider a program \(\mathsf{P}\) and two sets of points, \(\mathbf{x}\) and 
  \(\mathbf{y}\), as outlined in 
  \vref{tab:bg:gp:repr_ev:points}.
  Suppose also that \(\mathsf{P}[\mathbf{x}]\) is the set of points generated
  by evaluating \(\mathsf{P}\) on the points of \(\mathbf{x}\), and that
  \(\mathsf{P}(x)\) is the result of evaluating \(\mathsf{P}\) on the point
  \(x\).
  Then, we can define the fitness of \(\mathsf{P}\) as:

  \begin{equation}
  \label{eq:bg:gp:repr_ev:fitness}
    \phi_\mathsf{P} = \mathrm{MSE}(\mathbf{y}, \mathsf{P}[\mathbf{x}])
      = \frac{1}{n} \sum_{i=1}^{n} (\mathbf{y}_i - \mathsf{P}(\mathbf{x}_i))^2
  \end{equation}

  This section elucidated the pivotal aspects of Genetic Programming (GP), 
  focusing on representation of individuals and their evaluation.
  Individuals, potential solutions to a problem, can be encoded in several ways
  including tree representation.
  A common statistical problem, symbolic regression, was illustrated using a set
  of points on a curve from a function involving both polynomial and
  trigonometric elements.
  By defining a set of functions and terminals, these points were represented as
  a tree.
  The fitness of an individual program was assessed using the Mean Squared Error
  (MSE) between the points on the curve generated by the function and the points
  produced by the program.
  Thus, in the context of this problem, the fittest individual is the one that
  minimizes this error, i.e., best fits the curve.
  The methods described here underscore the versatility and applicability of GP
  to various types of problems.
  This method of representation and evaluation provides a robust basis for
  generating evolving populations of programs. 
  In the next sections we will discuss the mechanisms for generating these
  populations and the evolutionary operators that act on them.
