\subsubsection{Mutation}
\label{sec:bg:gp:variation:mutation}
  Mutation, another crucial genetic operator, introduces new genetic material 
  into the population, thus maintaining genetic diversity and preventing 
  premature convergence to suboptimal solutions.
  In the context of GP, the mutation operator modifies a program in the
  population, while ensuring the resultant individual's syntactic correctness.

  In tree-based GP, a common form of mutation is the \emph{point 
  mutation}~\autocite{poliFieldGuideGenetic2008a,wilhelmstotterJeneticsJavaGenetica},
  which selects a random node from an individual and replaces it with a random
  primitive with the same arity.
  This operator is similar to the bit-flip mutation operator used on 
  \vref{sec:ga:var:mut}.
  As with bit-flip mutation, point mutation can also be applied with a certain
  probability to each node in an individual, meaning that more than one node
  can be mutated in a single individual.

  Suppose we mutate all the individuals in the population resulting from the
  crossover operation in \vref{sec:bg:gp:var:cx}, and that exactly
  one node is mutated in each individual.
  Let the selected nodes be \(\clubsuit = \sin\) in \(\mathbf{I}_2\),
  \(\spadesuit = 7\) in \(\mathbf{I}_3\), \(\heartsuit = 5\) in 
  \(\mathbf{O}_1\), and \(\diamondsuit = 2\) in \(\mathbf{O}_2\).
  Then, a possible result of applying the point mutation operator can be:

  \[
    M\!\left(
      \begin{bmatrix}
        I_2 \\ I_3 \\ O_1 \\ O_2
      \end{bmatrix}
    \right) = M\!\left(
      \begin{bmatrix}
        7 - (5 + \clubsuit(x)) \\
        \spadesuit + 2 \\
        \heartsuit \cdot 7 \\
        x^\diamondsuit - (5 + \sin(x))
      \end{bmatrix}
    \right) = \left(
      \begin{bmatrix}
        7 - (5 + \clubsuit'(x)) \\
        \spadesuit' + 2 \\
        \heartsuit' \cdot 7 \\
        x^{\diamondsuit'} - (5 + \sin(x))
      \end{bmatrix}
    \right) = \left(
      \begin{bmatrix}
        7 - (5 + \cos(x)) \\
        6 + 2 \\
        6 \cdot 7 \\
        x^3 - (5 + \sin(x))
      \end{bmatrix}
    \right)
  \]

  Here, \(\clubsuit'\), \(\spadesuit'\), \(\heartsuit'\), and \(\diamondsuit'\)
  are random primitives with the same arity as \(\clubsuit\), \(\spadesuit\),
  \(\heartsuit\), and \(\diamondsuit\), respectively.
  That being, \(\clubsuit' = \cos\), \(\spadesuit' = 6\), \(\heartsuit' = 6\),
  and \(\diamondsuit' = 3\).

  The fitness of the individuals in the population after applying the subtree 
  mutation operator is then evaluated.
  The results of this fitness evaluation are shown in 
  \vref{tab:bg:gp:var:mut:fitness}.
  A summary of the fitness of the population is presented in 
  \vref{tab:bg:gp:var:mut:fitness:summary}.

  \begin{table}[ht!]
    \centering
    \begin{tabular}{c|c|r}
      \multicolumn{3}{c}{\textbf{Generation 1}} \\
      \hline
      \hline
      \textbf{Individual} & \textbf{Program}        & \textbf{Fitness} \\
      \hline
      \(\mathbf{I}_2\)    & \(7 - (5 + \sin(x))\)   & 137.398\,836 \\
      \(\mathbf{I}_3\)    & \(7 + 2\)               & 331.924\,267 \\
      \(\mathbf{O}_1\)    & \(5 \cdot 7\)           & 1944.288\,127 \\
      \(\mathbf{O}_2\)    & \(x^2 - (5 + \sin(x))\) & 33.740\,766
    \end{tabular}
    \caption{Population after applying the subtree crossover operator.}
    \label{tab:bg:gp:variation:crossover:subtree:fitness}
  \end{table}

  \begin{table}[H]
    \centering
    \begin{tabular}{|c|r|c|}
      \hline
            & \textbf{Fitness} & \textbf{Individual}  \\
      \hline
      Best  & \(33.740\,766\) & \(O_2\) \\
      Worst & \(1944.288\,127\) & \(O_1\) \\
      \hline
      \hline
      Average & \multicolumn{2}{c|}{\(611.837\,999\)} \\
      \hline
      Standard deviation & \multicolumn{2}{c|}{\(896.858\,214\)} \\
      \hline
    \end{tabular}
    \caption{
      Fitness summary of the population after applying the subtree crossover
      operator.
    }
    \label{tab:bg:gp:variation:crossover:subtree:fitness:summary}
  \end{table}

Just like the crossover operator, mutation can also significantly influence 
the fitness and diversity of the population. By generating new structures 
in the population, mutation can help prevent stagnation and maintain diversity, 
thus avoiding premature convergence to suboptimal solutions.

In the next section, we discuss the combination of crossover and mutation 
operators and their role in navigating the search space effectively.
