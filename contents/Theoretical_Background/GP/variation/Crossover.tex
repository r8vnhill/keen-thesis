\subsubsection{Crossover}
\label{sec:bg:gp:variation:crossover}
  Since the individuals in GP are programs, all the variation operators must
  preserve the syntactic correctness of the individuals.
  In the case of crossover, this means that the resulting individuals must be
  syntactically correct programs.
  
  Because of this, the crossover operator used in \vref{sec:bg:ga:var:cx} is not
  applicable to GP\footnote{%
    Although this is true in most GP problems, there could be scenarios where
    the crossover operator used in GA could be used in GP, as seen in
    \vref{chap:beacon}.
  } since it doesn't consider the syntactic correctness of the individuals.

  The operator used in GP is dependant on the representation of the individuals.
  In the case of tree-based GP, the most basic crossover operator is the
  \emph{subtree crossover}\footnote{%
    See \vref{sec:keen:operators:crossover:single_node}.
  }.
  This operator selects a random node from each parent and swaps the subtrees
  rooted at those nodes.
  