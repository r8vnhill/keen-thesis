\subsection{The Generalization Problem in GP}
\label{sec:bg:gp:generalization}

  \subsubsection{Understanding Generalization}
    Generalization in Genetic Programming (GP) relates to a program's 
    proficiency in handling unseen data, not solely the data it evolved with. A 
    truly generalized program goes beyond rote memorization of training data. It 
    discerns underlying patterns, equipping it to make precise predictions or 
    decisions on new, unencountered instances.

  \subsubsection{Significance of Generalization in GP}
    Generalization holds a paramount position in GP. The overarching objective 
    of evolving programs is to address real-world challenges. A program's 
    utility diminishes if it can't generalize, limiting its effectiveness to 
    familiar data. Particularly in predictive modeling, the emphasis isn't on a 
    model's ability to recall past data, but on its precision with upcoming, 
    unpredictable data. Therefore, the act of generalizing surpasses basic 
    accuracy measures on training datasets, signaling a model's resilience and 
    practical relevance.

  \subsubsection{The Overfitting Dilemma}
    Overfitting stands as an adversary to generalization. A program succumbs to 
    overfitting when it molds excessively to training data, picking up on its 
    noise and irregularities rather than its broad trends. Such overfitted 
    programs may boast of stellar accuracy on training datasets but falter with 
    unseen data. Within the GP context, visualizing overfitting can be akin to 
    observing a program that's evolved into a convoluted form, brimming with 
    redundant branches or logic catering to the peculiarities of training data.

  \subsubsection{Addressing the Generalization Problem}
    The academic landscape has brought forth numerous techniques to counter the 
    generalization conundrum in GP, some such works include:

    \begin{itemize}
      \item Chen et al.'s approach hinged on feature selection tailored for 
        high-dimensional symbolic 
        regression~\autocite{chenFeatureSelectionImprove2017}.
      \item Kushchu advocated the adoption of multiple fitness functions, a 
        measure aimed at bolstering 
        generalization~\autocite{kushchuGeneticProgrammingEvolutionary2002}.
      \item Enhancing generalization through the Rademacher distribution was a 
        strategy put forth by Chen et 
        al.~\autocite{chenRademacherComplexityEnhancing2022}.
    \end{itemize}

    While this thesis will sidestep an in-depth exploration of the 
    generalization issue in GP, it's pivotal to acknowledge its profound 
    implications in GP and its standing as a fervently pursued research 
    frontier.
