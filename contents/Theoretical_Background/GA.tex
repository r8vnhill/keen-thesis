\section{Genetic Algorithms}
\label{sec:genetic_algorithms}
  Genetic Algorithms (GA)\footnote{
    Also known as Simple Genetic Algorithms (SGA) 
    \autocite{yuIntroductionEvolutionaryAlgorithms2010}, or Traditional Genetic Algorithms (TGA)
    \autocite{shiffmanNatureCode2012}.
  } \autocite{hollandAdaptationNaturalArtificial1992a,kozaGeneticProgrammingProgramming1992a,yuIntroductionEvolutionaryAlgorithms2010,shiffmanNatureCode2012}
  are a kind of EA where a \emph{population} of \emph{individuals}\footnote{
    See definition \ref{def:individual}.
  } representing candidate solutions to an optimization problem evolves towards better solutions.
  Each individual is defined by its location in the search space, which is called its
  \emph{genotype}\footnote{See definition \ref{def:genotype}}, and its fitness value, which is
  calculated with a \emph{fitness function}.

  \begin{definition}[Fitness function]
  \label{def:fitness_function}
    A \emph{fitness function} is a function \(f: S \rightarrow \mathbb{R}^n\), where \(S\) is the
    search space and \(n\) is the number of objectives of the optimization problem, that takes an
    individual as input and returns a vector of real numbers representing how close the individual 
    is to the global optimum of each objective.

    The fitness function is usually defined by the user of the algorithm, and it is problem
    dependent.
  \end{definition}

  A classical GA works as follows:

  \begin{algorithm}
    \caption{Genetic Algorithm}\label{alg:genetic_algorithm}
    \begin{algorithmic}
      \State \(\mathit{population} \gets \mathrm{initializePopulation()}\)
      \State \(\mathrm{evaluate}(\mathit{population})\)
      \Repeat
        \State \(\mathit{parents} \gets \mathrm{selectParents}(\mathit{population})\)
        \State \(\mathit{offspring} \gets \mathrm{crossover}(\mathit{parents})\)
        \State \(\mathrm{mutate}(\mathit{offspring})\)
        \State \(\mathrm{evaluate}(\mathit{offspring})\)
        \State \(\mathit{population} \gets \mathrm{selectSurvivors}(\mathit{population},
          \mathit{offspring})\)
      \Until{termination condition is met}
      \State \textbf{return} \(\mathrm{fittest}(\mathit{population})\)
    \end{algorithmic}
  \end{algorithm}