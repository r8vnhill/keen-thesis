\section{Evolutionary Algorithms}
\label{sec:evolutionary_algorithms}
  In the field of computational intelligence, \emph{evolutionary algorithms} 
  (EA)~\autocite{yuIntroductionEvolutionaryAlgorithms2010} are a family of algorithms inspired by the process of 
  natural selection. They are part of the larger field of \textit{evolutionary computation},\footnote{Which comprises both the practice and the study of evolutionary algorithms. See \vref{def:evolutionary_computation}.} which is a 
  subfield of \textit{metaheuristics}.\footnote{See \vref{def:metaheuristic}}

  EAs are algorithms that perform optimization or learning tasks by evolving solutions to a given problem via emergent 
  intelligence~\autocite{peterj.angelineGeneticProgrammingEmergent1994}. These tasks may range from function 
  optimization to machine learning or game AI development. EAs have three main characteristics:

  \begin{itemize}
    \item \textbf{\textsc{Population-based:}} These algorithms work with a \textbf{population of solutions}, allowing 
      them to explore the search space in \textit{parallel}.
    \item \textbf{\textsc{Fitness-oriented}:} The solutions in the population are evaluated using a \emph{fitness 
      function}, which is a \textbf{problem-dependent function} that assigns a value to each solution based on its 
      quality. The goal of the algorithm is to find the solution with the highest\footnote{
        In some cases, the goal is to minimize the fitness function, in which case the algorithm will aim to find the 
        solution with the lowest fitness.
      } fitness.
    \item \textbf{\textsc{Variation-driven}:} The candidate solutions are modified using \emph{genetic operators}, such 
      as mutation, crossover, and selection, to create new solutions. These operators are usually based on the 
      biological processes of \textit{mutation} and \textit{recombination}.
  \end{itemize}
  
  \begin{remark}
    \enquote{Parallel} in this context means that the algorithm is exploring multiple points in the search space in the
    same generation (iteration). It should not be confused with parallel computing, which is a technique used to speed
    up the execution of the algorithm by running it on multiple processors.

    Even though EAs are parallel in nature, they can be run on a single processor. Nevertheless, this parallel nature
    makes them a good candidate for parallel computing; in this thesis we will not explore the use of parallel computing
    to speed up the execution of certain stages of the algorithm.
  \end{remark}

  While these principles serve as the foundation for most EAs, it's important to note that some variants may prioritize 
  some principles over others, or introduce new principles. This diversity allows EAs to be adapted to a wide range of 
  problems and scenarios.
