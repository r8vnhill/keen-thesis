\section{Evolutionary Algorithms}
\label{sec:evolutionary_algorithms}
  In the field of computational intelligence, evolutionary algorithms (EA)
  \autocite{yuIntroductionEvolutionaryAlgorithms2010} are a family of algorithms inspired by the 
  process of natural selection. 
  They are part of the larger field of evolutionary 
  computation,\footnote{See definition \ref{def:evolutionary_computation}} which is a subfield of 
  metaheuristics.\footnote{See definition \ref{def:metaheuristic}}

  EAs are algorithms that perform optimization or learning tasks by evolving solutions to a given
  problem.
  They have 3 main characteristics:

  \begin{itemize}
    \item \textbf{Population-based:} this algorithms work with a population of solutions, this
      allows them to explore the search space in parallel.\footnote{
        Not to be confused with parallelization, which is a technique used to speed up the execution
        of an algorithm by running it in parallel in multiple processors.
        \enquote{Parallel} in this context means that the algorithm is exploring multiple points in
        the search space (Def. \ref{def:search_space}) at the same time.
      }
    \item \textbf{Fitness-oriented:} the solutions in the population are evaluated using a fitness
      function, which is a function that assigns a value to each solution based on how good it is.
      The goal of the algorithm is to find the solution with the highest\footnote{
        In some cases, the goal is to minimize the fitness function, in which case the algorithm
        will try to find the solution with the lowest fitness.
      } fitness.
    \item \textbf{Variation-driven:} the candidate solutions are modified using genetic operators
      to create new solutions.
      These operators are usually based on the biological processes of mutation and recombination.
  \end{itemize}
