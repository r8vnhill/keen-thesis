\subsection{Initialization}
\label{sec:genetic_algorithms:initialization}
  GA operates on a group of individuals called a \emph{population}.
  The algorithm designer must define the size of the population, and how to initialize it.
  The initialization process is usually random, but it can also be guided by some prior knowledge
  about the problem being solved.
  For example, if the problem is to find a solution to a maze, the population could be initialized
  with individuals that represent paths from the start to the end of the maze.
  This would speed up the search process, since the algorithm would not have to start from scratch.

  Once the population is initialized, the algorithm performs an evaluation of each individual in
  the population, and assigns a \emph{fitness value} to each individual.
  This is done in an effort to learn something about the problem, and to guide the search process
  towards better solutions.

  In the case of the \emph{One Max} problem, there is no prior knowledge about the problem, so the
  population via a blind search of the search space (in other words, the initialization is random).
  This is done by generating a random binary string of length \(n\) for each individual in the
  population.

  Let's assume that we have a population of size 4, and that the length of the binary strings is
  \(n = 4\). 

  The initialization process could generate the following individuals:\footnote{
    Since the nature of genetic algorithms is stochastic, the initialization process could generate
    different individuals each time the algorithm is run.
    For this example, we selected a specific set of individuals in a way that makes it easier to
    get a grasp of the algorithm.
  }

  \begin{table}[H]
    \label{tab:genetic_algorithms:initialization:population}
    \centering
    \begin{tabular}{c|c|c}
      \multicolumn{3}{c}{\textbf{Generation 0}} \\
      \hline
      \hline
      \textbf{Individual} & \textbf{Binary string} & \textbf{Fitness} \\
      \hline
      \(I_1\) & 1100 & 2 \\
      \(I_2\) & 0001 & 1 \\
      \(I_3\) & 0000 & 0 \\
      \(I_4\) & 0100 & 1 \\
    \end{tabular}
    \caption{Population of individuals in generation 0}
  \end{table}

  \begin{table}[H]
    \centering
    \begin{tabular}{|c|c|c|}
      \hline
      & \textbf{Fitness} & \textbf{Individual}  \\
      \hline
      Best & 2 & \(I_1\) \\
      Worst & 0 & \(I_3\) \\
      \hline
      \hline
      Average & \multicolumn{2}{c|}{1} \\
      \hline
      Standard deviation & \multicolumn{2}{c|}{0.817} \\
      \hline
    \end{tabular}
    \caption{Fitness of the individuals in generation 0}
    \label{tab:genetic_algorithms:initialization:population_fitness}
  \end{table}
  
  In the initialization phase of a genetic algorithm, we define and setup the population of 
  individuals to be used in the search process. 
  This population can be randomly generated or informed by some prior knowledge about the problem at 
  hand. 
  Each individual is evaluated to determine its fitness, guiding the algorithm's search for optimal 
  solutions. 
  In our \enquote{One Max} problem example, we initialized a population of four individuals with 
  binary strings of length \(n = 4\) and evaluated their fitness. 
  This setup marks the beginning of the evolutionary process, setting the stage for the subsequent 
  stages of selection (\vref{sec:background:genetic_algorithms:selection}), 
  crossover (\vref{sec:genetic_algorithms:variation:crossover}), and 
  mutation (\vref{sec:genetic_algorithms:variation:mutation}).
