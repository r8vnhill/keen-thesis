\subsection{Variation}
\label{sec:genetic_algorithms:variation}
  Variation is the process of creating new individuals from existing ones in the pursuit of 
  exploring the solution space.
  This is crucial in a Genetic Algorithm (GA) to avoid premature convergence to sub-optimal
  solutions.
  In a GA, variation is achieved by applying \emph{variation operators} to the individuals in the 
  population.
  The most common variation operators are \emph{crossover} and \emph{mutation}, which will be 
  explored in this section.

  \begin{Definition}[Variation operator]
    \label{def:variation_operator}
    A variation operator is used to create new individuals from existing ones.
    Formally, it is a variadic function represented as 
    
    \[
      \varphi : \mathbb{P} \times \mathbb{R} \times \cdots \to \mathbb{P};\; 
      (P, \rho, \dots) \mapsto \varphi(P, \rho, \dots)
    \]
    
    where:

    \begin{itemize}
      \item \(\mathbb{P}\) is the set of all possible populations,
      \item \(\mathbb{R}\) is the set of real numbers,
      \item \(P\) is the population to be varied,
      \item \(\rho\) is the probability of applying the operator to an individual in the population.
    \end{itemize}

    The additional arguments depend on the specific implementation of the variation operator.
    The role of these arguments will be clarified in \vref{sec:keen:operators}.
  \end{Definition}

  \subimport{./}{Crossover.tex}
  \subimport{./}{Mutation.tex}
