\subsubsection{Mutation}
\label{sec:genetic_algorithms:variation:mutation}
  One problem with the crossover operator is that it can only produce individuals that are similar
  to the ones in the current population.
  This is because the crossover operator produces output by performing a recombination of the
  genetic material of a population.
  In practice, this usually means that the crossover operation is prone to premature convergence,
  for example, when the population has converged to a local optimum.
  This is the case with problems that have a large number of local optima, such as with the
  \emph{Rastrigin function} optimization.\footnote{
    Rastrigin, L. A. \enquote{Systems of extremal control.} Mir, Moscow (1974).
  }

%