\subsubsection{Mutation}
\label{sec:bg:ga:var:mut}
  While the crossover operator effectively recombines existing genetic 
  material, it is limited by the genetic diversity already present in the 
  population.
  This can sometimes lead to premature convergence, especially for complex 
  problems characterized by numerous local optima, like the \emph{Rastrigin function}.\footnote{Refer to \vref{sec:test_functions:rastrigin}.}

  To combat this and infuse fresh \textit{diversity}, the \textit{mutation} 
  operator is employed.
  It introduces small, probabilistic changes to the genetic makeup of 
  individuals.

  \begin{definition}[Mutation operator]
  \label{def:mutation_operator}
    The mutation operator introduces variations in an individual's genetic 
    material based on a predefined probability, resulting in a new population. 
    Formally, a mutation operator can be represented as:

    \[ 
      M :\: \mathbb{P} \times [0,\, 1] \times \cdots \to \mathbb{P};\; 
      (P,\, \mu,\, \dots) \mapsto M(P,\, \mu,\, \dots) 
    \]

    Where:

    \begin{itemize}
      \item \(\mathbb{P}\) denotes the set of all possible populations.
      \item \(\mathbb{R}\) signifies the set of real numbers.
      \item \(P\) stands for the current population.
      \item \(\mu\) indicates the \textit{mutation rate} -- the chance of an 
        individual undergoing mutation.
    \end{itemize}

    Additional parameters vary depending on the specific mutation operator in   
    play.
  \end{definition}

  For example, the \enquote{One Max} problem might employ a \emph{bit-flip} 
  mutation.\footnote{
    For more details, refer to \vref{sec:keen:op:mut:bit_flip}.
  }
  This mutation scans every gene in an individual, flipping it based on a 
  predetermined probability.

  Let's say we apply this mutation with a rate of \(\mu = 1\) to the population 
  post-crossover (as discussed in \vref{sec:bg:ga:var:cx}).
  The resulting mutated population, \(\mathbf{O}'\), is illustrated in 
  \vref{tab:bg:ga:var:cx:mutation}.

  
  \begin{table}[ht!]
    \centering
    \begin{tabular}{|c|c|c|c|}
      \multicolumn{4}{c}{\textbf{Generation 0} \(\to\) \textbf{Generation 1}} \\
      \hline
      \hline
      \(\mathbf{I}\) & \(\Phi_\mathbf{I}\) & \(\mathbf{O}'\) & \(\Phi_\mathbf{O}\) \\
      \hline
      \Gape[2pt][2pt]{\(\begin{bmatrix} 0000 \\ 1101 \\ 0101 \end{bmatrix}\)}
        & \(\begin{bmatrix} 0 \\ 3 \\ 2 \end{bmatrix}\)
        & \(\begin{bmatrix} 1111 \\ 0010 \\ 1010 \end{bmatrix}\) 
        & \(\begin{bmatrix} 4 \\ 1 \\ 2 \end{bmatrix}\) \\[1em]
      \hline
    \end{tabular}
    \caption{
      Illustration of the \emph{bit-flip} mutation operator applied to the 
      population resulting from the crossover operation in 
      \vref{sec:bg:ga:var:cx}.
    }
    \label{tab:bg:ga:var:cx:mutation}
  \end{table}

  By analyzing the resulting offspring, as highlighted in 
  \vref{tab:bg:ga:var:mut:2}, we can observe the mutation's role in enhancing 
  diversity.
  For instance, no member of the original population had a 1 in the third 
  position, rendering crossover incapable of producing such a gene 
  configuration.
  However, post-mutation, three individuals exhibit this trait.

  
  \begin{table}[ht!]
    \centering
    \begin{tabular}{c | c | c }
      \multicolumn{3}{c}{\textbf{Generation 1}} \\
      \hline
      \hline
      \textbf{Individual} & \textbf{Binary String}  & \textbf{Fitness} \\
      \hline
      \(I_2\)             & \(0001\)                & 1 \\
      \(O'_1\)            & \(1111\)                & 4 \\
      \(O'_2\)            & \(0010\)                & 1 \\
      \(O'_3\)            & \(1010\)                & 2
    \end{tabular}
    \caption{
      Population after applying the \emph{bit-flip} mutation operator to the 
      population resulting from the crossover operation in 
      \vref{sec:bg:ga:var:cx}.
    }
    \label{tab:bg:ga:var:mut:2}
  \end{table}

  \begin{table}[H]
    \centering
    \begin{tabular}{|c|c|c|}
      \hline
            & \textbf{Fitness} & \textbf{Individual}  \\
      \hline
      Best  & 4 & \(O'_1\) \\
      Worst & 0 & \((I_2,\, O'_2)\) \\
      \hline
      \hline
      Average & \multicolumn{2}{c|}{2} \\
      \hline
      Standard deviation & \multicolumn{2}{c|}{1.414} \\
      \hline
    \end{tabular}
    \caption{
      Fitness of the population after applying the \emph{bit-flip} mutation 
      operator to the population resulting from the crossover operation in
      \vref{sec:bg:ga:var:cx}.
    }
    \label{tab:bg:ga:var:mut:fitness}
  \end{table}

  In summary, the mutation operator serves a pivotal role in genetic 
  algorithms.
  It rejuvenates population diversity, mitigating early convergence to 
  suboptimal solutions.
  By fostering exploration of the search landscape, it allows potentially 
  superior traits to surface.
  Nonetheless, the mutation rate's calibration is a balancing act: excessively 
  high rates might destabilize favorable traits, while overly conservative 
  rates might inadequately deter premature convergence.
  The specific mutation operator and its associated rate significantly 
  influence the genetic algorithm's exploratory efficiency.

  Having concluded the variation phase, we can transition to the subsequent 
  stage of the genetic algorithm.
