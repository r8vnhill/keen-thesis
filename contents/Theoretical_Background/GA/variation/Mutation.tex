\subsubsection{Mutation}
\label{sec:genetic_algorithms:variation:mutation}
  One limitation of the crossover operator is its reliance on existing genetic 
  material in the population.

  This constraint can lead to premature convergence, particularly for problems 
  with numerous local optima such as the \emph{Rastrigin function} 
  optimization.\footnote{See \vref{sec:test_functions:rastrigin}.}

  To counteract this and introduce \textit{diversity} in the population, we use 
  the \textit{mutation} operator.

  This operator alters the genetic material of an individual within the 
  population according to a specific probability.

  \begin{definition}[Mutation operator]
  \label{def:mutation_operator}
    A mutation operator is a function that alters the genetic material of 
    individuals within a population based on a certain probability, thereby 
    producing a new population.

    Formally, a mutation operator is a variadic function 
    
    \[
      M :\: \mathbb{P} \times \mathbb{R} \times \cdots \to \mathbb{P};\; 
        (P,\, \mu,\, \dots) \mapsto M(P,\, \mu,\, \dots)
    \]
    
    where:

    \begin{itemize}
      \item \(\mathbb{P}\) represents the set of all possible populations;
      \item \(\mathbb{R}\) represents the set of real numbers;
      \item \(P\) is the current population;
      \item \(\mu\) represents the \textit{mutation rate}—the probability that 
        an individual in the population will undergo mutation.
    \end{itemize}

    The other arguments are specific to the mutation operator being used.
  \end{definition}

  For instance, in the \enquote{One Max} problem, we can use a \emph{bit-flip} 
  mutation.\footnote{See \vref{sec:keen:operators:mutation:bit_flip}.} 
  This operator scans each gene in an individual and substitutes it with its 
  complement according to a predetermined probability.

  Suppose we set the \emph{mutation rate} \(\mu = 1\), and apply the mutation 
  operator to the population resulting from the crossover operation described in
  \vref{sec:bg:ga:var:cx}.
  As shown in \vref{tab:genetic_algorithms:variation:mutation:1}, the resulting 
  population would be 
  \(\mathbf{O} = \{(1111,\, 4), (0010,\, 1), (1010,\, 3)\}\).

  
  \begin{table}[ht!]
    \centering
    \begin{tabular}{|c|c|c|c|}
      \multicolumn{4}{c}{\textbf{Generation 0} \(\to\) \textbf{Generation 1}} \\
      \hline
      \hline
      \(\mathbf{I}\) & \(\Phi_\mathbf{I}\) & \(\mathbf{O}\) & \(\Phi_\mathbf{O}\) \\
      \hline
      \Gape[2pt][2pt]{\(\begin{bmatrix} 0000 \\ 1101 \\ 0101 \end{bmatrix}\)}
        & \(\begin{bmatrix} 0 \\ 3 \\ 2 \end{bmatrix}\)
        & \(\begin{bmatrix} 1111 \\ 0010 \\ 1010 \end{bmatrix}\) 
        & \(\begin{bmatrix} 4 \\ 1 \\ 2 \end{bmatrix}\) \\[1em]
      \hline
    \end{tabular}
    \caption{
      Illustration of the \emph{bit-flip} mutation operator applied to the 
      population resulting from the crossover operation in 
      \vref{sec:bg:ga:var:cx}.
    }
    \label{tab:genetic_algorithms:variation:crossover:mutation}
  \end{table}

  If the mutated offspring were used to generate the next population, as shown 
  in \vref{tab:genetic_algorithms:variation:mutation:2}, you can observe the 
  increased diversity.

  \begin{table}[ht!]
    \centering
    \begin{tabular}{c | c | c }
      \multicolumn{3}{c}{\textbf{Generation 1}} \\
      \hline
      \hline
      \textbf{Individual} & \textbf{Binary String}  & \textbf{Fitness} \\
      \hline
      \(I_2\)             & \(0001\)                & 1 \\
      \(O'_1\)            & \(1111\)                & 4 \\
      \(O'_2\)            & \(0010\)                & 1 \\
      \(O'_3\)            & \(1010\)                & 2
    \end{tabular}
    \caption{
      Population after applying the \emph{bit-flip} mutation operator to the 
      population resulting from the crossover operation in 
      \vref{sec:bg:ga:var:cx}.
    }
    \label{tab:genetic_algorithms:variation:mutation:2}
  \end{table}

  \begin{table}[H]
    \centering
    \begin{tabular}{|c|c|c|}
      \hline
            & \textbf{Fitness} & \textbf{Individual}  \\
      \hline
      Best  & 4 & \(O'_1\) \\
      Worst & 0 & \((I_2,\, O'_2)\) \\
      \hline
      \hline
      Average & \multicolumn{2}{c|}{2} \\
      \hline
      Standard deviation & \multicolumn{2}{c|}{1.414} \\
      \hline
    \end{tabular}
    \caption{
      Fitness of the population after applying the \emph{bit-flip} mutation 
      operator to the population resulting from the crossover operation in
      \vref{sec:bg:ga:var:cx}.
    }
    \label{tab:genetic_algorithms:variation:mutation:fitness}
  \end{table}
 
  Clearly, the mutation operator has introduced diversity into the population.
  In the original population, no individual had a 1\footnote{
    \(P = \{
      11\textbf{{\color{red}0}}0,\, 
      00\textbf{{\color{red}0}}1,\, 
      00\textbf{{\color{red}0}}0,\, 
      01\textbf{{\color{red}0}}0
    \}\)
  } in the third position.
  Therefore, the crossover operator could never produce an individual with a 1 
  in that position.
  But the mutation operator has introduced three individuals with a 1 in the 
  third position.\footnote{
    \(P' = \{
      0001,\, 
      11\textbf{{\color{red}1}}1,\, 
      00\textbf{{\color{red}1}}0,\, 
      10\textbf{{\color{red}1}}0
    \}\)
  } 

  
  In conclusion, the mutation operator plays a crucial role in genetic 
  algorithms by introducing diversity into the population and preventing 
  premature convergence to local optima.
  It facilitates a more thorough exploration of the search space, allowing new 
  and potentially beneficial traits to emerge.
  However, it's essential to note that a high mutation rate might disrupt 
  advantageous traits, while a low rate might not sufficiently prevent premature
  convergence.
  The specific mutation operator and the mutation rate used are crucial factors 
  in shaping the genetic algorithm's search process.

  With this, the variation process is complete, and we can proceed to the next 
  step of the genetic algorithm.
%