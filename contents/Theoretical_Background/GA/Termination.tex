\subsection{Termination}
\label{sec:genetic_algorithms:termination}
  After each generation -- when a new population is fully created -- the
  genetic algorithm verifies if the termination criteria have been met.
  If so, the algorithm terminates and returns the best individual identified.
  Otherwise, the process continues to the next generation.

  Consider a scenario where the termination criterion is defined as the 
  discovery of an individual possessing the maximum number of ones, represented 
  as $1111$.
  This would correspond to a condition where $\phi_\mathbf{G} = 4$.

  Recall that we found the individual $1111$ after applying the variation 
  operators to the population. 
  As a result, the termination criterion is met and the genetic algorithm 
  concludes its process.

  It's worth noting that not all search space has been explored, as demonstrated 
  in \vref{tab:genetic_algorithms:termination:search_space}.
  The algorithm's fitness-oriented search strategy means it performs a guided, 
  rather than exhaustive, search. 
  However, the increasing fitness of the population's individuals across 
  generations indicates convergence towards an optimal solution.
  
  \begin{table}[ht!]
    \centering
    \begin{tabular}{|c||c|c|c|c|}
      \hline
            & $00$ & $01$ & $10$ & $11$ \\
      \hline
      \hline
      $00$  & \cellcolor{darkgray}  & \cellcolor{darkgray} 
        & \cellcolor{darkgray}  & \\
      \hline
      $01$  & \cellcolor{darkgray}  & \cellcolor{darkgray}  
        &                       & \\
      \hline
      $10$  &                       &                       
        & \cellcolor{darkgray}  & \\
      \hline
      $11$  & \cellcolor{darkgray}  & \cellcolor{darkgray}  
        &                       & \cellcolor{darkgray} \\
      \hline
    \end{tabular}
    \caption{
      Candidates from the search space that were explored by the genetic 
      algorithm.
      Cells that are coloured in dark gray represent candidates that were
      explored by the genetic algorithm.
      Each individual is defined by the row and column that it occupies in the
      search space, where the row represents the first 2 bits of the individual
      and the column represents the last 2 bits of the individual; e.g. the
      individual $0001$ is located in the first row and second column of the
      table.
    }
    \label{tab:genetic_algorithms:termination:search_space}
  \end{table}

  For small search spaces like in our example, the distinction between this 
  algorithm and a purely random search may seem minimal.
  But for larger search spaces, as explored later in this thesis, the difference
  becomes highly significant.

  It's important to underline that genetic algorithms, being stochastic in 
  nature, do not guarantee discovery of the optimal solution.
  Their effectiveness depends on various factors such as the fitness function, 
  the representation scheme, the variation operators, and the selection 
  strategy.
  These components and their impact on performance across different problems
  will be thoroughly examined in this thesis.

  In summary, the termination phase of the genetic algorithm represents a 
  crucial step in determining the overall process outcome.
  By utilizing a targeted termination criterion -- such as the discovery of an
  individual with the highest possible fitness score -- the algorithm
  effectively navigates the search space.
  While not exhaustive in its exploration, the algorithm uses a fitness-oriented
  strategy to guide its trajectory towards an optimal solution.
  It's essential to recognize the inherent limitations of genetic algorithms due
  to their stochastic nature.
  Despite these, their potential to outperform random searches, especially in
  large search spaces, is considerable.
  However, success relies heavily on choosing appropriate parameters and
  procedures, a topic to be explored in-depth in subsequent sections of this
  thesis.
