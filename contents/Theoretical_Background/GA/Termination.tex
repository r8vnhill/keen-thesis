\subsection{Termination}
\label{sec:genetic_algorithms:termination}
  The genetic algorithm assesses the termination criteria after generating each 
  new population.
  If met, the algorithm ends and outputs the best-found individual.
  Otherwise, the generational cycle repeats.

  Suppose our termination criterion is the identification of an individual with 
  all ones, represented as 1111.
  This would mean the fitness function, \(\phi_\mathbf{G}\), attains the value 
  4.

  During our exploration, after applying the variation operators, we indeed 
  discovered the individual 1111.
  This satisfies our termination criterion, prompting the algorithm to conclude 
  its operation.

  Yet, as illustrated in \vref{tab:bg:ga:termination:search_space}, the 
  algorithm has not scoured the entire search space.
  Instead, the genetic algorithm prioritizes a fitness-guided exploration over 
  a complete one.
  Still, witnessing the increasing fitness of individuals over generations 
  signals a trend towards an optimal or near-optimal solution.

  \begin{table}[ht!]
    \centering
    \begin{tabular}{|c||c|c|c|c|}
      \hline
            & $00$ & $01$ & $10$ & $11$ \\
      \hline
      \hline
      $00$  & \cellcolor{darkgray}  & \cellcolor{darkgray} 
        & \cellcolor{darkgray}  & \\
      \hline
      $01$  & \cellcolor{darkgray}  & \cellcolor{darkgray}  
        &                       & \\
      \hline
      $10$  &                       &                       
        & \cellcolor{darkgray}  & \\
      \hline
      $11$  & \cellcolor{darkgray}  & \cellcolor{darkgray}  
        &                       & \cellcolor{darkgray} \\
      \hline
    \end{tabular}
    \caption{
      A map of the search space explored by the genetic algorithm.
      Dark gray cells denote the candidates the algorithm reviewed.
      The position of each individual corresponds to its binary representation, 
      using the row for the first two bits and the column for the last two.
    }
    \label{tab:bg:ga:termination:search_space}
  \end{table}

  In small search arenas, the distinction between a genetic algorithm and 
  random searches might appear negligible.
  However, as the search space expands, which we delve into later in this 
  thesis, the difference becomes substantial.

  Despite their stochastic nature, genetic algorithms do not always promise 
  optimal outcomes.
  Their efficacy hinges on various components, including the fitness function, 
  representation, variation operators, and selection techniques.
  This thesis delves deeper into these elements and analyzes their performance 
  across diverse problems.

  In essence, the termination phase is pivotal in dictating the final outcome 
  of the genetic algorithm.
  By setting a clear termination condition, like achieving the highest fitness 
  score, the algorithm efficiently navigates the search terrain.
  Although it might not exhaustively search, it employs a fitness-centric 
  strategy to steer closer to optimal solutions.
  It's vital to acknowledge the intrinsic limitations of genetic algorithms. 
  Yet, when configured right, their capability to overshadow random searches, 
  especially in vast spaces, is undeniable.
  We'll further explore this subject in the subsequent segments of this thesis.
