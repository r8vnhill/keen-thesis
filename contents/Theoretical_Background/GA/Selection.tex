\subsection{Selection}
\label{sec:genetic_algorithms:selection}
  After the initialization, the GA enters the main loop.
  The algorithm imitates by granting the fitter individuals higher opportunities to breed and to
  survive to the next generation.

  Suppose that we have a population \(P\) of \(N\) individuals, each with a fitness value \(f_i\),
  where \(i \in \{1, \ldots, N\}\).
  Let \(\sigma\) be the survival rate,\footnote{Also known as elitism.} that is, the proportion of 
  individuals that will survive to the next generation.
  Then, the GA will select \(\lfloor\sigma N\rfloor\) individuals to survive (unmodified) to the 
  next generation, and \(\lceil(1 - \sigma)N\rceil\) individuals to be replaced by the 
  offspring.\footnote{Floor and ceiling functions are interchangeable in this context, since their
  only purpose is to ensure that the number of individuals selected is \(N\).}

  The selection is then performed with a \emph{selection operator} (\(\Sigma\)) that takes the 
  population and the number of individuals to select as parameters, and returns a subset of the
  population with the selected individuals.
  The selection operator is usually implemented as a \emph{stochastic} operator, that is, it
  introduces some randomness in the selection process.

  To illustrate this, we will use a \textit{roulette wheel} selection operator\footnote{
    See \vref{sec:keen:operators:selection:roulette_wheel}.
  } with a population of the 4 individuals shown in the previous section and a survival rate of 
  0.25.
  In a roulette wheel selection, each individual is assigned a selection probability proportional
  to its fitness value.
  Typically, the selection probability of an individual \(i\) is calculated as follows:

  \begin{equation}
    \label{eq:selection_probability}
    p_i = \frac{f_i}{\sum_{j=1}^{N}f_j}
  \end{equation}

  In our example, the selection probabilities are shown in \vref{tab:selection_probabilities}.

  \begin{table}[ht!]
    \centering
    \begin{tabular}{|l|r|r|}
      \hline
      Individual  & Fitness & Selection Probability \\
      \hline
      \(I_1\)     & 2       & 0.5  \\
      \(I_2\)     & 1       & 0.25   \\
      \(I_3\)     & 0       & 0.0   \\
      \(I_4\)     & 1       & 0.25  \\
      \hline
    \end{tabular}
    \caption{Selection probabilities for the individuals in the example population.}
    \label{tab:selection_probabilities}
  \end{table}

  The selection operator then selects individuals at random, with a probability equal to their
  selection probability.
  Let's say that the selection operator selects \(I_2\) as a survivor to the next generation, then
  \(I_1\), \(I_3\), and \(I_4\) will be replaced by the offspring.

  This section has introduced the concept of selection in GAs.
  This topic will be further explored in \vref{sec:keen:operators:selection}.
  Next, we will explore the variation operators, which are responsible for creating the offspring
  that will replace the individuals that were not selected to survive to the next generation.
% End: contents\Theoretical_Background\GA\Selection.tex