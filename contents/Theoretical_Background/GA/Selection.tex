\subsection{Selection}
\label{sec:background:genetic_algorithms:selection}
  Once initialization is complete, the Genetic Algorithm (GA) enters its main loop, where the core 
  evolutionary processes take place.
  In the GA, a mechanism that simulates natural selection operates, providing fitter individuals 
  with higher chances of survival and breeding opportunities.

  Suppose that we have a population \(P\) of \(N\) individuals, each with a fitness value \(f_i\), 
  where \(i \in \{1, \ldots, N\}\).
  Let \(\sigma\) be the survival rate, a parameter controlling the degree of elitism.
  This is the proportion of individuals that will survive (unmodified) to the next generation.
  The GA will then select \(\lfloor\sigma N\rfloor\) individuals to survive to the next generation,
  and \(\lceil(1 - \sigma)N\rceil\) individuals to be replaced by the offspring.\footnote{
    The sole purpose of employing both \textit{floor} and \textit{ceiling} functions is to guarantee
    that the total number of individuals chosen for survival and replacement equals \(N\), which
    makes these functions interchangeable in this context.
  }

  \begin{definition}[Selection operator]
    \label{def:selection_operator}
      An operator used to select individuals from a population.

      Formally, a selection operator is a function 
      \[
        \Sigma : \mathbb{P} \times \mathbb{N} \times \cdots \to \mathbb{P};
        (P, n, \dots) \mapsto \Sigma(P, n, \dots)
      \]
      where: 
      
      \begin{itemize}
        \item \(\mathbb{P}\) is the set of populations;
        \item \(\mathbb{N}\) is the set of positive natural numbers;
        \item \(P\) is a population;
        \item \(n\) is the number of individuals to select from \(P\);
        \item \(\Sigma(P,\, n)\) is the population of \(n\) individuals selected
          from \(P\).
      \end{itemize}
  \end{definition}

  The selection operator is typically implemented as a \emph{stochastic} operator, introducing some 
  randomness into the selection process.

  As an illustration, consider a \textit{roulette wheel} selection operator\footnote{
    See \vref{sec:keen:operators:selection:roulette_wheel} for a detailed description of the 
    roulette wheel selection operator.
  } applied to a population of four individuals with a survival rate of 0.25.
  In this selection scheme, each individual is assigned a selection probability proportional to its 
  fitness value (assuming higher fitness is better).
  The selection probability of an individual \(i\) is calculated as follows:

  \begin{equation}
    \label{eq:selection_probability}
    p_i = \frac{f_i}{\sum_{j=1}^{N}f_j}
  \end{equation}

  In our example, the selection probabilities are detailed in \vref{tab:selection_probabilities}.

  \begin{table}[ht!]
    \centering
    \begin{tabular}{|l|r|r|}
      \hline
      Individual  & Fitness & Selection Probability \\
      \hline
      \(I_1\)     & 2       & 50\%  \\
      \(I_2\)     & 1       & 25\%  \\
      \(I_3\)     & 0       & 0\%   \\
      \(I_4\)     & 1       & 25\%  \\
      \hline
    \end{tabular}
    \caption{Selection probabilities for the individuals in the example population.}
    \label{tab:selection_probabilities}
  \end{table}

  The selection operator then selects individuals at random, each with a probability equal to their 
  selection probability. 
  Suppose \(I_2\) is selected to survive to the next generation, then \(I_1\), \(I_3\), and \(I_4\) 
  will be replaced by the offspring.

  This section has introduced the concept of selection in GAs, which will be explored further in 
  \vref{sec:keen:operators:selection}. 
  Next, we will delve into variation operators responsible for generating the offspring that will 
  replace the individuals not selected to survive to the next generation.
%