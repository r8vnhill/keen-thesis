\subsection{Selection}
\label{sec:bg:ga:select}
  Following the initialization phase, a genetic algorithm (GA) progresses 
  through its main evolutionary cycle.
  Key to this cycle is the process of selection, which mirrors natural selection 
  by favoring individuals with higher fitness for reproduction.

  Given a population \(P\) comprising \(N\) individuals, each identified by 
  their fitness value \(\phi_i\) (where \(i \in \{1,\, \dots,\, N\}\)), the 
  survival rate \(\sigma\) helps control elitism.\footnote{
    For more on elitism, refer to \vref{def:elitism}.
  } This rate determines the portion of the population that proceeds to the next generation without change.
  Specifically, \(\lfloor\sigma N\rfloor\) individuals continue, while the 
  remaining \(\lceil(1 - \sigma)N\rceil\) are replaced by their
  descendants.\footnote{
    Using both the \textit{floor} and \textit{ceiling} functions ensures a 
    consistent total population of \(N\).
  }

  For the remaining of this document, we will use the following notation to represent functions with named parameters:

  \[
    f(x_1: \tau_1, \dots, x_n: \tau_n) \to \tau_r
  \]
  
  where:

  \begin{itemize}
    \item \(f\): Function name.
    \item \(x_i\): Parameter name.
    \item \(\tau_i\): Parameter type.
    \item \(\tau_r\): Return type.
  \end{itemize}
  
  \begin{definition}[Selection Operator]
  \label{def:selection_operator}
    A tool for choosing specific members from a population, formally expressed as:

    \[
      \Sigma(P: \mathbb{P},\, n: \mathbb{N},\, \dots) \to \mathbb{P}
    \]

    Parameters include:
    
    \begin{itemize}
      \item \(\mathbb{P}\): Set of possible populations.
      \item \(\mathbb{N}\): Set of natural numbers.
      \item \(P\): A given population.
      \item \(n\): Number of selections from \(P\).
    \end{itemize}
  \end{definition}

  This selection operator usually integrates randomness, bringing some unpredictability to the process.
  As an exemplification, let's consider the \textit{roulette wheel} selection operator.\footnote{
    Refer to \vref{sec:keen:op:select:roulette} for details.
  } Here, individuals are assigned selection probabilities according to their fitness, as expressed by:

  \begin{equation}
    \label{eq:selection_probability}
    \rho_\Sigma(i) = \frac{\phi_i}{\sum_{j=1}^{N}\phi_j}
  \end{equation}

  \begin{table}[ht!]
    \centering
    \begin{tabular}{|l|r|r|}
      \hline
      Individual & Fitness & Selection Probability \\
      \hline
      \(I_1\)    & 2       & 50\% \\
      \(I_2\)    & 1       & 25\% \\
      \(I_3\)    & 0       & 0\%  \\
      \(I_4\)    & 1       & 25\% \\
      \hline
    \end{tabular}
    \caption{Probabilities of selection for our sample population.}
    \label{tab:selection_probabilities}
  \end{table}

  Following this, individuals are probabilistically selected for survival.
  For instance, if \(I_2\) persists, the remaining \(I_1\), \(I_3\), and \(I_4\) 
  yield their places to the offspring --given a survival rate of 25\%.

  Concluding this introduction to selection in GAs, the subsequent sections will 
  investigate this process further and introduce the variation operators crucial 
  for offspring generation.
  