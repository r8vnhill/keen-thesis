\section{Genetic Algorithms}
\label{sec:genetic_algorithms}
  Genetic Algorithms (GA)\footnote{
    Also known as Simple Genetic Algorithms (SGA) 
    \autocite{yuIntroductionEvolutionaryAlgorithms2010}, or Traditional Genetic Algorithms (TGA)
    \autocite{shiffmanNatureCode2012}.
  }~\autocite{hollandAdaptationNaturalArtificial1992a,kozaGeneticProgrammingProgramming1992a,yuIntroductionEvolutionaryAlgorithms2010,shiffmanNatureCode2012}
  are a kind of EA where a \emph{population} of \emph{individuals}\footnote{
    See \vref{def:individual}.
  } representing candidate solutions to an optimization problem evolves towards better solutions.
  Each individual is defined by its location in the search space, which is called its
  \emph{genotype}\footnote{See \vref{def:genotype}}, and its fitness value, which is
  calculated with a \emph{fitness function}.

  A classical GA works as follows:

  \begin{algorithm}
    \caption{Genetic Algorithm}\label{alg:genetic_algorithm}
    \begin{algorithmic}
      \State \(\mathit{population} \gets \mathrm{initializePopulation()}\)
      \State \(\mathrm{evaluate}(\mathit{population})\)
      \Repeat
        \State \(\mathit{parents} \gets \mathrm{selectParents}(\mathit{population})\)
        \State \(\mathrm{alter}(\mathit{offspring})\)
        \State \(\mathit{population} \gets \mathrm{selectSurvivors}(\mathit{population},
          \mathit{offspring})\)
      \Until{termination condition is met}
      \State \textbf{return} \(\mathrm{fittest}(\mathit{population})\)
    \end{algorithmic}
  \end{algorithm}

  Where \(\mathrm{initializePopulation()}\) creates a random population of individuals, and
  \(\mathrm{evaluate}(\mathit{population})\) calculates the fitness of each individual in the
  population.

  The algorithm then repeats the following steps until a termination condition is met:

  \begin{enumerate}
    \item \(\mathrm{selectParents}(\mathit{population})\) selects a subset of individuals from the
      population to be the parents of the next generation.
    \item \(\mathrm{alter}(\mathit{offspring})\) alters the offspring to introduce new genetic
      material into the population.
    \item \(\mathrm{evaluate}(\mathit{offspring})\) calculates the fitness of each individual in the
      offspring.
    \item \(\mathrm{selectSurvivors}(\mathit{population}, \mathit{offspring})\) selects the
      individuals that will survive to the next generation.
  \end{enumerate}

  Finally, the algorithm returns the fittest individual in the population.

  The actual implementation of each of these steps depends on the problem being solved, and there
  are many different ways to implement each of them.

  \subimport{./}{Representation.tex}
  \subimport{./}{Initialization.tex}
  \subimport{./}{Selection.tex}
  \subimport{./variation/}{_variation.index.tex}
