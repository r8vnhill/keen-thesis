\subsection{Representation and Evaluation}
\label{ssec:genetic_algorithms:representation}
  One of the most important aspects of a GA is the representation of the individuals.
  The representation is the encoding of potential solutions to the problem into a form that can be
  manipulated by the algorithm.
  This defines the search space of the algorithm, and it is one of the main factors that
  determines the performance of the algorithm.

  The most general representation of an individual is a matrix of genes\footnote{
    See definition \ref{def:gene}.
  } called the \emph{genotype} of the individual, where each column of the matrix is called a
  \emph{chromosome}\footnote{
    See definition \ref{def:chromosome}.
  }.
  
  \begin{definition}[Cardinality of the search space]
  \label{def:cardinality_of_the_search_space}
    The \emph{cardinality of the search space} is the number of different individuals that 
    can be represented by the encoding.

    Formally, given a vector of alphabets \((\mathcal{A}_1, \mathcal{A}_2, \dots, \mathcal{A}_n)\), 
    and a representation \(\mathbf{G}\) with \(n\) chromosomes of length \((m_1, m_2, \dots, 
    m_n)\) where each chromosome is encoded using the alphabet \(\mathcal{A}_i\), the cardinality 
    of the search space \(S\) is defined as:
    
    \begin{equation}
      \label{eq:cardinality_of_the_search_space}
      |S| = \prod_{i=1}^n |\mathcal{A}_i|^{m_i}
    \end{equation}

    Note that this definition assumes that the chromosomes are independent, which may not be the 
    same for all evolutionary algorithms.
  \end{definition}

  To illustrate this concept, consider the following problem: given a binary string of length
  \(n\), find the string that has the most ones.\footnote{
    This is known as the \emph{One Max} problem \autocite{OneMaxProblema} or \emph{Ones Counting}
    problem \autocite{wilhelmstotterJeneticsJavaGenetica}.
  }
  In this case, we can use a single column matrix \(\mathbf{G}\) to represent the individual, 
  where each gene \(g_i \in \mathcal{A}\) represents the \(i\)-th bit of the string, where 
  \(\mathcal{A} = \{0, 1\}\) is the alphabet containing the two possible values of a bit.
  
  Then,
  \[
    |S| = \prod_{i=1}^1 |\mathcal{A}|^n = 2^n
  \]

  Knowing this, we can conclude that an exhaustive search of the search space would require 
  evaluating \(2^n\) individuals, and thus the algorithm would have a time complexity of 
  \(\mathcal{O}(2^n)\).
  
  This is a very simple example, but we can see how a naive search algorithm would have a very
  high time complexity.
  This would be of the utmost importance in a real world problem, where the search space would be
  much larger.

  With a representation defined, we can now define an evaluation method for the individuals, which
  is done using a \emph{fitness function}.

  \begin{definition}[Fitness function]
  \label{def:fitness_function}
    A \emph{fitness function} is a function \(f: S \rightarrow \mathbb{R}^n\), where \(S\) is the
    search space and \(n\) is the number of objectives of the optimization problem, that takes a
    genotype as input and returns a vector of real numbers representing how close the individual 
    is to the global optimum of each objective.

    The fitness function is usually defined by the user of the algorithm, and it is problem
    dependent.
  \end{definition}

  The one max problem is a maximization problem with a single objective, so the fitness function
  would be defined as follows:

  \begin{equation}
    \label{eq:fitness_function:one_max}
    f(\mathbf{G}) = \sum_{i=1}^n g_i
  \end{equation}

  In the \emph{Representation and Evaluation} section, we examined the key aspect of a genetic 
  algorithm (GA) — the representation of individuals, their encoding and search space.
  The performance of a GA significantly depends on how solutions are encoded to form individuals.
  We use the concepts of genotype, chromosome, and gene to describe the individual's representation.
  The cardinality of the search space, defined as the total number of different individuals that can
  be represented, is crucial as it impacts the algorithm's complexity.
  We introduced the \enquote{One Max} problem as an example, using a binary string representation.
  The fitness function, which evaluates individuals' fitness, plays a critical role in navigating 
  the search for optimal solutions.
  In the \enquote{One Max} problem, the fitness function sums the binary string elements, 
  representing the number of ones in the string.