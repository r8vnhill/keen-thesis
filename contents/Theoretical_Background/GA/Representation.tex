\subsection{Representation and Evaluation}
\label{sec:bg:ga:repr}
  A pivotal component of a Genetic Algorithm (GA) is the representation of 
  individuals, which encodes potential solutions into a form manipulable by the 
  GA.
  This representation delineates the algorithm's search space and is a prime 
  determinant of its performance.

  The predominant representation is a matrix of genes\footnote{See 
  \vref{def:gene}} termed the \emph{genotype}\footnote{
    See~\autocite{wilhelmstotterJeneticsJavaGenetica}
  }, wherein each column is denoted as a \emph{chromosome}\footnote{
    See \vref{def:chromosome}
  }.

  % TODO: Continue here
  \begin{definition}[Cardinality of the search space]
  \label{def:cardinality_of_the_search_space}
    The \emph{cardinality of the search space} is the number of different individuals that can be represented by the encoding.

    Formally, given a vector of alphabets \((\mathcal{A}_1,\, \mathcal{A}_2,\, \dots,\, \mathcal{A}_n)\), 
    and a representation \(\mathbf{G}\) with \(n\) chromosomes of length 
    \((m_1,\, m_2,\, \dots,\, m_n)\) where each chromosome is encoded using the
    alphabet \(\mathcal{A}_i\), the cardinality of the search space \(S\) is 
    defined as:
    
    \begin{equation}
      \label{eq:cardinality_of_the_search_space}
      |S| = \prod_{i=1}^n |\mathcal{A}_i|^{m_i}
    \end{equation}

    Note that this definition assumes that the chromosomes are independent, which may not be the 
    same for all evolutionary algorithms.
  \end{definition}

  \begin{remark}
    In the original publication of the GA~\autocite{hollandAdaptationNaturalArtificial1992a},
    the genotype was known as the \emph{environment} (\(E\)) and the search 
    space was defined as a class \(\mathcal{E}\) of all possible environments.
  \end{remark}

  To illustrate this concept, consider the following problem: given a binary 
  string of length \(n\), find the string that has the most ones, this is known 
  as the \emph{One Max} problem (OMP)~\autocite{andonovUnboundedKnapsackProblem2000}
  \footnote{
    Also \emph{Ones Counting} problem~\autocite{wilhelmstotterJeneticsJavaGenetica}, or \emph{Max Ones} problem~\autocite{ECJ}.
  } (refer to \vref{sota:omp} for a more detailed description of the problem).
  In this case, we can use a single column matrix \(\mathbf{G}\) to represent the individual, 
  where each gene \(g_i \in \mathcal{A}\) represents the \(i\)-th bit of the string, where 
  \(\mathcal{A} = \{0, 1\}\) is the alphabet containing the two possible values of a bit.
  
  Then,
  \[
    |S_\mathrm{OMP}| = \prod_{i=1}^1 |\mathcal{A}|^n = 2^n
  \]

  Knowing this, we can conclude that an exhaustive search of the search space 
  would require 
  evaluating \(2^n\) individuals, and thus the algorithm would have a time complexity of 
  \(\mathcal{O}(2^n)\).
  
  This is a very simple example, but we can see how a naive search algorithm would have a very
  high time complexity.
  This would be of the utmost importance in a real world problem, where the search space would be
  much larger.

  With a representation defined, we can now define an evaluation method for the individuals, which
  is done using a \emph{fitness function}.

  \begin{definition}[Fitness function]
  \label{def:fitness_function}
    A \emph{fitness function} is a function \(\phi: S \rightarrow \mathbb{R}^n\), where \(S\) is the
    search space and \(n\) is the number of objectives of the optimization problem, that takes a
    genotype as input and returns a vector of real numbers representing how close the individual 
    is to the global optimum of each objective.

    The fitness function is usually defined by the user of the algorithm, and it is problem
    dependent.
  \end{definition}

  \begin{definition}[Batch fitness function]
  \label{def:batch_fitness_function}
    A \emph{batch fitness function} \(\Phi: \mathbb{P} \rightarrow \mathbb{R}^{m \times n}\) is a 
    function that maps a population to a matrix of real numbers, where \(m\) is the number of
    individuals in the population and \(n\) is the number of objectives of the optimization
    problem.
  \end{definition}

  The one max problem is a maximization problem with a single objective, so the
  fitness function would be defined as follows:

  \begin{equation}
    \label{eq:fitness_function:one_max}
    \phi_\mathbf{G} = \sum_{i=1}^n g_i
  \end{equation}

  In summary, the chosen representation and evaluation methodology in genetic 
  algorithms play a foundational role in influencing the efficiency and 
  performance of the algorithm.
  The representation, particularly the genotype, steers the search space of the 
  algorithm, while the fitness function offers a means to measure the quality 
  of potential solutions within that space.
  By appreciating these key concepts and their intricacies, one gains deeper 
  insight into the mechanics of genetic algorithms and the significance of 
  their design choices.
  As showcased by the studied problem, even ostensibly straightforward 
  challenges can lead to complexities, emphasizing the importance of judicious 
  representation and evaluation strategies in any GA application.