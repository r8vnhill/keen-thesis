\section{Distributed Evolutionary Algorithms in \textit{Python} (DEAP)}
\label{sec:sota:deap}

  \emph{Distributed Evolutionary Algorithms in Python} 
  (DEAP)~\autocite{DEAPDocumentationDEAP} is a powerful evolutionary computation
  framework designed for rapid prototyping and validation of concepts.
  It stands apart from many other evolutionary computation libraries due to its
  significant modularity and versatility, which enables the construction of a
  broad range of evolutionary algorithms, genetic algorithms, and even hybrid
  algorithms.
  DEAP is open-source and available under the \textit{GNU Lesser General Public
  License v3.0} (LGPL-3.0)~\autocite{GNULesserGeneral}.

  DEAP is structured around two primary components: the \texttt{Creators} and
  the \texttt{Toolbox}.
  The \texttt{Creators} module facilitates the generation of new classes
  integral to the genetic algorithm, such as individuals and populations.
  In contrast, the \texttt{Toolbox} serves as a comprehensive repository for
  various operators necessary in evolutionary algorithms, including the
  evaluation, selection, mutation, and crossover functions.

  The following illustrates the use of DEAP to solve the OMP problem using a
  genetic algorithm:

  \lstinputlisting[
    language=Python, 
    caption={A simple genetic algorithm using DEAP}, 
    label={src:sota:deap:example}
  ]{deap_1.py}

  When executed, the program outputs the following:

  \begin{verbatim}
    Target fitness reached at generation 12.
    Best individual is: 11111111111111111111
    with fitness: 20.0
  \end{verbatim}

  This example demonstrates the use of DEAP to implement a simple genetic
  algorithm, with the added functionality to stop evolution once a target
  fitness is achieved.

  Here is a breakdown of the code:

  \begin{enumerate}
    \item[1-2] DEAP uses a meta-factory to \textit{dynamically} create
      user-defined classes.
      Here, we create two classes: \texttt{FitnessMax} for the fitness
      (single-objective, to be maximized) and \texttt{Individual} to represent
      an individual in the population.
      The second argument to \texttt{create} is the base class, meaning that
      \texttt{FitnessMax} is a subclass of \texttt{base.Fitness} and
      \texttt{Individual} is a subclass of \texttt{list}.
    \item[3] The \texttt{Toolbox} is created, it is used to store most of the
      functions and arguments required to perform the genetic algorithm.
    \item[4-5] We register two functions in the toolbox: \texttt{attr\_bool} 
      which generates a random binary number (0 or 1), and \texttt{individual}
      which creates a new individual using the \texttt{attr\_bool} function.
      The individual consists of 20 binary numbers (genes).
    \item[6-7] We then register two more functions: \texttt{population} which
      creates a population of individuals, and \texttt{evaluate} which evaluates
      an individual's fitness as the sum of its genes.
    \item[8-10] We register three more functions in the toolbox: \texttt{mate}
      for performing \textit{two-point crossover}, \texttt{mutate} with a
      \textit{bit-flip mutator} with a 5\% probability, and \texttt{select}
      for performing tournament selection with a tournament size of 3.
    \item[11] We set the target fitness to 20 (since we aim to maximize the sum
      of the genes, the maximum possible fitness is 20) 
    \item[12-32] In the main section of the code, we seed the random number
      generator to ensure reproducibility, create a population of 20
      individuals, and set up the \textit{Hall of Fame} and a statistics object
      to keep track of the maximum fitness in the population.
    \item[19-29] A while loop is used to run the genetic algorithm until the
      target fitness is reached.
      Inside the loop:

      \begin{enumerate}
        \item[21-22] We apply crossover and mutation to the population using the
          \texttt{varAnd} function,\footnote{This function simply applies the
          variation operators to the population.} and then evaluate the fitness
          of the offspring
        \item[23-24] We assign the newly computed fitness values to the
          individuals.
        \item[25] We replace the old population with the selected individuals
          from the offspring.
        \item[26-27] We update the Hall of Fame and compile the statistics.
        \item[28-29] We check if the maximum fitness has reached the target
          fitness.
          If it has, we break the loop and the algorithm stops.
      \end{enumerate}

    \item[30-32] Finally, we print the results.
  \end{enumerate}

  This example illustrates the key aspects of using DEAP: creating custom
  classes, setting up a toolbox, defining and registering functions, and
  manually controlling the loop of the genetic algorithm.

  DEAP provides robust support for multi-objective algorithms and
  parallelization, which are common requirements in complex optimization
  problems.
  The library also includes a set of benchmark functions and examples to help
  users understand the various algorithms' behavior and performance.

  DEAP is widely regarded as one of the most comprehensive and cutting-edge
  genetic algorithm frameworks currently available.
  It boasts a wealth of documentation and a robust community of users and
  contributors.
  However, certain facets of DEAP may pose challenges for users, as reflected in
  the preceding code example.

  \begin{itemize}
    \item DEAP's code can be quite verbose, and the utilization of a toolbox
      might initially confuse newcomers, given that it's not a standard
      convention in evolutionary algorithms.
      This characteristic becomes more evident when juxtaposed with the code
      syntax of other frameworks.
    \item A trade-off exists between the framework's flexibility and ease of
      use.
      While DEAP's malleability is commendable, it heavily relies on code
      injection.
      This reliance complicates static analysis, resulting in warnings such as
      \enquote{Unresolved attribute reference} in \textit{PyCharm} and 
      \enquote{\texttt{reportGeneralTypeIssues}} in \textit{Visual Studio
      Code} with \textit{Pylance} (on lines 2, 5, 6, 14, 22, and 25), which
      can inhibit certain IDE capabilities, including auto-completion and
      refactoring.
    \item DEAP's dependence on code injection can potentially introduce risks
      in production environments.
      Even though code injection is not inherently unsafe, if handled
      carelessly, it can result in system vulnerabilities.
    \item \textit{Python}'s dynamic typing can be a double-edged sword.
      While it offers flexibility, type errors will not be detected until
      runtime, which can potentially lead to bugs and render the code more
      challenging to comprehend.
    \item For efficiency, DEAP necessitates the \textit{NumPy} library, which
      isn't part of the standard \textit{Python} library.
      Although this is not inherently negative, it does introduce an extra
      dependency that needs to be managed.
    \item Enhancing DEAP's functionality mandates an in-depth comprehension of
      the library's architecture, and in many instances, necessitates manual
      modifications to the algorithm's execution loop.
      This is evident in the previous example, where manual control of the
      algorithm's loop is required to halt upon reaching the target fitness.
    \item The library's tendency to abbreviate variable names can render the
      code difficult to interpret.
      Without referring to the documentation, understanding the meaning of
      variables such as \texttt{cxpb} and \texttt{mutpb} can be challenging.
      The problem can be exacerbated by \textit{Python}'s keyword arguments
      (\texttt{kwargs}), as they are not explicitly declared, preventing IDEs
      from providing valuable information about them.
  \end{itemize}

  Despite this, DEAP's flexibility and modularity have made it a popular choice
  among researchers and practitioners in the field of evolutionary computation.
