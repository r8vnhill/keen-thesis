\section{Agile Artificial Intelligence in Pharo}
\label{sec:sota:pharo}
  Alexandre Bergel's \emph{\enquote{Agile Artificial Intelligence in 
  Pharo}}~\autocite{bergelAgileArtificialIntelligence2020} delivers a 
  comprehensive exploration of genetic algorithms, encapsulating their theory 
  and application within the \textit{Pharo} programming environment.
  This language, dynamic and reflective, finds its roots in \textit{Smalltalk}.
  The source code of the framework is licensed under a \textit{freeware license}
  and is available on 
  \textit{GitHub}.\footnote{\url{https://github.com/Apress/agile-ai-in-pharo/tree/master}}

  Bergel's tome not only introduces readers to genetic algorithms but also to 
  neural networks and the concept of neuroevolution.
  The framework's design and the reasoning behind it are meticulously detailed, 
  offering readers insights into its three main components: \emph{genetic 
  operators}, \emph{selection operators}, and the \emph{evolution engine}.
  Accompanying this detailed framework are test cases.
  These serve dual roles: exemplifying the framework's application and guiding 
  enthusiasts in developing their own robust and extensible frameworks.

  The book also provides utilities for the visualization of genetic algorithms
  and neural networks.
  These are implemented using \textit{Roassal}, a visualization engine for
  \textit{Pharo}.
  The engine is capable of rendering a variety of graphs, including directed and
  undirected graphs, trees, and more relevant to this document, a summary of the
  evolution fitness over time.
  
  To showcase the framework's utility, the OMP problem resolution via a genetic 
  algorithm is provided:

  \begin{code}*{A simple genetic algorithm using Bergel's framework.}{
    label=lst:sota:pharo:omp
  }{smalltalk}
    engine := GAEngine new.
    engine random: (Random seed: 11).
    engine populationSize: 20.
    engine numberOfGenes: 20.
    engine createGeneBlock: [ :rand :index :ind |
      (0 to: 1) atRandom: rand ].
    engine fitnessBlock: [ :ind | ind count: [ :each | each = 1 ] ].
    engine endIfFitnessIsAbove: 20.
    engine run.
    engine trace: 'Target fitness reached at generation '.
    engine traceCr: engine logs last generationNumber.
    engine trace: 'Best individual is: '.
    engine traceCr: engine logs last fittestIndividual genes .
    engine trace: 'with fitness: '.
    engine traceCr: engine logs last bestFitness .
  \end{code}

  Breaking it down:

  \begin{enumerate}
    \item[1] A new genetic algorithm engine is created.
    \item[2] The random seed is set to 11. 
    \item[3-4] The population and number of genes are both set to 20.
    \item[5-6] The gene creation block is outlined for initial population 
      generation, returning a random digit either 0 or 1.
    \item[7] The fitness block is then detailed, here counting the number of 
      1s in an individual.
    \item[8] The algorithm's termination condition is set, halting if an 
      individual's fitness surpasses 20.
    \item[9] The algorithm is then executed.
    \item[10-18] The algorithm's logs are then printed to the console.
      Note that the engine keeps a log of the evolution process.
  \end{enumerate}

  While Bergel's framework, as presented in the book, stands as a clear and 
  potent genetic algorithm tool for \textit{Pharo}, its primary intent is 
  instructional.
  As such, it may not rival the robustness of seasoned frameworks like DEAP or 
  \textit{Jenetics}.
