\section{Agile Artificial Intelligence in Pharo}
\label{sec:sota:pharo}
  Alexandre Bergel's \emph{\enquote{Agile Artificial Intelligence in 
  Pharo}}~\autocite{bergelAgileArtificialIntelligence2020} delivers a 
  comprehensive exploration of genetic algorithms, encapsulating their theory 
  and application within the \textit{Pharo} programming environment.
  This language, dynamic and reflective, finds its roots in \textit{Smalltalk}.

  Bergel's tome not only introduces readers to genetic algorithms but also to 
  neural networks and the concept of neuroevolution.
  The framework's design and the reasoning behind it are meticulously detailed, 
  offering readers insights into its three main components: \emph{genetic 
  operators}, \emph{selection operators}, and the \emph{evolution engine}.
  Accompanying this detailed framework are test cases.
  These serve dual roles: exemplifying the framework's application and guiding 
  enthusiasts in developing their own robust and extensible frameworks.

  To showcase the framework's utility, the OMP problem resolution via a genetic 
  algorithm is provided:

  \begin{listing}[H]
    \caption{A simple genetic algorithm using Bergel's framework.}
    \label{src:sota:pharo:example}
  \end{listing}
  \vspace{-2em}
  \begin{minted}{smalltalk}
    | engine |
    engine := GAEngine new.
    engine populationSize: 20.
    engine numberOfGenes: 20.
    engine createGeneBlock: [ :rand :index :ind | (0 to: 1) atRandom: rand ].
    engine fitnessBlock: [ :ind | ind count: [ :each | each = 1 ] ].
    engine endIfFitnessIsAbove: 19.
    engine run.
    Transcript show: engine "Target fitness reached at generation "; 
      engine logs last generation.
    Transcript show: engine "Best individual is: "; 
      engine logs last fittestIndividual.
    Transcript show: engine "with fitness: ";
      engine logs last bestFitness.
  \end{minted}

  Breaking it down:

  \begin{enumerate}
      \item[1-2] A new genetic algorithm engine is declared and initialized.
      \item[3-4] The population and number of genes are both set to 20.
      \item[5] The gene creation block is outlined for initial population 
        generation, returning a random digit either 0 or 1.
      \item[6] The fitness block is then detailed, here counting the number of 
        1s in an individual.
      \item[7] The algorithm's termination condition is set, halting if an 
        individual's fitness surpasses 19.
      \item[8] Finally, the algorithm is executed.
  \end{enumerate}

  While Bergel's framework, as presented in the book, stands as a clear and 
  potent genetic algorithm tool for \textit{Pharo}, its primary intent is 
  instructional.
  As such, it may not rival the robustness of seasoned frameworks like DEAP or 
  \textit{Jenetics}.
