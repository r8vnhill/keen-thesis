\section{GeneticSharp: Comprehensive Overview of a .NET Genetic Algorithm Library}
\label{sec:sota:geneticsharp}
  \textit{GeneticSharp} is a high-performance, extensible, cross-platform 
  \textit{C\#} Genetic Algorithm library tailored for multi-threading.
  This open-source library, designed to streamline the integration of genetic algorithms into applications, is licensed under the 
  \textit{MIT License}~\autocite{MITLicense2006}.

  \textit{GeneticSharp} revolves around three foundational concepts: 
  \mintinline[breaklines]{csharp}{Chromosome}, 
  \mintinline[breaklines]{csharp}{Population}, and 
  \mintinline[breaklines]{csharp}{GeneticAlgorithm}:

  \begin{itemize}
    \item \texttt{Chromosome} serves as the foundation for various chromosome 
      types, responsible for gene storage and genetic operator management.
    \item \texttt{Population} maintains a collection of chromosomes undergoing 
      evolution.
    \item \texttt{GeneticAlgorithm} acts as the algorithm's primary driver, 
      encompassing its main loop.
  \end{itemize}

  Distinctively, \textit{GeneticSharp} capitalizes on parallel computing, 
  accelerating fitness evaluation, crossover, and mutation.
  Furthermore, its extension-oriented design simplifies the addition of new 
  operators or further customizations to the genetic algorithm.

  Consider the following example, which demonstrates solving the OMP using 
  \textit{GeneticSharp}:

  \begin{code}*{Solving OMP with \textit{GeneticSharp}}{
    label={lst:sota:geneticsharp:example}
  }{csharp}
    public sealed class BinaryChromosome : BinaryChromosomeBase {
        public BinaryChromosome(int length) : base(length) {
            CreateGenes();
        }
        public override IChromosome CreateNew() {
            return new BinaryChromosome(Length);
        }
    }
    public class OneMaxFitness : IFitness {
        public double Evaluate(IChromosome chromosome) {
            return chromosome.GetGenes().Count(gene => (int)gene.Value == 1);
        }
    }
    public class ReproducibleRandom : RandomizationBase {
        private static readonly object GlobalLock = new();
        private static readonly ThreadLocal<Random?> ThreadRandom = new(NewRandom);
        private static Random? Instance => ThreadRandom.Value;
        private static Random NewRandom() {
            lock (GlobalLock) {
                return new Random(11);
            }
        }
        public override int GetInt(int min, int max) {
            Debug.Assert(Instance != null, nameof(Instance) + " != null");
            return Instance.Next(min, max);
        }
        public override float GetFloat() {
            Debug.Assert(Instance != null, "ReproducibleRandom.Instance != null");
            return (float)Instance.NextDouble();
        }
        public override double GetDouble() {
            Debug.Assert(Instance != null, "ReproducibleRandom.Instance != null");
            return Instance.NextDouble();
        }
    }
    const int chromosomeLength = 20;
    const int populationSize = 20;
    RandomizationProvider.Current = new ReproducibleRandom();
    var selection = new EliteSelection();
    var crossover = new UniformCrossover();
    var mutation = new FlipBitMutation();
    var fitness = new OneMaxFitness();
    var chromosome = new BinaryChromosome(chromosomeLength);
    var population = new Population(populationSize, populationSize, chromosome);
    var ga = new GeneticAlgorithm(population, fitness, selection, crossover, mutation) {
        Termination = new FitnessThresholdTermination(chromosomeLength)
    };
    ga.Start();
    Console.WriteLine($"Target fitness reached at generation: {ga.GenerationsNumber}");
    var bestChromosome = ga.BestChromosome as BinaryChromosome;
    Console.WriteLine($"Best individual is: {bestChromosome}");
    Console.WriteLine($"with fitness: {bestChromosome?.Fitness}");
  \end{code}

  Dissecting the code reveals the following:

  \begin{enumerate}
    \item[1-8] The \texttt{BinaryChromosome} class is sealed\footnote{
        Details on \vref{def:sealed_class}.
      } to inhibit subclassing and encapsulates a chromosome with binary genes. 
      Sealing is an advisable practice here due to the constructor's invocation 
      of the virtual \texttt{CreateGenes} method, thus eliminating potential 
      ambiguities from subclass method overrides.
      \begin{enumerate}
        \item[2-4] The constructor, accepting a length parameter, invokes the 
          base constructor.
          Given that the base may call a virtual method, sealing thwarts 
          unforeseen behaviors from subclasses unknowingly overriding it.
        \item[4-7] \texttt{CreateNew} method proffers a fresh 
          \texttt{BinaryChromosome} instance retaining the original length.
      \end{enumerate}
    \item[9-13] \texttt{OneMaxFitness} class quantifies a chromosome's fitness.
      \begin{enumerate}
        \item[10-12] \texttt{Evaluate} method tallies genes equating to 1 
          (equivalent to binary \mintinline{csharp}{true}), serving as the OMP's fitness metric.
      \end{enumerate}
    \item[14-35] The \mintinline{csharp}{ReproducibleRandom} class furnishes 
      reproducibly random numbers, critical for consistent genetic algorithm 
      outcomes across iterations.
      \begin{enumerate}
        \item[15-17] A global lock alongside a thread-local 
          \mintinline{csharp}{Random} instance 
          (\mintinline{csharp}{ThreadRandom}) ensures synchronized seeding 
          across threads for consistent randomness.
        \item[18-22] \mintinline{csharp}{NewRandom} method spawns a 
          \mintinline{csharp}{Random} object seeded with 11, bolstered by the 
          \mintinline{csharp}{lock} keyword for thread-safety.
        \item[23-34] Methods \mintinline{csharp}{GetInt}, 
          \mintinline{csharp}{GetFloat}, and \mintinline{csharp}{GetDouble} 
          yield values from the \mintinline{csharp}{Random} instance.
      \end{enumerate}
    \item[36-37] Constants pertinent to the genetic algorithm, such as 
      chromosome length and population size, are delineated.
    \item[38] Randomization defaults to the \texttt{ReproducibleRandom} class.
    \item[39-44] Configuration lines dictate the genetic algorithm's components, 
      encompassing selection (elite), crossover (uniform), mutation (flip bit), 
      and the fitness function (\texttt{OneMaxFitness}).
    \item[45-47] The genesis population and genetic algorithm instance are 
      instantiated.
    \item[46] The algorithm's cessation is predicated on a chromosome attaining 
      peak fitness, tantamount to its length.
    \item[48] The genetic algorithm is set in motion.
    \item[49-52] Outputs include the generation count to attain target fitness, 
      the paramount chromosome, and its corresponding fitness.
  \end{enumerate}

  Despite its many advantages, \textit{GeneticSharp} also has certain 
  limitations:

  \begin{itemize}
    \item \textbf{Overwhelming verbosity:} The library's detailed nature can 
      overshadow its functionality, especially in contrast to succinct 
      frameworks like Jenetics.
    \item \textbf{Externalized fitness function:} The architecture demands that 
      fitness functions be isolated in distinct classes.
      This design choice might make trivial problems seem more convoluted than 
      when using, for example, higher-order functions.
    \item \textbf{Sparse documentation:} Compared to competitors like DEAP, the 
      library lags in both instructional content and practical examples.
      Novices might find this shortage hampers their initial engagement with the 
      library.
    \item \textbf{Parallelization trade-offs:} Incorporating parallel 
      functionality amplifies the complexity of defining custom genetic 
      operators.
      Although parallel processing boosts execution speed, developers need to 
      grapple with the intricacies it introduces.
    \item \textbf{Advanced C\# intricacies:} Features unique to advanced 
      \textit{C\#} such as the \mintinline{csharp}{sealed} attribute can 
      disorient beginners.
      New users might be deterred by such advanced constructs, especially 
      without prior experience in \textit{C\#}.
    \item \textbf{Developer-centric focus:} \textit{GeneticSharp} caters 
      predominantly to application creation, potentially sidelining academic 
      researchers.
      Scholars deeply entrenched in genetic algorithm studies might find the 
      library's emphasis misaligned with their interests.
    \item \textbf{Budding community support:} The fledgling nature of the 
      \textit{GeneticSharp} community in 2023 could impede access to resources 
      and prompt support.
      This nascent state might affect user engagement, library updates, and peer 
      assistance.
    \item \textbf{Conventional algorithm scope:} The library's focus remains 
      anchored to traditional genetic algorithms, omitting newer evolutionary 
      algorithm derivations.
      Practitioners might find the lack of support for paradigms like genetic 
      programming limiting.
  \end{itemize}

  In summary, \textit{GeneticSharp} offers a dynamic, parallelized platform for 
  crafting genetic algorithms in \textit{C\#}. 
  Its customizability and developer-centric approach make it a compelling choice 
  for those aiming to harness genetic algorithms within their software projects. 
  Yet, users must be cognizant of its more intricate API, the nuances introduced 
  by parallelization, and potential challenges arising from a smaller community 
  and limited documentation.
  Ultimately, while \textit{GeneticSharp} solidifies its position as a valuable 
  tool within the \textit{.NET} arena, its optimal utilization depends on 
  aligning its capabilities with the user's specific requirements and 
  familiarity with genetic algorithms.
