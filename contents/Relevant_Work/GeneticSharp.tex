\section{GeneticSharp: Comprehensive Overview of a .NET Genetic Algorithm Library}
\label{sec:sota:geneticsharp}
  \textit{GeneticSharp} is a high-performance, extensible, cross-platform 
  \textit{C\#} Genetic Algorithm library tailored for multi-threading.
  This open-source library, designed to streamline the integration of genetic algorithms into applications, is licensed under the 
  \textit{MIT License}~\autocite{MITLicense2006}.

  \textit{GeneticSharp} revolves around three foundational concepts: 
  \mintinline[breaklines]{csharp}{Chromosome}, 
  \mintinline[breaklines]{csharp}{Population}, and 
  \mintinline[breaklines]{csharp}{GeneticAlgorithm}:

  \begin{itemize}
    \item \texttt{Chromosome} serves as the foundation for various chromosome 
      types, responsible for gene storage and genetic operator management.
    \item \texttt{Population} maintains a collection of chromosomes undergoing 
      evolution.
    \item \texttt{GeneticAlgorithm} acts as the algorithm's primary driver, 
      encompassing its main loop.
  \end{itemize}

  Distinctively, \textit{GeneticSharp} capitalizes on parallel computing, 
  accelerating fitness evaluation, crossover, and mutation.
  Furthermore, its extension-oriented design simplifies the addition of new 
  operators or further customizations to the genetic algorithm.

  Consider the following example, which demonstrates solving the OMP using 
  \textit{GeneticSharp}:

  \begin{code}*{Solving OMP with \textit{GeneticSharp}}{
    label={lst:sota:geneticsharp:example}
  }{csharp}
    public sealed class BinaryChromosome : BinaryChromosomeBase {
        public BinaryChromosome(int length) : base(length) {
            CreateGenes();
        }
        public override IChromosome CreateNew() {
            return new BinaryChromosome(Length);
        }
    }
    public class OneMaxFitness : IFitness {
        public double Evaluate(IChromosome chromosome) {
            return chromosome.GetGenes().Count(gene => (int)gene.Value == 1);
        }
    }
    public class ReproducibleRandom : RandomizationBase {
        private static readonly object GlobalLock = new();
        private static readonly ThreadLocal<Random?> ThreadRandom = new(NewRandom);
        private static Random? Instance => ThreadRandom.Value;
        private static Random NewRandom() {
            lock (GlobalLock) {
                return new Random(11);
            }
        }
        public override int GetInt(int min, int max) {
            Debug.Assert(Instance != null, nameof(Instance) + " != null");
            return Instance.Next(min, max);
        }
        public override float GetFloat() {
            Debug.Assert(Instance != null, "ReproducibleRandom.Instance != null");
            return (float)Instance.NextDouble();
        }
        public override double GetDouble() {
            Debug.Assert(Instance != null, "ReproducibleRandom.Instance != null");
            return Instance.NextDouble();
        }
    }
    const int chromosomeLength = 20;
    const int populationSize = 20;
    RandomizationProvider.Current = new ReproducibleRandom();
    var selection = new EliteSelection();
    var crossover = new UniformCrossover();
    var mutation = new FlipBitMutation();
    var fitness = new OneMaxFitness();
    var chromosome = new BinaryChromosome(chromosomeLength);
    var population = new Population(populationSize, populationSize, chromosome);
    var ga = new GeneticAlgorithm(population, fitness, selection, crossover, mutation) {
        Termination = new FitnessThresholdTermination(chromosomeLength)
    };
    ga.Start();
    Console.WriteLine($"Target fitness reached at generation: {ga.GenerationsNumber}");
    var bestChromosome = ga.BestChromosome as BinaryChromosome;
    Console.WriteLine($"Best individual is: {bestChromosome}");
    Console.WriteLine($"with fitness: {bestChromosome?.Fitness}");
  \end{code}

  Dissecting the code reveals the following:

  \begin{enumerate}
    \item[1-8] The \texttt{BinaryChromosome} class is sealed\footnote{
        Details on \vref{def:sealed_class}.
      } to inhibit subclassing and encapsulates a chromosome with binary genes. 
      Sealing is an advisable practice here due to the constructor's invocation 
      of the virtual \texttt{CreateGenes} method, thus eliminating potential 
      ambiguities from subclass method overrides.
      \begin{enumerate}
        \item[2-4] The constructor, accepting a length parameter, invokes the 
          base constructor.
          Given that the base may call a virtual method, sealing thwarts 
          unforeseen behaviors from subclasses unknowingly overriding it.
        \item[4-7] \texttt{CreateNew} method proffers a fresh 
          \texttt{BinaryChromosome} instance retaining the original length.
      \end{enumerate}
    \item[9-13] \texttt{OneMaxFitness} class quantifies a chromosome's fitness.
      \begin{enumerate}
        \item[10-12] \texttt{Evaluate} method tallies genes equating to 1 
          (equivalent to binary \mintinline{csharp}{true}), serving as the OMP's fitness metric.
      \end{enumerate}
    \item[14-35] The \mintinline{csharp}{ReproducibleRandom} class furnishes 
      reproducibly random numbers, critical for consistent genetic algorithm 
      outcomes across iterations.
      \begin{enumerate}
        \item[15-17] A global lock alongside a thread-local 
          \mintinline{csharp}{Random} instance 
          (\mintinline{csharp}{ThreadRandom}) ensures synchronized seeding 
          across threads for consistent randomness.
        \item[18-22] \mintinline{csharp}{NewRandom} method spawns a 
          \mintinline{csharp}{Random} object seeded with 11, bolstered by the 
          \mintinline{csharp}{lock} keyword for thread-safety.
        \item[23-34] Methods \mintinline{csharp}{GetInt}, 
          \mintinline{csharp}{GetFloat}, and \mintinline{csharp}{GetDouble} 
          yield values from the \mintinline{csharp}{Random} instance.
      \end{enumerate}
    \item[36-37] Constants pertinent to the genetic algorithm, such as 
      chromosome length and population size, are delineated.
    \item[38] Randomization defaults to the \texttt{ReproducibleRandom} class.
    \item[39-44] Configuration lines dictate the genetic algorithm's components, 
      encompassing selection (elite), crossover (uniform), mutation (flip bit), 
      and the fitness function (\texttt{OneMaxFitness}).
    \item[45-47] The genesis population and genetic algorithm instance are 
      instantiated.
    \item[46] The algorithm's cessation is predicated on a chromosome attaining 
      peak fitness, tantamount to its length.
    \item[48] The genetic algorithm is set in motion.
    \item[49-52] Outputs include the generation count to attain target fitness, 
      the paramount chromosome, and its corresponding fitness.
  \end{enumerate}

  Despite its many advantages, \textit{GeneticSharp} also has certain 
  limitations:

  \begin{itemize}
    \item The library's verbosity might be daunting, especially when compared to 
      succinct frameworks such as Jenetics.
    \item The fitness function is defined in a separate class, which may be 
      cumbersome for simple problems that could be better served by using higher
      order functions.
    \item Its documentation and example base are relatively limited, 
      particularly when juxtaposed against counterparts like DEAP, which could 
      hinder its user-friendliness for beginners.
    \item Parallelization, while beneficial, introduces intricacies when 
      crafting custom genetic operators.
    \item The use of advanced \textit{C\#} features, like the 
      \mintinline{csharp}{sealed} attribute in 
      \mintinline{csharp}{BinaryChromosome}, may confuse newcomers.
    \item \textit{GeneticSharp} primarily targets application development, 
      potentially making it less attractive to researchers delving deep into 
      genetic algorithms.
    \item The \textit{GeneticSharp} community is relatively nascent as of 2023, 
      leading to potential challenges in resource availability, update frequency, 
      and community support.
    \item The library focuses on traditional genetic algorithms, neglecting other 
      evolutionary algorithm variants like genetic programming.
  \end{itemize}

  In summary, \textit{GeneticSharp} offers a dynamic, parallelized platform for 
  crafting genetic algorithms in \textit{C\#}. 
  Its customizability and developer-centric approach make it a compelling choice 
  for those aiming to harness genetic algorithms within their software projects. 
  Yet, users must be cognizant of its more intricate API, the nuances introduced 
  by parallelization, and potential challenges arising from a smaller community 
  and limited documentation.
  Ultimately, while \textit{GeneticSharp} solidifies its position as a valuable 
  tool within the \textit{.NET} arena, its optimal utilization depends on aligning 
  its capabilities with the user's specific requirements and familiarity with 
  genetic algorithms.
