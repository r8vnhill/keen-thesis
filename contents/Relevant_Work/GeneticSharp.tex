% TODO: Continue from here
\section{GeneticSharp: A .NET-based Genetic Algorithm Framework}
\label{sec:sota:geneticsharp}
  \emph{GeneticSharp}~\autocite{giacomelliGeneticSharp2023} is a fast, 
  extensible, multi-platform and multithreading \textit{C\#} Genetic Algorithm 
  library that \textbf{simplifies the development of applications} using 
  genetic algorithms.
  It is open-source and licensed under the \textit{MIT License}~\autocite{MITLicense2006}.

  The central theme of \textit{GeneticSharp} revolves around three main 
  concepts: \mintinline[breaklines]{csharp}{Chromosome}, 
  \mintinline[breaklines]{csharp}{Population}, and 
  \mintinline[breaklines]{csharp}{GeneticAlgorithm}.
  The \texttt{Chromosome} class is the base for all types of chromosomes, 
  storing genes and managing genetic operators.
  The \texttt{Population} class holds a set of chromosomes which are evolved 
  over time.
  The \texttt{GeneticAlgorithm} is the central point of execution and contains 
  the genetic algorithm's main loop.

  \textit{GeneticSharp} stands out with its support for parallelism, which 
  greatly speeds up the execution of fitness evaluation, crossover, and 
  mutation.
  It also introduces the concept of extensions, providing a simple way to add 
  new operators and customization to the genetic algorithm.

  Here's an example of how to solve the OMP using \textit{GeneticSharp}:

  \begin{minted}[linenos]{csharp}
    public sealed class BinaryChromosome : BinaryChromosomeBase {
        public BinaryChromosome(int length) : base(length) {
            CreateGenes();
        }
        public override IChromosome CreateNew() {
            return new BinaryChromosome(Length);
        }
    }
    public class OneMaxFitness : IFitness {
        public double Evaluate(IChromosome chromosome) {
            return chromosome.GetGenes().Count(gene => (int)gene.Value == 1);
        }
    }
    public class ReproducibleRandom : RandomizationBase {
        private static readonly object GlobalLock = new();
        private static readonly ThreadLocal<Random?> ThreadRandom = new(NewRandom);
        private static Random? Instance => ThreadRandom.Value;
        private static Random NewRandom() {
            lock (GlobalLock) {
                return new Random(11);
            }
        }
        public override int GetInt(int min, int max) {
            Debug.Assert(Instance != null, nameof(Instance) + " != null");
            return Instance.Next(min, max);
        }
        public override float GetFloat() {
            Debug.Assert(Instance != null, "ReproducibleRandom.Instance != null");
            return (float)Instance.NextDouble();
        }
        public override double GetDouble() {
            Debug.Assert(Instance != null, "ReproducibleRandom.Instance != null");
            return Instance.NextDouble();
        }
    }
    const int chromosomeLength = 20;
    const int populationSize = 20;
    RandomizationProvider.Current = new ReproducibleRandom();
    var selection = new EliteSelection();
    var crossover = new UniformCrossover();
    var mutation = new FlipBitMutation();
    var fitness = new OneMaxFitness();
    var chromosome = new BinaryChromosome(chromosomeLength);
    var population = new Population(populationSize, populationSize, chromosome);
    var ga = new GeneticAlgorithm(population, fitness, selection, crossover, mutation) {
        Termination = new FitnessThresholdTermination(chromosomeLength)
    };
    ga.Start();
    Console.WriteLine($"Target fitness reached at generation: {ga.GenerationsNumber}");
    var bestChromosome = ga.BestChromosome as BinaryChromosome;
    Console.WriteLine($"Best individual is: {bestChromosome}");
    Console.WriteLine($"with fitness: {bestChromosome?.Fitness}");
  \end{minted}

  Here is an explanation of the code:

  \begin{enumerate}
    \item[1-8] The \texttt{BinaryChromosome} class, which is sealed to 
      prevent further subclassing, represents a chromosome with binary genes.
      Sealing the class is a best practice when a class has a constructor 
      that calls a virtual method (in this case, the \texttt{CreateGenes})
      By sealing the class, we prevent any confusion that could arise if a 
      subclass were to override the virtual method.
      \begin{enumerate}
        \item[2-4] This is the constructor which takes a length as an argument 
          and passes it to the base constructor.
          Note that the base constructor might call a virtual method, and in 
          order to prevent the virtual method from being overridden by a 
          subclass (which could lead to unexpected behavior if the subclass 
          isn't aware that the method is called in the constructor), we seal 
          the class.
        \item[4-7] The \texttt{CreateNew} method returns a new instance of
          \texttt{BinaryChromosome} with the same length.
      \end{enumerate}
    \item[9-13] The \texttt{OneMaxFitness} class calculates the fitness of a 
      chromosome.
      \begin{enumerate}
        \item[10-12] The \texttt{Evaluate} method counts the number of genes 
          with value 1 (representing binary \mintinline{csharp}{true}) in the 
          chromosome.
          This is the measure of fitness in the OneMax problem.
      \end{enumerate}
    \item[14-35] The \mintinline{csharp}{ReproducibleRandom} class is used for 
      generating random numbers in a reproducible way.
      This is especially important to ensure that the results of the genetic
      algorithm are reproducible across multiple runs.
      \begin{enumerate}
        \item[15-17] A global lock and a thread-local 
          \mintinline{csharp}{Random} instance 
          (\mintinline{csharp}{ThreadRandom}) are declared.
          These ensure that each thread gets its own
          \mintinline{csharp}{Random} instance, but they're all seeded in a 
          synchronized way to provide reproducibility across multiple threads.
        \item[18-22] The \mintinline{csharp}{NewRandom} method is used to 
          create a new \mintinline{csharp}{Random} object with a seed of 11.
          The method is thread-safe due to the use of the 
          \mintinline{csharp}{lock} keyword.
        \item[23-34] The \mintinline{csharp}{GetInt}, 
          \mintinline{csharp}{GetFloat}, and \mintinline{csharp}{GetDouble} 
          methods are overridden to return values from the 
          \mintinline{csharp}{Random} instance.
      \end{enumerate}
    \item[36-37] Here, we define some constants for the genetic algorithm, 
      including the length of the chromosomes and the size of the population.
    \item[38] This line sets the current randomization provider to our 
      \texttt{ReproducibleRandom} class.
    \item[39-44] These lines set up the different components of the genetic 
      algorithm, including the selection method (elite selection), crossover 
      method (uniform crossover), mutation method (flip bit mutation), and 
      fitness function (\texttt{OneMaxFitness}).
    \item[45-47] These lines create the initial population and the genetic 
      algorithm instance.
    \item[46] This line sets the termination condition for the genetic 
      algorithm to be when a chromosome achieves the maximum possible fitness 
      (equal to the chromosome length).
    \item[48] This line starts the genetic algorithm.
    \item[49-52] These lines print the number of generations it took to reach the target fitness, the best chromosome, and its fitness.
  \end{enumerate}

  \textit{GeneticSharp}'s strengths lie in its simplicity and ease of use.
  It presents a straightforward and flexible API that allows the creation of 
  custom chromosomes, fitness functions, and genetic operators, and the 
  framework's design allows easy parallelization of computations.

  While \textit{GeneticSharp} offers numerous benefits, it also possesses 
  certain drawbacks.

  \begin{itemize}
    \item Its API's verbosity may be challenging, particularly when contrasted 
      with more succinct frameworks like Jenetics.
      This is evident in our previous example where defining classes for the 
      chromosome, fitness function, and randomization provider was necessary.
    \item \textit{GeneticSharp}'s documentation lacks depth in comparison to 
      its counterparts (in particular DEAP), which coupled with the shortage of 
      available examples, may make it less user-friendly, especially for 
      newcomers.
    \item Supporting parallelization within \textit{GeneticSharp} may introduce 
      complexities when implementing custom genetic operators.
      A case in point is the \mintinline{csharp}{ReproducibleRandom} class in our example.
      It necessitates synchronization techniques to ensure that each thread 
      possesses its own \mintinline{csharp}{Random} instance.
      This forces users to familiarize themselves with the framework's 
      threading model.
    \item The \mintinline{csharp}{BinaryChromosome} class is marked as 
      \mintinline{csharp}{sealed}, an advanced feature of \textit{C\#} which 
      could be perplexing for novices.
    \item \textit{GeneticSharp} is designed primarily for application 
      development, thereby potentially limiting its appeal to researchers 
      aiming to experiment with various genetic algorithms.
    \item As of 2023, the \textit{GeneticSharp} community remains relatively 
      small compared to communities dedicated to more established machine 
      learning or AI libraries.
      This may result in fewer resources, less frequent updates, and limited 
      community support.
    \item In its current version (as of 2023), \textit{GeneticSharp} 
      exclusively supports classic genetic algorithms, lacking support for 
      other evolutionary algorithms like genetic programming.
  \end{itemize}
  
  In conclusion, \textit{GeneticSharp} provides a robust, multi-threaded 
  platform for implementing genetic algorithms using \textit{C\#}.
  Its support for parallelism and customizable genetic operators highlight its 
  adaptability and utility.
  With its emphasis on application development, it provides an intuitive 
  platform for developers to integrate genetic algorithms into their software.

  However, its merits must be balanced against its potential drawbacks, 
  including a more verbose API compared to some other frameworks, limited 
  documentation and examples, complexities introduced by parallelization 
  support, and the use of advanced features that may not be beginner-friendly.
  As of 2023, the community around \textit{GeneticSharp} is smaller, which 
  might lead to fewer resources, less frequent updates, and limited support.
  The framework currently only supports classic genetic algorithms, which may 
  restrict those wishing to explore other evolutionary algorithms.

  Overall, \textit{GeneticSharp} is a powerful tool in the \textit{.NET} 
  ecosystem for developers needing to leverage genetic algorithms in their 
  applications.
  As with any tool, it is essential to understand its capabilities and 
  limitations to make the most effective use of it
