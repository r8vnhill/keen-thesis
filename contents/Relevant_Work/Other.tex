\section{Other Libraries}
\label{sec:sota:other}
  The preceding sections elucidated a select set of leading frameworks in 
  EC.
  Yet, our research ambit encompassed a broader array of libraries.
  This thorough examination was pivotal in sculpting our proposed
  \textit{Kotlin}-based EC framework.
  By discerning the strengths and nuances of each library, we endeavored to 
  assimilate their best features, circumventing common shortcomings.

  Below is a succinct overview of additional frameworks:

  \begin{itemize}
    \item \textbf{EvolvingObjects} (EO)~\autocite{EvolvingObjectsWelcomeEvolving} - 
      A template-based \textit{C++} framework, it encompasses a myriad of evolutionary algorithms and operators, such as GAs, ES, and PSO.
      At present, certain features like GP are not supported.
    \item \textbf{Inspyred}~\autocite{InspyredBioinspiredAlgorithms} - 
      A Python-oriented framework, Inspyred, inspired by De Jong~\autocite{dejongEvolutionaryComputationUnified2006},
      distinctly separates algorithm-specific computations from problem-centric 
      ones.
      It encompasses a diverse array of evolutionary algorithms, including GAs 
      and ES.
      Currently, it does not support GP.
    \item \textbf{Pyevolve} - This Python-centric framework provides multiple 
      evolutionary algorithms, inclusive of GAs and GP.
      However, it's worth noting that it's no longer under active maintenance.
    \item \textbf{PGAPack}~\autocite{SchlatterbeckPgapackParallel} - 
      Crafted in \textit{C}, this parallel genetic algorithm library boasts of 
      features like interoperability with \textit{Fortran}, \textit{C}, and 
      \textit{C++}, compatibility across varied parallel architectures, and a 
      \textit{Python} interface~\autocite{SchlatterbeckPgapyPythona}.
    \item \textbf{pagmo}~\autocite{Pagmo2023} - 
      A \textit{C++} library, pagmo prioritizes massively parallel optimization, 
      offering a unified interface for diverse optimization algorithms, 
      including evolutionary ones.
      It also features a \textit{Python} interface~\autocite{Pygmo2023}.
    \item \textbf{easy\_ga}~\autocite{EasyGaSrc} - 
      Based in \textit{Rust}, easy\_ga facilitates swift GA prototyping, 
      predominantly supporting traditional GAs.
    \item \textbf{genevo}~\autocite{GenevoRust} -
      Another \textit{Rust}-oriented framework, genevo also chiefly supports classic GAs.
    \item \textbf{Evolutionary Computation Framework} (ECF)~\autocite{djakobovicECFEvolutionaryComputation2023} - 
      ECF, written in \textit{C++}, offers a rich set of evolutionary 
      algorithms, rivalling frameworks like DEAP and ECJ in terms of 
      comprehensiveness.
      However, it grapples with issues akin to ECJ, such as limited 
      documentation and an inherent steep learning curve.
  \end{itemize}

  The multifariousness and capabilities intrinsic to these libraries underscore 
  the expansive potential within the realm of genetic algorithms.
  Our deep dive into each has equipped us to craft a \textit{Kotlin}-based 
  framework that endeavors to synergize the strengths of each while introducing 
  innovative attributes.
