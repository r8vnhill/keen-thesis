\section{Other Libraries}
\label{sec:sota:other}
  In the preceding sections, we detailed a select set of premier frameworks 
  within EC.
  However, the scope of our research extended beyond this list to a wider array 
  of libraries.
  This comprehensive survey was instrumental in shaping our proposed 
  \textit{Kotlin}-based EC framework.
  Through understanding the strengths and peculiarities of each library, we 
  aimed to incorporate their best features while sidestepping prevalent 
  pitfalls.

  Below, we provide a concise overview of some additional frameworks:

  \begin{itemize}
    \item \textbf{EvolvingObjects} 
      (EO)~\autocite{EvolvingObjectsWelcomeEvolving}: An object-oriented 
      \textit{C++} framework.
      It offers various evolutionary algorithms and operators, including GAs, 
      ES, and Particle Swarm Optimization (PSO).
      Notably, certain aspects, like GP, are not currently supported.
    \item \textbf{Inspyred}~\autocite{InspyredBioinspiredAlgorithms}: A 
      Python-focused framework inspired by 
      De Jong~\autocite{dejongEvolutionaryComputationUnified2006}.
      It clearly distinguishes between algorithmic computations and 
      problem-specific ones.
      While it supports a range of evolutionary algorithms, GP is not among 
      them.
    \item \textbf{Pyevolve}: A Python-based framework that provides a gamut of 
      evolutionary algorithms, including GAs and GP.
      However, it is currently inactive.
    \item \textbf{PGAPack}~\autocite{SchlatterbeckPgapackParallel}: Crafted in 
      \textit{C}, it is a parallel genetic algorithm library.
      It offers compatibility with \textit{Fortran} and \textit{C++} and 
      a myriad of parallel architectures, it also features a \textit{Python}
      interface~\autocite{SchlatterbeckPgapyPythona}.
    \item \textbf{pagmo}~\autocite{Pagmo2023}: A \textit{C++} library that 
      prioritizes parallel optimization.
      It delivers a cohesive interface for various optimization algorithms and 
      provides a \textit{Python} interface~\autocite{Pygmo2023}.
    \item \textbf{easy\_ga}~\autocite{EasyGaSrc}: Developed in \textit{Rust}, 
      this framework simplifies GA prototyping, primarily focusing on 
      traditional GAs.
    \item \textbf{genevo}~\autocite{GenevoRust}: Another \textit{Rust} 
      framework, it majorly supports classic GAs.
    \item \textbf{Evolutionary Computation Framework} 
      (ECF)~\autocite{djakobovicECFEvolutionaryComputation2023}: ECF is a 
      \textit{C++} framework that provides a comprehensive set of evolutionary 
      algorithms.
      Its depth rivals that of DEAP and ECJ.
      Nonetheless, it faces challenges like limited documentation and a 
      challenging learning curve.
  \end{itemize}

  The breadth and capabilities of these libraries highlight the vast potential 
  within the domain of genetic algorithms.
  Our exhaustive exploration empowers us to develop a \textit{Kotlin}-based 
  framework that amalgamates the strengths of each while pioneering unique 
  features.
