\section{The One Max Problem}
\label{sota:omp}
  For the purpose of illustrating the use of the different frameworks, we will
  use the \textit{One Max problem} introduced on \vref{sec:bg:ga:repr}.

  The \emph{One Max problem} is a classic and straightforward optimization
  problem often used as a benchmark in the study of evolutionary algorithms and
  other heuristic search methods.
  It serves as a deceptively simple yet effective test of an algorithm's
  optimization ability.

  Given a binary string of length $n$, the \textit{One Max problem} is to find a
  binary string such that the sum of its bits (counting the number of ones) is
  maximized.
  In formal terms, if we denote the binary string as $x = (x_1,\, x_2,\, ...,\,
  x_n)$ where each $x_i \in \{0,\,1\}$ for all $i = 1,\, 2,\, \dots,\, n$, the 
  \textit{fitness function} $\phi(x)$ to be maximized can be expressed as:

  \begin{equation}
    \phi(x) = \sum_{i=1}^{n} x_i
  \end{equation}

  The function $\phi(x)$ counts the number of ones in the string $x$.
  The maximum possible value of $\phi(x)$ is $n$, which is achieved when all 
  bits in the string are one.
  The \textit{One Max problem} is an instance of a unimodal problem since
  there's only one local maximum which is also a global maximum.

  It's important to note that despite its simplicity, the \textit{One Max
  problem} does provide a non-trivial task for many search algorithms.
  For a binary string of length $n$, there are $2^n$ possible solutions.
  For larger $n$, an exhaustive search of the solution space is not feasible,
  hence the need for efficient optimization algorithms.

  Due to its characteristics, the \textit{One Max problem} is often used to
  evaluate the performance of optimization algorithms especially genetic and
  evolutionary algorithms.
  It's particularly well-suited for genetic algorithms as the operations of
  crossover and mutation can directly change the number of ones in a binary
  string, thus impacting the fitness of a potential solution.

  Despite the straightforward objective function, the \textit{One Max problem} is
  invaluable in the study of heuristic search methods due to its accessibility,
  simplicity and the vastness of its search space, which permits the analysis
  and comparison of the performance of different optimization techniques.
