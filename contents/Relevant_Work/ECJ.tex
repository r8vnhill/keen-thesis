\section{ECJ: A Java-based Evolutionary Computation Research System}
\label{sec:sota:ecj}
  The \emph{Evolutionary Computation in Java} (ECJ)~\autocite{ECJ} is a powerful and flexible framework for conducting research and experiments in 
  the field of evolutionary computation.
  ECJ has been under active development for more than two decades, and during 
  this period, it has grown into one of the most comprehensive 
  open-source\footnote{No license is provided with the ECJ source code.} 
  libraries for evolutionary computation.

  What sets ECJ apart from other libraries is its ability to handle a broad 
  range of evolutionary computation paradigms.
  It offers support for genetic algorithms, genetic programming, 
  multi-objective optimization, co-evolution, and many more.
  It also comes with built-in functionality for distributed computing, allowing 
  researchers to harness the power of large-scale computing clusters for their 
  experiments.

  The ECJ library is designed with flexibility and extensibility in mind.
  It employs a design pattern that uses parameter files for configuration, thus 
  enabling researchers to customize the algorithms according to their specific 
  needs without having to modify the source code.
  The library is highly modular, and its components can be easily replaced or 
  extended.

  The following code sample demonstrates how to solve the \textit{OneMax} 
  problem using ECJ:\footnote{We couldn't find a way to set the random seed in
  ECJ, so the results are not reproducible.}

  \begin{code}*{ECJ's solution to the OMP}{label=lst:sota:ecj:omp}{java}
    public class OneMax extends Problem implements SimpleProblemForm {
      @Override
      public void evaluate(final EvolutionState state,
                          final Individual ind,
                          final int subpopulation,
                          final int threads) {
        if (ind.evaluated) {
          return;
        }
        if (!(ind instanceof BitVectorIndividual)) {
          state.output.fatal("Individual must be a BitVectorIndividual")
        }
        int sum = 0;
        // This is necessary to safe-cast ind to BitVectorIndividual
        BitVectorIndividual ind2 = null;
        if (ind instanceof BitVectorIndividual) {
          ind2 = (BitVectorIndividual) ind;
        }
        if (ind2 != null) {
          for (int x = 0; x < ind2.genome.length; x++) {
            sum += (ind2.genome[x] ? 1 : 0);
          }

          if (!(ind2.fitness instanceof SimpleFitness)) {
            state.output.fatal("Fitness must be a SimpleFitness")
          }
          ((SimpleFitness) ind2.fitness)
              .setFitness(state,
                  /* fitness = */ sum / (double) ind2.genome.length,
                  /* isIdeal = */ sum == ind2.genome.length);
          ind2.evaluated = true;
        }
      }
    }
  \end{code}

  An explanation of the code:

  \begin{enumerate}
    \item[1] The class \texttt{OneMax} extends \texttt{Problem} and implements 
      \texttt{SimpleProblemForm}, which is ECJ's interface for problems that 
      can be solved with generational evolutionary algorithms.
    \item[2-6] The \texttt{evaluate} method is overridden from the
      \texttt{Problem} class.
      It is responsible for evaluating the fitness of an \texttt{Individual} (a 
      candidate solution in the population).
      The arguments include the current state of the evolution, the individual 
      to be evaluated, the subpopulation to which the individual belongs, and 
      the number of the thread executing this method (useful in a multithreaded 
      setting).
    \item[7-9] The method checks if the individual has already been evaluated.
      If it has, then the method immediately returns to avoid unnecessary 
      computation.
    \item[10-12] If the individual is not an instance of 
      \texttt{BitVectorIndividual} (which represents individuals whose genome 
      consists of bits), a fatal error is reported.
      The OneMax problem assumes a bit string representation of individuals.
    \item[13-18] Variable \texttt{sum} is initialized to keep track of the 
      total number of 1s in the individual's bit string representation.
      The individual is then safe-casted to \texttt{BitVectorIndividual}.
      This is necessary because the \texttt{ind} argument is of type
      \texttt{Individual}, but we need to work with its
      \texttt{BitVectorIndividual} specifics (such as its \texttt{genome} 
      property).
    \item[20-22] If the casting is successful, a loop iterates through the 
      genome (bit string) of the individual, incrementing \texttt{sum} for 
      every bit set to 1.
    \item[24-26] If the fitness of the individual is not an instance of 
      \texttt{SimpleFitness} (which represents a single floating-point fitness 
      value), a fatal error is reported.
    \item[27-30] The individual's fitness is then set to the proportion of bits 
      set to 1 in the bit string 
      (\mintinline{java}{sum / (double) ind2.genome.}\\
      \mintinline{java}{length}).
      If all bits are set to 1 (i.e., \texttt{sum} equals the length of the 
      genome), then the individual is marked as ideal.
    \item[31] Finally, \texttt{ind2.evaluated} is set to \texttt{true}, 
      indicating that the individual has been evaluated.
  \end{enumerate}

  To run the OneMax problem, we need to create a parameter file that specifies 
  the problem, the evolutionary algorithm, and the parameters of the algorithm.
  The following is a sample parameter file for the OneMax problem:

  \begin{code}*{Configuration file for the OMP}{
    label=lst:sota:ecj:omp:config
  }{text}
    breedthreads = 1
    evalthreads = 1
    seed.0 = 11
    eval.problem = ec.app.onemax.OneMax
    pop.subpop.0.species.ind = ec.vector.BooleanVectorIndividual
    pop.subpop.0.species.genome-size = 20
    pop.subpop.0.species.fitness = ec.FitnessSimple
    pop.subpop.0.size = 20
    pop.subpop.0.species.pipe = ec.vector.breed.VectorMutationPipeline
    pop.subpop.0.species.pipe.source.0 = ec.vector.breed.VectorCrossoverPipeline
    pop.subpop.0.species.pipe.source.0.prob = 0.5
    pop.subpop.0.species.pipe.source.0.source.0 = ec.select.TournamentSelection
    pop.subpop.0.species.pipe.source.0.source.1 = ec.select.TournamentSelection
    eval.problem.terminateFitness = 20
  \end{code}

  This is a parameter file for ECJ, which specifies various configuration 
  settings for an evolutionary computation run.
  

  \begin{enumerate}
    \item[1-2] \texttt{breedthreads} and \texttt{evalthreads} determine the 
      number of threads to be used for breeding and evaluation operations, 
      respectively.
      Both are set to 1, which means a single-threaded operation.
    \item[3] \texttt{seed.0} sets the initial seed for the random number 
      generator.
      Different seeds will lead to different runs, even with identical 
      configuration parameters.
    \item[4] \texttt{eval.problem} sets the problem to be solved, in this case, 
      \mintinline{text}{ec.app.onemax.OneMax}, which refers to the OneMax 
      problem class implemented earlier.
    \item[5-6] \texttt{pop.subpop.0.species.ind} sets the type of individual in 
      the population.
      \mintinline{text}{ec.vector.BooleanVectorIndividual} means that each 
      individual will be a Boolean vector, which is appropriate for the OneMax 
      problem.
      \mintinline{text}{pop.subpop.0.species.genome-size} sets the length of 
      the Boolean vector to 20.
    \item[7-8] \texttt{pop.subpop.0.species.fitness} sets the type of fitness 
      to be used.
      \texttt{ec.FitnessSimple} represents a single floating-point fitness 
      value.
      \texttt{pop.subpop.0.size} sets the population size to 20.
    \item[9] \texttt{pop.subpop.0.species.pipe} defines the pipeline for 
      breeding new individuals.
      \mintinline{text}{ec.vector.breed.VectorMutationPipeline} means a 
      mutation operation will be applied.
    \item[10] \texttt{pop.subpop.0.species.pipe.source.0} sets the source of 
      individuals for the mutation operation to be a crossover operation 
      \\(\texttt{ec.vector.breed.VectorCrossoverPipeline}), with a crossover 
      probability of 0.5.
    \item[12-13] \texttt{pop.subpop.0.species.pipe.source.0.source.0} and 
      \texttt{pop.subpop.0.species.pipe.source.0.source.1} set the source of 
      individuals for the crossover operation to be tournament selection 
      (\texttt{ec.select.TournamentSelection}).
    \item[14] \texttt{eval.problem.terminateFitness} sets the fitness level at 
      which the evolutionary run should terminate.
      When an individual reaches a fitness of 20, the run will stop, implying 
      that a perfect solution (all bits set to 1) has been found for the OneMax 
      problem.
  \end{enumerate}

  To run the OneMax problem, we need to execute the following command:

  \begin{code}{Running the OMP with ECJ}{label=lst:sota:ecj:omp:run}{powershell}
    java -cp .
    ec.Evolve -file .\ec\app\onemax\one_max.properties  
  \end{code}

  This command will run the OneMax problem with the parameter file
  \texttt{one\_max.properties} in the package \texttt{ec.app.onemax}.
  According to the official ECJ documentation, this command should be able to 
  solve the OneMax problem, however, we were unable to get it to work.
  The following is the output of the command:

  \begin{code}{
    Output of the command in \cref{lst:sota:ecj:omp:run}
  }{label=lst:sota:ecj:omp:run:output}{text}
    # Code omitted for brevity
    Exception in thread "main" ec.util.ParamClassLoadException:
    No class name provided.
    PARAMETER: state
            at ec.util.ParameterDatabase.getInstanceForParameter(ParameterDatabase.java:493)
            at ec.Evolve.initialize(Evolve.java:479)
            at ec.Evolve.initialize(Evolve.java:412)
            at ec.Evolve.main(Evolve.java:758)
  \end{code}

  The error message indicates that the parameter \texttt{state} is missing.
  This parameter is required by ECJ, but is not specified in the parameter file.
  We tried to add the parameter to the file, but the error persisted.
  We also tried to run the OneMax problem with the parameter file provided by 
  the ECJ documentation, but the error persisted.
  We were unable to find a solution to this problem.

  While ECJ stands as one of the most comprehensive and sophisticated 
  Evolutionary Computation (EC) frameworks available, it presents users with a 
  daunting learning curve.
  We've identified several areas of concern that contribute to this complexity:

  \begin{itemize}
    \item \textbf{Verbose and obscure code:} The framework's verbosity hinders 
      code comprehension.
      To illustrate, the \texttt{eval} method in the \texttt{OneMax} class 
      spans 31 lines, yet only 3 of these lines (20-22) actually pertain to the 
      OneMax problem itself—the rest are framework-related.
    \item \textbf{Complex configuration:} Setting up the framework involves 
      numerous layers of boilerplate code, necessitating a deep understanding 
      of the underlying concepts and terminology.
      Lines such as 
      \mintinline{text}{pop.subpop.0.species.pipe.source.0.source.0} can be 
      particularly opaque to newcomers.
    \item \textbf{Outdated documentation and implementation:} The 
      \texttt{README.md} file has not been updated in 4 years, and the OneMax 
      class equivalent was last refreshed 5 years ago.
      Even the most recent version (27) of the framework dates back 4 years, 
      rendering much of the documentation and implementation obsolete.
    \item \textbf{Legacy Java syntax:} ECJ's reliance on older Java syntax, 
      specifically the lack of generics, results in frequent casting and 
      \mintinline{java}{instanceof} checks.
      This not only makes the code more brittle but also complicates 
      understanding.
    \item \textbf{Exclusion from popular package managers:} The absence of ECJ 
      from widely-used package managers like \textit{Maven} or \textit{JitPack} 
      imposes additional burdens on integration with other projects, as manual 
      download and compilation become necessary.
    \item \textbf{Limited IDE compatibility:} The design of ECJ emphasizes 
      command line compilation and execution, undermining the utility of modern 
      Integrated Development Environments (IDEs) and complicating debugging 
      efforts.
    \item \textbf{Inconvenience for integration:} Rather than being crafted for 
      use as a library, the framework is designed with the expectation of 
      direct source code inclusion.
      This requirement to import source code into users' own projects increases 
      the difficulty of integration.
    \item \textbf{Generic name complicates research:} The term \enquote{ECJ} is 
      sufficiently common to cause confusion when seeking information on the 
      framework.
      Queries such as \enquote{ECJ} or \enquote{ECJ Java} yield results 
      concerning the \textit{European Court of Justice} or the \textit{Eclipse 
      Compiler for Java}, respectively, rather than the intended framework.
    \item \textbf{Limited community and resources:} As of 2023, ECJ lacks a 
      robust community of developers or users.
      The dearth of online resources like tutorials or Stack Overflow 
      discussions compounds the challenges for new users seeking to leverage 
      the framework effectively.
    \item \textbf{Non-utilization of modern Java features:} Beyond the lack of 
      generics, the framework fails to harness modern Java features like lambda 
      expressions, lists, or Optional types.
      These features, by enabling more concise and readable code, can 
      significantly enhance error prevention.
  \end{itemize}

  In conclusion, \textit{the Evolutionary Computation in Java} (ECJ) framework, 
  while extensive and capable, poses considerable barriers to entry for users 
  due to its complexity, outdated documentation, and heavy reliance on legacy 
  \textit{Java} syntax.
  Furthermore, despite its versatility in supporting various evolutionary 
  computation paradigms and distributed computing, the steep learning curve, 
  absence from popular package managers, and limited compatibility with modern 
  IDEs make it challenging for newcomers and experienced developers alike.
  The verbose code, intricate configuration requirements, and limited use of 
  modern \textit{Java} features also contribute to the difficulty in understanding and 
  integrating the ECJ into projects.
  Our exploration further exposed issues with the execution of sample problems, 
  highlighting the need for updated documentation and maintenance.
  However, ECJ's modular design, extensibility, and breadth of capabilities 
  maintain its status as a valuable tool for research in the field of 
  evolutionary computation.
  Future efforts could focus on modernizing the framework, improving 
  documentation, and fostering an active user community to mitigate the 
  identified issues and enhance its usability.
