\section{Problem Description}
\label{sec:knapsack:probl}
  The \emph{Knapsack Problem} is a canonical combinatorial optimization problem. 
  It involves selecting items, each characterized by a specific weight and 
  value, to maximize the total value while adhering to a weight constraint. 

  The goal is to ascertain the optimal subset of items (maximize profit) for the 
  knapsack, ensuring the total weight remains under the knapsack's capacity.

  Two primary variants of the Knapsack Problem exist: \emph{0-1 Knapsack}
  (\(\mathrm{K_{0 / 1}}\)) and \emph{Unbounded Knapsack} (\(\mathrm{K_{*}}\)).

  \subsection{0-1 Knapsack Problem}
    In this variant, each item can be taken once or disregarded.
    This constraint can be expressed as \(x_i \in \{0, 1\}\).
    The mathematical formulation is:

    \begin{equation}
      \label{eq:knapsack:01}
      \begin{aligned}
        \text{maximize} \quad & \sum_{i=1}^{n} v_i x_i \\
        \text{subject to} \quad & \sum_{i=1}^{n} w_i x_i \leq W \\
          & x_i \in \{0, 1\} \quad \forall i \in \{1, 2, \dots, n\}
      \end{aligned}
    \end{equation}

    Analyzing the problem's search space, we note that the potential 
    combinations of selected items equate to a size of \(2^n\).
    Given that the \emph{Subset Sum Problem}, to which this problem is 
    reducible, is NP-hard, \(\mathrm{K}_{0/1}\) inherits this complexity.

    \paragraph{Example:}
      Consider 8 items \((w,\, v) = \{(11,\, 1),\, (21,\, 11),\, (31,\, 21),\, 
      (33,\, 23),\, (43,\, 33),\, (53,\, 43),\, (55,\, 45),\, (65,\, 55)\}\). For a knapsack with a 
      capacity \(W = 110\), the optimal selection consists of items 1, 3, 4, 5, 
      and 6.
      This results in a cumulative value of 159 and a total weight of 109.


  \textbf{Unbounded Knapsack Problem:}
  Here, there's no limit to the number of instances of each item that can be selected.

  \textbf{Example:}
  Given items with weights [10, 20, 30] and values [60, 100, 120] and \(W = 100\), one can pick the first item three times to get the maximum value of 180.

  \textbf{Search Space Dimensions:}
  For the 0-1 Knapsack, the search space is of size \(2^n\), making the problem NP-hard. For the Unbounded Knapsack, the search space can be considered infinite since there's no limit on the number of each item.
