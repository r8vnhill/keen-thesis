\section{Solution Strategies for Knapsack Problems}
\label{sec:knapsack:sol}

  Given two distinct knapsack problem variants, our approach encompasses unique 
  solutions tailored to each.
  We employ the \textit{Keen} framework's built-in Genetic Algorithm (GA) 
  capabilities for the \(\mathrm{K}_{0 / 1}\) problem and augment the
  \textit{Keen} framework for the \( \mathrm{K}_{*} \) problem.

  \subsection{0-1 Knapsack Problem}
  \label{sec:knapsack:sol:01}

  To solve the 0-1 Knapsack problem:

  \begin{enumerate}
    \item \textbf{Problem Representation:} We represent each item's inclusion in 
      the knapsack as a binary string.
      This string is facilitated by the \texttt{BoolGene} class, while the 
      collection of these genes is managed by the \texttt{BoolChromosome} class, 
      essentially acting as a wrapper for the \texttt{BoolGene}.
    \item \textbf{Item Representation:} Each item is captured as a \texttt{Pair} 
      of integers, with the first element indicating the item's value and the 
      second indicating its weight.
    \item \textbf{Fitness Function Definition:} The core objective of the 
      fitness function is to maximize the total value, ensuring the aggregated 
      weight does not breach the knapsack's weight limit.
      Should the weight exceed the allowed capacity, a proportional penalty is 
      applied to the fitness score. 
  \end{enumerate}

  The fitness function for this problem, implemented in Kotlin, is as follows:

  \begin{code}{Fitness Function for the 0/1 Knapsack Problem via Keen}{
    label={lst:knapsack:sol:01:fitness}
  }{kotlin}
    fun fitnessFn(genotype: Genotype<Boolean, BoolGene>): Double {
        val profit = (genotype.flatten() zip items).sumOf { (isInBag, item) ->
            if (isInBag) item.first else 0
        }
        val weight = (genotype.flatten() zip items).sumOf { (isInBag, item) ->
            if (isInBag) item.second else 0
        }
        val penalty = if (weight > MAX_WEIGHT) abs(weight - MAX_WEIGHT) else 0
        return (profit - penalty).toDouble()
    }
  \end{code}
