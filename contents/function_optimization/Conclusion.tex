\section{Conclusion}
\label{sec:fn_opt:conclusion}

    In wrapping up this analysis, the \textit{Keen} framework's Tournament Selector is distinguished by its robust 
    performance in consistently achieving and maintaining optimal solutions, as evidenced by its minimal error rates 
    across various functions. This demonstrates its versatility and effectiveness in a range of optimization 
    challenges. Conversely, the Random and Roulette Selectors, though occasionally reaching the global optima, 
    demonstrate limitations in sustaining these solutions, suggesting the advantage of a target fitness criterion for 
    these methods over the stagnation termination criterion more suited to the Tournament Selector. The uniformity in 
    selection times and evolutionary results across all selectors underscores the \textit{Keen} framework's reliability, although the variance in performance against functions with numerous local optima emphasizes the importance of strategic selector choice tailored to the problem's specifics.

    These findings reiterate the practical strengths of various selection strategies and the \textit{Keen} framework's 
    comprehensive capabilities, guiding their application in computational tasks. While the analysis aligns with 
    established research, it underscores the framework's ready-to-use functionality and efficiency in problem-solving, 
    contributing to the broader discourse in evolutionary computation.

    Furthermore, the implementation showcases the framework's adaptability to different optimization functions, 
    selection strategies, and operators, facilitated by its modular design. This adaptability, along with features such 
    as first-class function support and the capability to tailor genes to specific search spaces, enhances the 
    framework's utility for diverse optimization tasks. Notably, the framework's provision for visualizing fitness 
    evolution offers a unique tool for debugging and analysis, setting it apart from other frameworks.
