\section{Conclusion}
\label{sec:fn_opt:conclusion}

    In summarizing the findings of this study, it's evident that the Tournament Selector within the \textit{Keen} 
    framework stands out for its effectiveness in securing and holding onto optimal solutions, marked by the lowest 
    average errors across tested functions. This indicates its strong adaptability and reliability in various 
    optimization scenarios. On the other hand, the Random and Roulette Selectors, despite their occasional success in 
    reaching global optima, fall short in consistently maintaining these solutions over time. This underscores the 
    potential benefit of employing a target fitness criterion for these strategies, as opposed to the stagnation 
    termination criterion that suits the Tournament Selector. The consistent performance in selection times and 
    evolutionary outcomes across all selectors attests to the \textit{Keen} framework's stability. Yet, the mixed 
    results in handling functions with numerous local optima underscore the necessity of selecting the right strategy 
    based on the specific demands of the optimization problem. These insights not only shed light on the comparative 
    strengths and weaknesses of different selection strategies but also provide practical direction for their 
    application in a wide range of computational tasks.

    It's crucial to acknowledge that the comparative analysis and performance evaluation presented here align with 
    existing studies in the field. The primary aim is to highlight the framework's capability for efficient 
    problem-solving and its ready-to-use solutions, rather than to introduce novel findings in algorithmic performance.