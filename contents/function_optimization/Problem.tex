\section{Problem Description}
\label{sec:fn_opt:problem}
    This chapter focuses on optimizing various real functions, detailed in \vref{app:test_functions}, chosen for their 
    complexity and prevalence in optimization studies. Our objective is to identify the \textbf{global minimum} for 
    these functions, addressing what is known as the global optimization challenge.

    The core problem involves finding a point \(x^* \in \mathbb{R}^n\) for a function \(f : \mathbb{R}^n \to 
    \mathbb{R}\), where \(f(x^*)\) is the lowest possible value of \(f\). This is termed the \textbf{global minimum} of 
    \(f\), and the task of locating this minimum is the \textit{function optimization problem}.

    The search space, or the realm of potential solutions, varies by function. For instance, the search space for the 
    \textit{Cross-in-Tray} function is \([-10,\,10]^2\), while for the \textit{Easom} function, it's \([-100,\,100]^2\).

    Understanding the concepts of sets, cardinalities, and functions is essential due to the diverse nature of our 
    functions and their search spaces. These foundational concepts aid in grasping the search space's complexity and 
    are crucial for employing the \textit{Keen} framework in these optimization tasks.
    
    The following definitions, theorems, and corollaries will elucidate the necessary background.

    \begin{definition}[Set cardinality inequality]
    \label{def:fn_opt:cardinality_inequality}
        For any two sets \(A\) and \(B\), if there exists an injective function \(f : A \to B\), then \(|A| \leq |B|\).
    \end{definition}

    \begin{theorem}[Schröder--Bernstein theorem]
    \label{thm:fn_opt:schroder_bernstein}
        Given two sets \(A\) and \(B\), if there exist two injective functions \(f : A \to B\) and \(g : B \to A\), 
        then there exists a bijective function \(h : A \to B\).
    \end{theorem}

    \begin{corollary}
    \label{cor:fn_opt:cardinality}
        For any two sets \(A\) and \(B\), if \(|A| \leq |B|\) and \(|B| \leq |A|\), then \(|A| = |B|\).
    \end{corollary}

    \begin{proof}
        The conditions \(|A| \leq |B|\) and \(|B| \leq |A|\) imply that there exist injective functions 
        \(f: A \rightarrow B\) and \(g: B \rightarrow A\), respectively. By the Schröder--Bernstein Theorem 
        (\vref{thm:fn_opt:schroder_bernstein}), the existence of these injective functions guarantees a bijective 
        function \(h: A \rightarrow B\).

        A bijective function is one that is both injective and surjective. This means that every element of \(A\) is 
        mapped to a unique element in \(B\) and every element of \(B\) is the image of some element in \(A\).

        Therefore, there can't be more elements in \(A\) than in \(B\) (or vice versa), as this would contradict the 
        surjectivity or the injectivity of the function \(h\). Thus, by the definition of set cardinality 
        (\vref{def:fn_opt:cardinality_inequality}), we conclude that \(|A| = |B|\).
    \end{proof}

    \begin{theorem}
    \label{thm:fn_opt:cardinality}
        Let $(x,\,y) \in \mathbb{R}^2$ with $x \neq y$. Then, the cardinality of the interval $[x,\,y]$, denoted as 
        $|[x,\,y]|$, is the same as the cardinality of \(\mathbb{R}\). Thus, $|[x,\,y]|$ is uncountable.
    \end{theorem}

    \begin{proof}
        Let us define a function \(f: [x,\,y] \rightarrow \mathbb{R}\) defined as \(f(z) = 
        \tan\left(\frac{\pi(z-x)}{y-x} - \frac{\pi}{2}\right)\) which maps the interval \([x, y]\) bijectively onto 
        \(\mathbb{R}\).
        
        This means that there is an injective function \(f: [x,\,y] \rightarrow \mathbb{R}\) and an injective function 
        \(g: \mathbb{R} \rightarrow [x,\,y]\).

        By the definition of cardinality inequality (\vref{def:fn_opt:cardinality_inequality}), we conclude that 
        \(|[x,\,y]| \leq |\mathbb{R}|\) and \(|\mathbb{R}| \leq |[x,\,y]|\).

        By \vref{thm:fn_opt:schroder_bernstein}, we conclude that \(|[x,\,y]| = |\mathbb{R}|\).
    \end{proof}    

    These results make it clear that, in most cases, the search space for our real function optimization problems will 
    have an uncountable number of potential solutions. This vastness and complexity underline the necessity for robust 
    and efficient optimization methods like those provided by \textit{Keen}.