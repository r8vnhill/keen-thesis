\section{Problem Description}
\label{sec:fn_opt:problem}
    The primary task in this chapter revolves around the optimization of several real functions. These functions, which 
    can be found in \vref{app:test_functions}, are selected based on their complexity and their widespread use in the 
    field of optimization. We aim to find the \textbf{global minimum} for each of these functions, a problem known as 
    global optimization.

    In the context of this case study, the search space is defined as the set of all possible solutions to the 
    optimization problem. Since we are dealing with real-valued functions, the search space will be a subset of the set 
    of real numbers, \(\mathbb{R}\). The exact subset that forms the search space can vary depending on the function we 
    are optimizing. For example, the \textit{Cross-in-Tray} function has a search space of \([-10,\,10]^2\), whereas the 
    \textit{Easom} function has a search space of \([-100,\,100]^2\).
    
    Given the nature of our functions and their respective search spaces, it is vital to understand some fundamental 
    concepts related to sets, cardinalities, and functions. These concepts will not only help us better comprehend the 
    vastness and complexity of the search space but will also lay the groundwork for the use of \textit{Keen} in this 
    context.
    
    The following definitions, theorems, and corollaries will elucidate the necessary background.

    \begin{definition}[Set cardinality inequality]
    \label{def:fn_opt:cardinality_inequality}
        For any two sets \(A\) and \(B\), if there exists an injective function \(f : A \to B\), then \(|A| \leq |B|\).
    \end{definition}

    \begin{theorem}[Schröder--Bernstein theorem]
    \label{thm:fn_opt:schroder_bernstein}
        Given two sets \(A\) and \(B\), if there exist two injective functions \(f : A \to B\) and \(g : B \to A\), then 
        there exists a bijective function \(h : A \to B\).
    \end{theorem}

    \begin{corollary}
    \label{cor:fn_opt:cardinality}
        For any two sets \(A\) and \(B\), if \(|A| \leq |B|\) and \(|B| \leq |A|\), then \(|A| = |B|\).
    \end{corollary}

    \begin{proof}
        The conditions \(|A| \leq |B|\) and \(|B| \leq |A|\) imply that there exist injective functions 
        \(f: A \rightarrow B\) and \(g: B \rightarrow A\), respectively. By the Schröder--Bernstein Theorem 
        (\vref{thm:fn_opt:schroder_bernstein}), the existence of these injective functions guarantees a bijective 
        function \(h: A \rightarrow B\).

        A bijective function is one that is both injective and surjective. This means that every element of \(A\) is 
        mapped to a unique element in \(B\) and every element of \(B\) is the image of some element in \(A\).

        Therefore, there can't be more elements in \(A\) than in \(B\) (or vice versa), as this would contradict the 
        surjectivity or the injectivity of the function \(h\). Thus, by the definition of set cardinality 
        (\vref{def:fn_opt:cardinality_inequality}), we conclude that \(|A| = |B|\).
    \end{proof}

    \begin{theorem}
    \label{thm:fn_opt:cardinality}
        Let $(x,\,y) \in \mathbb{R}^2$ with $x \neq y$.
        Then, the cardinality of the interval $[x,\,y]$, denoted as $|[x,\,y]|$,
        is the same as the cardinality of \(\mathbb{R}\).
        Thus, $|[x,\,y]|$ is uncountable.
    \end{theorem}

    \begin{proof}
        Let us define a function \(f: [x,\,y] \rightarrow \mathbb{R}\) defined 
        as \(f(z) = \tan\left(\frac{\pi(z-x)}{y-x} - \frac{\pi}{2}\right)\) 
        which maps the interval \([x, y]\) bijectively onto \(\mathbb{R}\).
        
        This means that there is an injective function \(f: [x,\,y] \rightarrow
        \mathbb{R}\) and an injective function \(g: \mathbb{R} \rightarrow
        [x,\,y]\).

        By the definition of cardinality inequality
        (\vref{def:fn_opt:cardinality_inequality}), we conclude that
        \(|[x,\,y]| \leq |\mathbb{R}|\) and \(|\mathbb{R}| \leq |[x,\,y]|\).

        By \vref{thm:fn_opt:schroder_bernstein}, we conclude that
        \(|[x,\,y]| = |\mathbb{R}|\).
    \end{proof}    

    These results make it clear that, in most cases, the search space for our
    real function optimization problems will have an uncountable number of
    potential solutions.
    This vastness and complexity underline the necessity for robust and
    efficient optimization methods like those provided by \textit{Keen}.