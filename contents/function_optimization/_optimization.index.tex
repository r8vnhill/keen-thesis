\chapter{Case Study: Real Function Optimization}
\label{chap:fn_opt}
  \section{Introduction}
  \label{sec:fn_opt:intro}
    Real function optimization is a prevalent task in numerous fields, from 
    \textit{data science} and \textit{machine learning} (ML) to 
    \textit{operations research} and \textit{engineering}.
    It presents a common class of challenges that can be effectively addressed
    using GAs.

    This chapter embarks on a practical exploration of \textit{Keen}.
    We aim to illustrate the robust capabilities of \textit{Keen} by employing
    it to solve various classic optimization problems, the details of which are
    provided in \vref{app:test_functions}.

    Please note that this chapter doesn't intend to delve into the formulation
    of these optimization problems; the discussion is primarily centered around
    how \textit{Keen} can be leveraged as a solution tool.
    For a comprehensive understanding of \textit{Keen}, its design, and its
    features, refer to \vref{chap:keen}.
    
    The ensuing sections will navigate through the problem descriptions, the
    solutions employed using \textit{Keen}, the corresponding results, and the
    consequent analysis.
    In the following sections, we will systematically explore and demonstrate
    the efficiency and versatility of \textit{Keen} in tackling a range of
    optimization problems.

  \section{Problem Description}
  \label{sec:fn_opt:problem}
    \begin{definition}[Set cardinality inequality]
    \label{def:fn_opt:cardinality_inequality}
      For any two sets \(A\) and \(B\), if there exists an injective function
      \(f : A \to B\), then \(|A| \leq |B|\).
    \end{definition}

    \begin{theorem}[Schröder--Bernstein theorem]
    \label{thm:fn_opt:schroder_bernstein}
      Given two sets \(A\) and \(B\), if there exist two injective functions
      \(f : A \to B\) and \(g : B \to A\), then there exists a bijective
      function \(h : A \to B\).
    \end{theorem}

    The proof for \vref{thm:fn_opt:schroder_bernstein} is attributed to J.
    König~\autocite{konigTheorieEnsembles1906}.

    \begin{corollary}
    \label{cor:fn_opt:cardinality}
      For any two sets \(A\) and \(B\), if \(|A| \leq |B|\) and \(|B| \leq 
      |A|\), then \(|A| = |B|\).
    \end{corollary}

    \begin{proof}
      The conditions \(|A| \leq |B|\) and \(|B| \leq |A|\) imply that there
      exist injective functions \(f: A \rightarrow B\) and \(g: B
      \rightarrow A\), respectively.
      By the Schröder--Bernstein Theorem (\vref{thm:fn_opt:schroder_bernstein}),
      the existence of these injective functions guarantees a bijective function
      \(h: A \rightarrow B\).

      A bijective function is one that is both injective and surjective.
      This means that every element of \(A\) is mapped to a unique element in
      \(B\) and every element of \(B\) is the image of some element in \(A\).

      Therefore, there can't be more elements in \(A\) than in \(B\) (or vice
      versa), as this would contradict the surjectivity or the injectivity of
      the function \(h\).
      Thus, by the definition of set cardinality
      (\vref{def:fn_opt:cardinality_inequality}), we conclude that \(|A| =
      |B|\).
    \end{proof}

    \begin{theorem}
    \label{thm:fn_opt:cardinality}
      Let $(x,\,y) \in \mathbb{R}$ with $x \neq y$.
      Then, the cardinality of the interval $[x,\,y]$, denoted as $|[x,\,y]|$,
      is the same as the cardinality of \(\mathbb{R}\).
      Thus, $|[x,\,y]|$ is uncountable.
    \end{theorem}

    \begin{proof}
      Let us define a function \(f: [x,\,y] \rightarrow \mathbb{R}\) defined 
      as \(f(z) = \tan\left(\frac{\pi(z-x)}{y-x} - \frac{\pi}{2}\right)\) 
      which maps the interval \([x, y]\) bijectively onto \(\mathbb{R}\).
      
      This means that there is an injective function \(f: [x,\,y] \rightarrow
      \mathbb{R}\) and an injective function \(g: \mathbb{R} \rightarrow
      [x,\,y]\).

      By the definition of cardinality inequality
      (\vref{def:fn_opt:cardinality_inequality}), we conclude that
      \(|[x,\,y]| \leq |\mathbb{R}|\) and \(|\mathbb{R}| \leq |[x,\,y]|\).

      By \vref{thm:fn_opt:schroder_bernstein}, we conclude that
      \(|[x,\,y]| = |\mathbb{R}|\).
    \end{proof}    

  \section{Solution}
  \label{sec:fn_opt:sol}
    \Blindtext
  \section{Results}
  \label{sec:fn_opt:results}
    \Blindtext
  \section{Conclusion}
  \label{sec:fn_opt:conclusion}
    \Blindtext
