\section{Objectives}
\label{sec:objectives}
  The primary objectives of this thesis are set forth to establish a solid foundation for the design, implementation, 
  and evaluation of a novel, efficient, and user-friendly genetic algorithms framework.

  \subsection{Main Objective}
    The main objective of this study is to develop a genetic algorithms framework that is versatile, efficient, and
    user-friendly. The framework should be able to support classical GAs, as seen in \vref{sec:bg:ga}, and be easily 
    extensible to support future enhancements and expansions such as new algorithms, like GP, and new genetic operators.

  \subsection{Specific Objectives}
    Building upon the primary objectives, the following specific objectives have been established to guide this 
    research:
    
    \begin{enumerate}
      \item \textbf{Framework Efficiency:} Design the genetic algorithms framework to optimize computational efficiency.
        This should involve algorithmic improvements, effective use of the language features, and careful resource 
        management.
      \item \textbf{Comparative Study:} Conduct comparative studies between the proposed genetic algorithms framework
        and existing solutions, particularly in terms of syntactical complexity, and ease of use.
        This should provide a benchmark to evaluate the advantages and potential areas of improvement for our framework.
      \item \textbf{Real-world Applications:} Apply the proposed framework to address complex real-world problems.
        This will not only demonstrate the practical viability of the framework but also its potential to contribute to
        advancements in AI and EC.
      \item \textbf{Future Enhancements and Expansion:} Anticipate and propose potential enhancements and extensions 
        for the genetic algorithms framework. Discuss its adaptability to incorporate future advancements in AI and 
        EC, ensuring its sustainability and continued relevance.
    \end{enumerate} 
    