\chapter{Introduction}
\label{chap:introduction}
  \section{Motivation}
  \label{sec:motivation}
    The birth of artificial intelligence (AI) as we understand it today can be
    traced back to the mid-20th century, when Alan Turing posed the provocative
    question: \enquote{Can machines
    think?}~\autocite{turingCOMPUTINGMACHINERYINTELLIGENCE1950a}.
    From that moment on, the field has grown dramatically, developing an
    extensive range of applications that continue to expand and evolve.

    One of the early attempts to simulate evolution artificially dates back to
    1954 when Nills Aall
    Barricelli~\autocite{barricelliNumericalTestingEvolution1962} defined
    evolution as a purely statistical process.
    This pioneering work laid the foundations for the field of
    \emph{evolutionary computation} (EC), setting the criteria for a system to
    be deemed evolutionary.
    Barricelli posited that such a system's components must be capable of
    reproduction and mutation, enabling evolution via natural selection or
    survival of the fittest.

    Furthering these principles, John Holland's seminal work, \emph{Adaptation
    in Natural and Artificial
    Systems}~\autocite{hollandAdaptationNaturalArtificial1992a}, introduced
    \emph{genetic algorithms} (GA), a revolutionary approach to solving
    optimization problems inspired by the mechanisms of natural selection and
    genetics.
    This established the foundation for the field of \emph{evolutionary
    algorithms} (EA), spurring the development of numerous techniques and
    algorithms, including genetic algorithms, genetic programming, and
    evolutionary strategies.

    Despite the demonstrated success of evolutionary algorithms in solving
    complex optimization problems, from scheduling and data mining to machine
    learning and beyond, their implementation can be cumbersome, often requiring
    a significant amount of repetitive coding.
    As AI adoption continues to grow, so does the need for streamlined,
    user-friendly libraries and frameworks that enable efficient experimentation
    and implementation of these powerful algorithms.

    One promising platform for the development of such a framework is the
    \emph{Kotlin programming language}~\autocite{KotlinProgrammingLanguagea}.
    \textit{Kotlin} boasts an intuitive syntax, static typing, and seamless
    interoperability with
    \textit{Java}~\autocite{CallingJavaKotlin,CallingKotlinJava,KotlinJavaInteropGuide},
    making it a compelling choice for AI and EC practitioners.
    The language's stature has been recognized officially in Android app
    development~\autocite{AndroidKotlinfirstApproach}, and it enjoys growing
    popularity among developers, with over 60\% of Android developers now using
    it~\autocite{LatestProgrammingLanguages2023}.
    Additionally, Kotlin is ranked as the ninth fastest-growing language
    globally from 2021 to 2023, according to a study by
    GitHub~\autocite{TopProgrammingLanguages}, surpassing Python by 0.4\%.

    The Kotlin Multiplatform feature further enhances the language's appeal by
    enabling code sharing across a multitude of platforms, including JVM,
    JavaScript, Android, iOS, and native desktop applications.
    This capability fosters increased productivity and consistency in
    development work.

    Given these strengths, the main goal of this thesis is to leverage
    \textit{Kotlin} to design and implement a novel genetic algorithms
    framework, contributing to the ever-expanding usage of \textit{Kotlin} in
    the AI community.
    We anticipate that this framework will expedite research and applications in
    the field, accelerating the pace of innovation in evolutionary computation.

  \section{Hyphotesis and Research Questions}
  \label{sec:hyphotesis_and_research_questions}
    \Blindtext
  \section{Objectives}
  \label{sec:objectives}
    \Blindtext
  \section{Methodology}
  \label{sec:methodology}
    \Blindtext
  \section{Structure}
  \label{sec:structure}
    \Blindtext