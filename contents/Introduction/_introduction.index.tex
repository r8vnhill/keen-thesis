\chapter{Introduction}
\label{chap:introduction}
  \section{Motivation}
  \label{sec:motivation}
    The notion of learning machines has been around since the 1950s, when Alan Turing 
    \autocite{turingCOMPUTINGMACHINERYINTELLIGENCE1950a} asked a simple question: 
    \enquote{Can machines think?}.
    Since then, the field of artificial intelligence has grown exponentially, with new
    applications being developed every day.
    
    The first ideas on artificially simulating evolution date back to 1954, when Nills Aall
    Barricelli~\autocite{barricelliNumericalTestingEvolution1962} proposed a definition of
    evolution as a pure statistical process, establishing the basis for the field of
    \emph{evolutionary computation} (EC).
    In this work, Baricelly states that in order for a system to be considered an evolutionary
    system, its components must satisfy the following conditions:

    \begin{enumerate}
      \item Be able to reproduce.
      \item Be able to und-ergo hereditary changes (mutations) in order to permit evolution by a 
        process of survival of the fittest.
    \end{enumerate}

    John Holland~\autocite{hollandAdaptationNaturalArtificial1992a} will later formalize these
    ideas in his book \emph{Adaptation in Natural and Artificial Systems}, where he introduces
    the concept of \emph{genetic algorithms} (GA), a method for solving optimization problems
    based on the principles of natural selection and genetics.
    Holland's work will be the basis for the field of \emph{evolutionary algorithms} (EA).

  \section{Hyphotesis and Research Questions}
  \label{sec:hyphotesis_and_research_questions}
    \Blindtext
  \section{Objectives}
  \label{sec:objectives}
    \Blindtext
  \section{Methodology}
  \label{sec:methodology}
    \Blindtext
  \section{Structure}
  \label{sec:structure}
    \Blindtext