\chapter{Introduction}
\label{chap:introduction}
  \section{Motivation}
  \label{sec:motivation}
    The notion of learning machines has been around since the 1950s, when Alan Turing 
    \autocite{turingCOMPUTINGMACHINERYINTELLIGENCE1950a} asked a simple question: 
    \enquote{Can machines think?}.
    Since then, the field of artificial intelligence has grown exponentially, with new
    applications being developed every day.
    
    The first ideas on programs that simulate natural evolution were proposed by Nills Aall 
    Barricelli \autocite{barricelliNumericalTestingEvolution1962} in 1954, but was formalized (and
    popularized) by John Holland \autocite{hollandAdaptationNaturalArtificial1992a} in 1975 on his
    book about genetic algorithms.

    \blindtext

  \section{Hyphotesis and Research Questions}
  \label{sec:hyphotesis_and_research_questions}
    \Blindtext
  \section{Objectives}
  \label{sec:objectives}
    \Blindtext
  \section{Methodology}
  \label{sec:methodology}
    \Blindtext
  \section{Structure}
  \label{sec:structure}
    \Blindtext