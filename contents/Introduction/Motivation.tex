\section{Motivation}
\label{sec:motivation}
  The inception of artificial intelligence (AI) as we perceive it today is deeply rooted in the mid-20th century when 
  Alan Turing proposed the intriguing question, \enquote{Can machines 
  think?}~\autocite{turingCOMPUTINGMACHINERYINTELLIGENCE1950a}. This seminal query catalyzed a remarkable evolution in 
  the field, leading to the emergence of a diverse spectrum of AI applications that persistently extend and advance.

  Among the early endeavors to emulate evolution artificially, Nills Aall Barricelli's initiative in 1954 stands out.
  Barricelli construed evolution as a purely statistical process, setting a novel precedent for the field of 
  evolutionary computation (EC)~\autocite{barricelliNumericalTestingEvolution1962}. This groundbreaking exploration 
  established the parameters for an evolutionary system, which necessitated the system's components to exhibit 
  capabilities of reproduction and mutation, thereby facilitating evolution through natural selection or the survival 
  of the fittest.

  Expanding upon these foundational principles, John Holland's influential book, \emph{Adaptation in Natural and 
  Artificial Systems}~\autocite{hollandAdaptationNaturalArtificial1992a}, unveiled genetic algorithms (GA). This 
  innovative methodology for addressing optimization problems drew inspiration from natural selection and genetic 
  mechanisms. Holland's work solidified the field of evolutionary algorithms (EA), igniting the development of a 
  plethora of techniques and algorithms, including but not limited to, genetic algorithms, genetic programming, and 
  evolutionary strategies.

  Despite the proven efficacy of evolutionary algorithms in resolving intricate optimization problems ranging from 
  scheduling and data mining to machine learning and beyond, their execution can be challenging, frequently demanding 
  extensive, repetitive coding. As the uptake of AI continues to soar, the requirement for streamlined,  intuitive 
  libraries and frameworks that empower efficient experimentation and application of these potent algorithms grows 
  concomitantly.

  The \textit{Kotlin} programming language emerges as a promising platform for developing such a 
  framework~\autocite{KotlinProgrammingLanguagea}. \textit{Kotlin} distinguishes itself with its lucid syntax, static 
  typing, and flawless interoperability with Java, making it an attractive choice for AI and EC 
  enthusiasts~\autocite{CallingJavaKotlin,CallingKotlinJava,KotlinJavaInteropGuide}. The significance of 
  \textit{Kotlin} has been formally acknowledged in \textit{Android} app 
  development~\autocite{AndroidKotlinfirstApproach}, and it continues to gain popularity among developers, with over 
  60\% of \textit{Android} developers now using it~\autocite{LatestProgrammingLanguages2023}. Furthermore, according to 
  a \textit{GitHub} study, \textit{Kotlin} emerged as the ninth fastest-growing language worldwide from 2021 to 2023, 
  overtaking \textit{Python} by 0.4\%~\autocite{TopProgrammingLanguages}.

  The allure of \textit{Kotlin} is further amplified by its multiplatform feature, which permits code sharing across a 
  wide range of platforms, including JVM, \textit{JavaScript}, \textit{Android}, \textit{iOS}, and native desktop 
  applications. This cross-platform compatibility promotes enhanced productivity and consistency in software 
  development.

  \textit{Kotlin} offers support for functional programming (FP) and coroutines, significantly easing the development 
  of asynchronous and concurrent applications. This feature is particularly advantageous in the context of evolutionary 
  algorithms, which frequently require the parallel execution of multiple processes. While Kotlin currently lacks 
  certain FP features like pattern matching, plans are in place to introduce these enhancements in future versions, 
  further bolstering its capabilities for evolutionary computation. Moreover, \textit{Kotlin}'s capabilities align well 
  with the design of \textit{Domain Specific Languages} (DSLs)~\autocite{TypesafeBuildersKotlin}, which can be used to 
  concisely and intuitively define the problem domain.

  Given these compelling features, this thesis aims to exploit the prowess of \textit{Kotlin} to architect and 
  implement a pioneering genetic algorithms framework, thereby augmenting the ever-growing application of
  \textit{Kotlin} within the AI community. We expect this framework to stimulate research and applications in the 
  field, hastening the creation of novel algorithms and techniques.
