\section{Thesis Structure}
\label{sec:structure}

  \begin{enumerate}
    \item \textbf{Background (\vref{chap:bg}):} This chapter offers a necessary 
      foundation for understanding the central concepts utilized in this thesis.
      It encompasses a theoretical analysis of the relevant algorithms, setting 
      the stage for the discussions and investigations that follow.
    \item \textbf{State of the Art (\vref{chap:sota}):} Here, we present a 
      comprehensive review of the current advancements in the domain of 
      evolutionary computation.
      We particularly emphasize and examine the prevailing frameworks in this 
      field, facilitating a comparative context for our research.
    \item \textbf{Framework Design and Implementation (\vref{chap:keen}):} This 
      chapter elucidates the architecture of our framework, detailing the
      design considerations and the subsequent implementation.
    \item \textbf{Function Optimization Case Study (\vref{chap:fn_opt}):} In 
      this section, we put our framework to the test by applying it to a 
      real-world function optimization problem.
      We utilize 20 classical benchmark functions to assess the framework's 
      performance, with the functions elaborately displayed in 
      \vref{app:test_functions}.
    \item \textbf{Knapsack Problem Case Study 
      (\vref{chap:case_study_knapsack}):} This chapter features another 
      practical application of the framework, in this case, tackling the 
      Knapsack Problem.
      We test the framework against two different variants of this problem: the 
      0-1 Knapsack Problem and the Unbounded Knapsack Problem, evaluating its 
      adaptability and robustness.
    \item \textbf{Crash Reproduction Problem Case Study (\vref{chap:beacon}):} 
      In our third case study, we explore the application of the framework to a 
      crash reproduction problem, showcasing its utility in a distinctly 
      different domain.
    \item \textbf{Conclusion and Future Work (\vref{chap:conclusions}):} In the 
      concluding chapter, we summarize the breadth of work accomplished in this 
      thesis and propose potential avenues for future investigations and 
      enhancements based on our findings.
  \end{enumerate}

  This structure facilitates a progressive narrative of our research, allowing 
  for a coherent understanding of our objectives, methodology, and outcomes.
