\section{Summary and Contributions}
\label{sec:conclusions:summary}
    This thesis explores the domain of Evolutionary Computation (EC), a subset of Artificial Intelligence (AI) that 
    utilizes natural selection principles to address complex challenges. EC encompasses a diverse range of algorithms, 
    each distinct in capabilities and applications. The primary focus of this work is the application of EC in the 
    realms of software engineering and scientific computing.

    A significant contribution of this thesis is the development of a Kotlin-based framework for EC algorithms. The 
    framework stands out for its modular and scalable architecture, facilitating the incorporation of novel algorithms 
    and integration with existing ones. This design approach ensures adaptability and long-term relevance in the 
    evolving field of EC.

    An extensive review of existing EC frameworks was undertaken to inform the development of this novel framework. This 
    review involved a comprehensive analysis of leading frameworks, identifying their key characteristics and 
    functionalities. These insights were instrumental in shaping the design and development of the Kotlin-based EC 
    framework.

    The efficacy of the framework is demonstrated through two distinct case studies. The first involves applying a 
    Genetic Algorithm (GA) to solve classic numerical optimization challenges, showcasing the framework's ability to 
    tackle complex problems. The second case study implements a simplified Linear Genetic Programming (LGP) algorithm to 
    replicate software crashes. While the results were mixed, they underscore the framework's potential in facilitating 
    more intricate EC algorithms.

    The framework presented here is an evolving entity, with ongoing enhancements anticipated. As an open-source 
    project, it encourages collaboration and continuous improvement within the community. The framework is accessible at 
    \url{www.github.com/r8vnhill/keen}, inviting contributions and further exploration in the field of EC.

    A cornerstone of this thesis is the in-depth exploration of the theoretical aspects of Evolutionary Computation 
    (EC). It delves into the foundational concepts, methodologies, and principles of EC, providing a rich, academic 
    context that informs the practical applications developed. This comprehensive theoretical base not only enhances the 
    understanding of EC but also serves as a valuable reference for future research in the field.
    
    An innovative outcome of this research is the creation of \textit{StraitJakt}, a data validation tool developed 
    using Kotlin. This tool emerged as a natural extension of the need for robust data validation within the EC 
    framework. \textit{StraitJakt} excels in validating the integrity of data inputs and outputs, offering a highly 
    expressive and user-friendly interface. Its development signifies a notable contribution to the realm of data 
    validation, particularly in applications involving complex computational algorithms. Available as an open-source 
    resource, \textit{StraitJakt} can be accessed at \url{www.github.com/r8vnhill/jakt}, inviting developers and 
    researchers to utilize and further enhance its capabilities.
