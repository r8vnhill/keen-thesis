\section{Summary and Contributions}
\label{sec:conclusions:summary}
    This thesis explores the domain of Evolutionary Computation (EC), a subset of Artificial Intelligence (AI) that 
    utilizes natural selection principles to address complex challenges. EC encompasses a diverse range of algorithms, 
    each distinct in capabilities and applications. The primary focus of this work is the application of EC in the 
    realms of software engineering and scientific computing.

    A significant contribution of this thesis is the development of a framework for EC algorithms. The framework stands 
    out for its modular and scalable architecture, facilitating the incorporation of novel algorithms and integration
    with existing ones.

    The development of the new EC framework was guided by a detailed review of existing frameworks in the field. This 
    review analyzed the core features and functionalities of leading EC frameworks, providing valuable insights that 
    influenced the framework's design. Notable design inspirations include the use of factory patterns from
    \textit{Jenetics}~\autocite{wilhelmstotterJeneticsJavaGenetica}, the evolution engine concept adapted from 
    \textit{Jenetics} and Bergel's work~\autocite{bergelAgileArtificialIntelligence2020}, a domain-specific language 
    (DSL) for algorithm configuration inspired by \textit{ECJ}~\autocite{ECJ}, and the introduction of evolution 
    listeners to address the absence of such features in current frameworks. The aim was to address the limitations 
    observed in existing frameworks, particularly their lack of flexibility and adaptability for complex problem 
    domains, by proposing a more modular and extensible architecture. Key influences on the framework's development 
    were \textit{Jenetics}, \textit{DEAP}~\autocite{DEAPDocumentationDEAP}, and 
    \textit{GeneticSharp}~\autocite{giacomelliGeneticSharp2023}, which contributed significantly to its design choices.

    The efficacy of the framework is demonstrated through two distinct case studies. The first involves applying a 
    Genetic Algorithm (GA) to solve classic numerical optimization challenges, showcasing the framework's ability to 
    tackle complex problems. The second case study implements a simplified Linear Genetic Programming (LGP) algorithm to 
    replicate software crashes. While the results were mixed, they underscore the framework's potential in facilitating 
    more intricate EC algorithms.

    The framework presented here is an evolving entity, with ongoing enhancements anticipated. As an open-source 
    project, it encourages collaboration and continuous improvement within the community. The framework is accessible at 
    \url{www.github.com/r8vnhill/keen}, inviting contributions and further exploration in the field of EC.

    A cornerstone of this thesis is the in-depth exploration of the theoretical aspects of Evolutionary Computation 
    (EC). It delves into the foundational concepts, methodologies, and principles of EC, providing a rich, academic 
    context that informs the practical applications developed. This comprehensive theoretical base not only enhances the 
    understanding of EC but also serves as a reference for future research in the field.
    
    Regarding the first hypothesis, the findings illustrate the framework's modular design facilitates seamless expansion and incorporation of novel algorithms and genetic operators. This is exemplified by the integration of Linear Genetic Programming (LGP) for crash reproduction purposes.

    The second hypothesis finds support through the development of a \enquote{generalized crossover algorithm.} This algorithm accounts for varying numbers of parents and offspring, demonstrating its effectiveness in the function optimization case study by efficiently locating optimal solutions. The crash reproduction case, employing a more traditional crossover approach, further validates the algorithm's adaptability to different requirements. Future investigations could explore scenarios with parent counts diverging from the standard two.

    The introduction of monitoring tools within the framework stands out as a notable advancement, offering a clear visualization of population evolution which significantly aids algorithm analysis and debugging, as evidenced in the case study result sections.

    An innovative outcome of this research is the creation of \textit{StraitJakt}, a data validation 
    tool developed using Kotlin. This tool was developed as a response to the need for reliable data 
    validation within the EC framework, addressing hypothesis 2. \textit{StraitJakt} aims to ensure 
    the integrity of data inputs and outputs, providing a user-friendly interface. Its development 
    signifies a contribution to the realm of data validation, 
    particularly in applications involving complex computational algorithms. Available as an 
    open-source resource, 
    \textit{StraitJakt} can be accessed at \url{www.github.com/r8vnhill/strait-jakt}, inviting 
    developers and researchers to utilize and further enhance its capabilities.
