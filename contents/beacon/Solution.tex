\section{Solution}
\label{sec:solution}

    Our objective is to design a genetic algorithm that creates a program \(\mathsf{P'}\), which, when executed, 
    throws an exception \(E'\) with a stack trace \(S'\) closely resembling the stack trace \(S\) of a given 
    program \(\mathsf{P}\) throwing exception \(E\). To achieve this, we utilize the flexibility of the \textit{Keen} 
    framework, integrating dynamic manipulation of Kotlin code through reflection. This approach allows our LGP model to
    interact directly with Kotlin code.

    \subsection{Genetic Representation}
        In our genetic algorithm, a program is conceptualized as a list of functions. Correspondingly, we define 
        a function as a gene and a program as a chromosome. To adapt the \textit{Keen} framework to this novel genetic 
        structure, we extend its capabilities by implementing the \texttt{Gene} and \texttt{Chromosome} interfaces.

        The following Kotlin code illustrates the custom gene implementation, representing a function in our genetic model:
        \begin{code}{
            Caption: Implementation of a custom gene to represent a Kotlin function.
        }{label=lst:beacon:gene}{kotlin}
            class KFunctionGene(override val value: KFunction<*>, val functions: List<KFunction<*>>) :
                Gene<KFunction<*>, KFunctionGene> {
                val arity = value.parameters.size
                override fun duplicateWithValue(value: KFunction<*>) = KFunctionGene(value, functions)
                override fun generator() = functions.random(Domain.random)
                operator fun invoke(args: List<*>) = value.callBy(value.parameters.zip(args).toMap())
            }
        \end{code}

        The following code demonstrates the custom chromosome implementation, which represents a Kotlin program:
        \begin{code}{
            Caption: Implementation of a custom chromosome to represent a Kotlin program.
        }{label=lst:beacon:chromosome}{kotlin}
            class KFunctionChromosome(override val genes: List<KFunctionGene>) 
                : Chromosome<KFunction<*>, KFunctionGene> {
                override fun duplicateWithGenes(genes: List<KFunctionGene>) = 
                    KFunctionChromosome(genes)
                class Factory(override var size: Int, val geneFactory: () -> KFunctionGene) :
                    Chromosome.AbstractFactory<KFunction<*>, KFunctionGene>() {
                    override fun make() = KFunctionChromosome((0..<size).map { geneFactory() })
                }
            }
        \end{code}

        These straightforward representations lay the groundwork for solving intricate problems. One of \textit{Keen}'s 
        key strengths is its user-centric design, enabling easy customization of genetic material to address various 
        computational challenges.


    \subsection{Fitness Function}
        The fitness function \(\phi\), pivotal to our genetic algorithm, is formulated to evaluate how closely the 
        generated program \(\mathsf{P'}\) replicates the exception characteristics and stack trace of a given program 
        \(\mathsf{P}\). Let \(E\) denote the exception thrown by \(\mathsf{P}\) with stack trace \(S\). The fitness 
        function is defined as:
    
        \begin{equation}
            \phi(E, S) = 2 \cdot \delta_C(E) + \delta_M(E) + 2 \cdot \Delta_F(S)
        \end{equation}
    
        where:
        \begin{itemize}
            \item \(\delta_C(E)\) represents the \textit{class equality} between the exception \(E\) and the exception 
                thrown by \(\mathsf{P'}\). It is assigned a value of 1 if both exceptions are of the same class, and 0 
                otherwise.
            \item \(\delta_M(E)\) denotes the \textit{message equality} between the exception \(E\) and the exception 
                thrown by \(\mathsf{P'}\). It evaluates to 1 if both exceptions share identical messages, and 0 
                otherwise.
            \item \(\Delta_F(S)\) signifies the \textit{function presence} in the stack trace comparison between \(S\) 
                and the stack trace of \(\mathsf{P'}\). It takes a value between 0 and 1, representing the proportion of
                functions in \(S\) present in the stack trace of \(\mathsf{P'}\).
        \end{itemize}
    
        This fitness function aims to maximize the congruence between the stack traces and exceptions of the original and 
        the generated programs. The first two components, \(\delta_C(E)\) and \(\delta_M(E)\), focus on the fidelity of 
        the exception type and message in \(\mathsf{P'}\) relative to \(\mathsf{P}\). The third component, \(\Delta_F(S)\), 
        assesses the stack trace's accuracy in reflecting the original program's execution path. The assigned weights to 
        these components mirror their respective importance in achieving a faithful reproduction. The chosen values are 
        adaptations from Bergel \& Slater's methodology~\autocite{bergelBeaconAutomatedTest2021}, which in turn are
        inspired by the work of Soltani et al.~\autocite{soltaniGuidedGeneticAlgorithm2017}.
    
        The optimal score of the fitness function is 5, achievable when \(\mathsf{P'}\) throws an exception mirroring 
        the class and message of \(E\) and its stack trace \(S'\) includes all functions in \(S\) from the original 
        program.
    
        The fitness function is then implemented as in the following pseudocode:

        \begin{code}{%caption
            Implementation of the fitness function.
        }{label=lst:beacon:fitness}{kotlin}
            fun fitness(genotype) = genotype.let { functions ->
                try {
                    variables = []
                    functions.forEach { f ->
                        variables += f(variables.takeLast(f.arity))
                    }
                    0.0
                } catch (E) {
                    deltaC = classOf(E) == classOf(exception)
                    deltaM = E.message == exception.message
                    DeltaF = E.stackTrace.count { it.function in functions } / E.stackTrace.size
                    2 * deltaC + deltaM + 2 * DeltaF
                }
            }
        \end{code}

        This fitness function implementation works as follows: 
        \begin{itemize}
            \item It iteratively executes each function in the list (genotype), passing the last \enquote{arity} number 
                of arguments from a dynamically maintained list \texttt{variables}. 
            \item If no exception is thrown during execution, the function returns a fitness score of 0.0.
            \item In case of an exception, the function calculates three components: class equality (\(\delta_C\)), 
                message equality (\(\delta_M\)), and function presence (\(\Delta_F\)), which contribute to the fitness 
                score.
            \item The final score is computed as \(2 \cdot \delta_C + \delta_M + 2 \cdot \Delta_F\), encapsulating the 
                similarity of the thrown exception and its stack trace to those of the target exception.
        \end{itemize}
        
        Additionally, each defined function will return either an integer or \textit{nothing} to avoid type mismatches
        during execution. Note that, given our approach, since the fitness function will catch any exception thrown by
        the program, the program will always terminate gracefully, regardless of the exception type, meaning that even
        if type mismatches occur, the program will still calculate a fitness score correctly.

        This approach, while simplified, is expected to generate sufficiently challenging test cases to assess the 
        genetic algorithm's effectiveness in crash reproduction scenarios.

    \subsection{Genetic Operators}
        The genetic algorithm for our crash reproduction problem employs a set of specific genetic operators to navigate 
        the search space effectively. These operators are crucial for guiding the algorithm towards optimal solutions.
    
        \textbf{Selection Operator:} We use a \textit{Tournament Selector} with a tournament size of 3. This selector 
        picks three individuals randomly and selects the best among them, based on their fitness scores.
    
        \textbf{Crossover Operators:} To test the effectiveness of different crossover strategies, we examine two
        distinct crossover operators:

        \begin{itemize}
            \item \textbf{Single-Point Crossover:} This operator selects a random point within the chromosome and 
                swaps the genes between the two parents at that point. It is useful for exchanging genetic material 
                between the parents.
            \item \textbf{Uniform Crossover:} This operator iterates through the chromosome and returns a gene from 
                either parent with equal probability. This operator is not natively supported by the \textit{Keen}
                framework. However, we can implement it succinctly using a Combine Crossover like so:
        \end{itemize}

        \begin{code}{%
            Implementation of the uniform crossover operator using the \textit{Keen} framework.
        }{label=lst:beacon:uniform}{kotlin}
            val crossover = CombineCrossover({ it.random(Domain.random) })
        \end{code}
    
        \textbf{Mutation Operators:} To investigate the effectiveness of different mutation strategies, we examine three
        distinct mutation operators:
    
        \begin{itemize}
            \item \textbf{Random Mutation:} This operator targets a randomly chosen gene in the chromosome, replacing it
                with a new randomly generated gene. It introduces diversity by altering gene values.
            \item \textbf{Swap Mutation:} This operator selects two genes at random within the chromosome and swaps
                their positions. It is useful for rearranging the existing genetic structure without introducing new genes.
            \item \textbf{Inversion Mutation:} This operator inverts the order of a randomly chosen subset of genes
                within the chromosome. It is effective for changing the gene sequence, which might lead to significant
                variations in the resultant program behavior.
        \end{itemize}
        
        Through the comparison of these genetic operators, we aim to evaluate their respective contributions to the 
        overall effectiveness of the genetic algorithm in reproducing crash scenarios accurately.
    
    \subsection{Program Minimization}
        The genetic algorithm aims to produce a program that closely replicates the target program's exception and stack 
        trace. However, the resulting program may not be the most concise in terms of code length. To refine the output, 
        we introduce a program minimization procedure, designed to shorten the program while maintaining its essential 
        functionality.
    
        This minimization process is executed as a post-processing step after the genetic algorithm concludes. It 
        follows a systematic approach to reduce the program's length:
    
        \begin{enumerate}
            \item \textbf{Iterative Reduction:} The algorithm begins by sequentially removing functions, starting from 
                both the end and the beginning of the program.
            \item \textbf{Fitness Evaluation:} After each removal, the altered program's fitness score is reassessed.
            \item \textbf{Decision Making:} If the fitness score remains consistent with the original score, the 
                function removal is deemed non-impactful, and the reduction is retained. This step ensures that only 
                non-critical functions are eliminated.
            \item \textbf{Termination:} The process continues until any further removals lead to a change in the fitness 
                score, suggesting a functional alteration in the program. At this point, the minimization ceases.
        \end{enumerate}
    
        The resulting program, shorter yet functionally akin to the original output, is then presented as the final 
        solution.
    
        It is important to note that while this method efficiently reduces program length, it does not guarantee the 
        shortest possible solution. The primary goal is to strike a balance between brevity and functionality, yielding 
        programs that are sufficiently concise for effective evaluation in crash reproduction scenarios. The 
        effectiveness of this minimization technique serves as a testament to the genetic algorithm's capability in 
        generating robust and near-optimal solutions.
    
    \subsection{Implementation}
        The genetic algorithm for crash reproduction is encapsulated in a tool we have developed, named 
        \textit{Tracer}. This tool leverages the capabilities of the \textit{Keen} framework to efficiently navigate 
        the problem space. Below is a snippet demonstrating the core implementation of the evolution engine within 
        \textit{Tracer}:
    
        \begin{code}{%
            Caption: Implementation of the genetic algorithm using the \textit{Keen} framework.
        }{label=lst:beacon:algorithm}{kotlin}
            evolutionEngine(
                ::fitnessFunction, // Defines the fitness function
                genotypeOf { // Constructs the genotype
                    chromosomeOf {
                        KFunctionChromosome.Factory(10) { // Sets the chromosome size
                            KFunctionGene(functions.random(Domain.random), functions) // Creates a gene
                        }
                    }
                }
            ) {
                populationSize = 1000 // Specifies the size of the population
                alterers += listOf(mutator, SinglePointCrossover(0.3))
                limits += listOf(TargetFitness(5.0), GenerationLimit(1000))
                listeners += this@Tracer.listeners // Adds event listeners for monitoring the algorithm
            }
        \end{code}
    
        This implementation highlights several key components:

        \begin{itemize}
            \item \textbf{Fitness Function:} The \(\texttt{::fitnessFunction}\) evaluates the proximity of the generated 
                program's behavior to the target.
            \item \textbf{Genotype and Chromosome Structure:} The algorithm creates chromosomes composed of function
                genes, reflecting the potential solution structure.
            \item \textbf{Population Size:} A larger population size of 1000 individuals is set to provide a diverse
                genetic pool.
            \item \textbf{Alterers:} Both mutation and crossover strategies are employed to introduce genetic 
                variations.
            \item \textbf{Termination Criteria:} The algorithm terminates upon achieving a target fitness or reaching
                the maximum number of generations.
            \item \textbf{Listeners:} Event listeners are integrated for tracking the algorithm's progress.
        \end{itemize}
    
        \textit{Tracer} is implemented in a way that allows for easy adaptability to different optimization strategies,
        since it expects an engine that could be configured with different parameters, genetic operators, or even
        different algorithms that could be implemented as another type of engine.\footnote{
            In \textit{Keen} the concept of \textit{evolver} is used to represent an arbitrary evolutionary algorithm 
            that could be used to solve a problem. The \textit{evolutionEngine} function is a specific type of evolver 
            that uses a genetic algorithm to solve a problem.
        } Moreover, the tool could be modified to use another --non-evolutionary-- algorithm, such as a random search
        algorithm, by replacing the engine with another such algorithm.

        The implementation of the \texttt{run} method demonstrates the aforementioned ease of adaptability, as 
        implementing an alternative --non-evolutionary-- algorithm would only require replacing the assignment of the
        \texttt{result} variable with the result of the alternative algorithm. This is shown in the following 
        pseudo-code:

        \begin{code}{%
            Implementation of the \texttt{run} method in the \textit{Tracer} tool.
        }{label=lst:beacon:run}{kotlin}
            fun run(): MinimalCrashReproduction {
                val result = engine.evolve() // ^This line could be replaced with the result of another algorithm
                val program = minimize(result)  // Minimizes the program
                    .population
                    .groupBy { individual -> individual.fitness }
                    .maxBy { group -> group.key }.value
                    .minBy { genotype -> |genotype.flatten()| }
                return MinimalCrashReproduction(program)
            }
        \end{code}

        The complete implementation of the \textit{Tracer} tool, encompassing these elements, can be found in the 
        accompanying code listing referenced at \vref{lst:app:beacon:main}.
    
        The \textit{Tracer} tool can then be utilized to define \textbf{TC1} and \textbf{TC2} test cases, as
        demonstrated in the following code snippet:

        \begin{code}{%
            Implementation of the \textbf{TC1} and \textbf{TC2} test cases using the \textit{Tracer} tool.
        }{label=lst:beacon:tracer}{kotlin}
            // TC1
            val tracer = Tracer.create<IllegalArgumentException>(
                // Defines the set of functions with 0, 1, and 2 arguments
                functions0 + functions1 + functions2,
                targetMessage = "The number is greater than 100",
                // Defines the genetic operators, for example SwapMutator, SinglePointCrossover, etc.
                mutator = mutator,
                crossover = crossover,
                // Optionally include listeners for monitoring the algorithm
                // listeners = ...
            )
            val result = tracer.run()
            // Do something with the result

            // TC2
            val tracer = Tracer.create<IllegalArgumentException>(
                functions0 + functions1 + functions2,
                targetFunction = listOf("throwIAE"),
                mutator = mutator,
                crossover = crossover
            )
            val result = tracer.run()
            // Do something with the result
        \end{code}