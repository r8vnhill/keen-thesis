\section{Introduction}
\label{sec:beacon:intro}

    The realm of automatic test case generation in software engineering has seen significant advancements, particularly 
    with the integration of Generative Pre-trained Transformer (GPT) models. Tools like \textit{ChatGPT}, 
    \textit{GitHub Copilot}, and \textit{JetBrain's AI Assistant} exemplify the progress in this field. A notable 
    development in this area is \textit{ChatUniTest} by Xie et al.~\autocite{xieChatUniTestChatGPTbasedAutomated2023}, 
    a GPT-based tool for automated Java test case generation. It has shown promise when compared to established 
    tools like \textit{Evosuite}~\autocite{fraserEvoSuiteAutomaticTest2011} and 
    \textit{Randoop}~\autocite{pachecoRandoopFeedbackdirectedRandom2007}. Despite these advancements, concerns about 
    the environmental impact of large-scale language models, such as significant water and electricity consumption, 
    cannot be overlooked~\autocite{georgeEnvironmentalImpactAI2023, UWResearcherDiscusses}.

    Parallel to general test case generation, crash reproduction\footnote{
        The following section provides a formal definition of crash reproduction.
    }--a niche yet critical area in software engineering—focuses on generating test cases to replicate specific crashes 
    for debugging purposes. This task involves navigating extensive search spaces and understanding intricate codebase 
    semantics. Bergel \& Slater's \textit{Beacon}~\autocite{bergelBeaconAutomatedTest2021}, an automated tool for 
    Python, leverages a genetic algorithm to create Minimal Crash Reproduction (MCR) test cases. 
    Their work demonstrates the efficacy of this approach across various exceptions and program types. However, a 
    detailed performance analysis and comparison with other state-of-the-art tools are absent in their study.

    A promising avenue within this domain is Linear Genetic Programming (LGP), characterized by representing 
    individuals as linear instruction sequences. The output of the final instruction in this sequence defines the 
    individual's output. Building upon the concept introduced by Bergel \& Slater, this chapter presents a novel 
    LGP-based approach for generating MCR test cases in Kotlin. We aim to conduct a performance analysis of our 
    methodology. Due to the nascent stage of Kotlin-specific crash reproduction tools, direct performance comparisons 
    with other established tools will not be pursued. Instead, our focus will be on assessing our approach's 
    effectiveness in replicating crashes induced by diverse exceptions in various program types. An analysis of the
    implementation details, in particular, the capability of the framework to easily define complex structures to 
    represent candidate solutions and adaptability to new algorithms,\footnote{
        \enquote{New} in the sense of algorithms that are not yet implemented in the framework, in this case the LGP
        algorithm.
    } will also be conducted. Finally, we will discuss the limitations of our approach and outline potential avenues for
    future work.
    