\section{Problem Description}
\label{sec:beacon:problem}

    The primary goal of this study is to demonstrate a complex application of the \textit{Keen} framework, focusing on 
    the challenging domain of crash reproduction in Kotlin programs. Specifically, our objective is to replicate stack 
    traces accurately while ensuring that the resultant Kotlin programs are not only syntactically correct but also 
    concise, ideally minimized in terms of code length.

    This task presents itself as a maximization problem, where the aim is to maximize the similarity between the 
    original program's stack trace and that of the generated program. The inherent complexity lies in producing a Kotlin
    program that not only triggers a stack trace akin to the original but does so with the minimal number of code lines.

    Given the vastness of the search space, encompassing all conceivable Kotlin programs, practical constraints are 
    necessary to render this problem solvable. To this end, we confine our search to Kotlin programs with a maximum 
    length of 10 lines, comprising a predefined set of functions capable of handling up to three arguments.

    It is important to note, however, that even this \enquote{constrained} search space remains infinitely large. This 
    infinity is evident when considering a function with a single real number argument (\(\mathbb{R}\)), representing 
    an uncountably infinite set. 

    Similar to our previous case studies, the effectiveness of the \textit{Keen} framework in navigating this 
    immensely large search space will be evaluated. This will be accomplished by comparing the results achieved 
    through the framework against those obtained from a random search algorithm, highlighting \textit{Keen}'s 
    capabilities in efficiently addressing complex problems in the realm of crash reproduction.

    Specifically, we will try to reproduce two test cases:

    \begin{enumerate}
        \item \textbf{TC1}: A program that throws an \texttt{IllegalArgumentException} with the message \enquote{The 
            number is greater than 100}; and
        \item \textbf{TC2}: A program that throws an \texttt{IllegalArgumentException} that has been thrown by a 
            function with the name \texttt{throwIAE}.
    \end{enumerate}
    
    The concrete functions used in this study are presented in \vref{lst:app:beacon:fun}.
