\section{Results and Discussion}
\label{sec:results}
    This section presents an evaluation of our crash reproduction approach, focusing on its performance across various 
    genetic operators. We conducted experiments using three mutation operators—inversion, random, and swap—and two 
    crossover operators, namely single-point and uniform. To illustrate the fitness evolution achieved by each 
    combination of these genetic operators, we refer to \vref{fig:beacon:1} and \vref{fig:beacon:2}. Our 
    analysis covers two test cases: \texttt{TC1} and \texttt{TC2}. The performance results for \texttt{TC1} are detailed 
    in \vref{fig:beacon:1}, while those for \texttt{TC2} are depicted in \vref{fig:beacon:2}. 
    These results provide insights into the efficacy of different genetic 
    operator combinations in the context of crash reproduction.

    \begin{figure}[ht!]
        \centering
        \begin{subfigure}{0.45\textwidth}
            \includegraphics[width=\textwidth]{img/beacon_sp_inv_1.png}
            \caption{Inversion Mutation with a Single Point Crossover}
            \label{fig:beacon:1:inversion}
        \end{subfigure}
        \hfill
        \begin{subfigure}{0.45\textwidth}
            \includegraphics[width=\textwidth]{img/beacon_sp_random_1.png}
            \caption{Random Mutation with a Single Point Crossover}
            \label{fig:beacon:1:random}
        \end{subfigure}
        \begin{subfigure}{0.45\textwidth}
            \includegraphics[width=\textwidth]{img/beacon_sp_swap_1.png}
            \caption{Swap Mutation with a Single Point Crossover}
            \label{fig:beacon:1:swap}
        \end{subfigure}
        \hfill
        \begin{subfigure}{0.45\textwidth}
            \includegraphics[width=\textwidth]{img/beacon_uniform_inv_1.png}
            \caption{Inversion Mutation with a Uniform Crossover}
            \label{fig:beacon:3:inversion}
        \end{subfigure}
        \begin{subfigure}{0.45\textwidth}
            \includegraphics[width=\textwidth]{img/beacon_uniform_random_1.png}
            \caption{Random Mutation with a Uniform Crossover}
            \label{fig:beacon:3:random}
        \end{subfigure}
        \hfill
        \begin{subfigure}{0.45\textwidth}
            \includegraphics[width=\textwidth]{img/beacon_uniform_swap_1.png}
            \caption{Swap Mutation with a Uniform Crossover}
            \label{fig:beacon:3:swap}
        \end{subfigure}
        \caption{Comparison of the three genetic operators on the \texttt{TC1} test case.}
        \label{fig:beacon:1}
    \end{figure}
    
    \begin{figure}[ht!]
        \centering
        \begin{subfigure}{0.45\textwidth}
            \includegraphics[width=\textwidth]{img/beacon_sp_inv_2.png}
            \caption{Inversion Mutation with a Single Point Crossover}
            \label{fig:beacon:2:inversion}
        \end{subfigure}
        \hfill
        \begin{subfigure}{0.45\textwidth}
            \includegraphics[width=\textwidth]{img/beacon_sp_random_2.png}
            \caption{Random Mutation with a Single Point Crossover}
            \label{fig:beacon:2:random}
        \end{subfigure}
        \begin{subfigure}{0.45\textwidth}
            \includegraphics[width=\textwidth]{img/beacon_sp_swap_2.png}
            \caption{Swap Mutation with a Single Point Crossover}
            \label{fig:beacon:2:swap}
        \end{subfigure}
        \hfill
        \begin{subfigure}{0.45\textwidth}
            \includegraphics[width=\textwidth]{img/beacon_uniform_inv_2.png}
            \caption{Inversion Mutation with a Uniform Crossover}
            \label{fig:beacon:4:inversion}
        \end{subfigure}
        \begin{subfigure}{0.45\textwidth}
            \includegraphics[width=\textwidth]{img/beacon_uniform_random_2.png}
            \caption{Random Mutation with a Uniform Crossover}
            \label{fig:beacon:4:random}
        \end{subfigure}
        \hfill
        \begin{subfigure}{0.45\textwidth}
            \includegraphics[width=\textwidth]{img/beacon_uniform_swap_2.png}
            \caption{Swap Mutation with a Uniform Crossover}
            \label{fig:beacon:4:swap}
        \end{subfigure}
        \caption{Comparison of the three genetic operators on the \texttt{TC2} test case.}
        \label{fig:beacon:2}
    \end{figure}

    The experimental results demonstrate varied performances among different genetic operators in the context of crash 
    reproduction. Specifically, for test case \texttt{TC1}, the inversion and swap mutation operators paired with 
    single-point crossover achieved the target fitness value of 5.0, whereas the random mutation operator did not. In 
    contrast, when utilizing uniform crossover, none of the operator combinations attained the fitness goal of 5.0. For 
    \texttt{TC2}, both inversion and random mutation operators with single-point crossover reached the target fitness, 
    while the swap mutation operator fell short. However, using uniform crossover, the inversion and swap mutation 
    operators were successful in achieving the fitness target, unlike the random mutation operator.

    These findings imply that the effectiveness of genetic operators significantly depends on the specific test case. In 
    \texttt{TC1}, inversion and swap mutation operators outperform the random mutation operator, while in \texttt{TC2}, 
    inversion and random mutation operators show superior performance compared to the swap mutation operator. 
    Additionally, the uniform crossover operator generally underperforms relative to the single-point crossover in crash 
    reproduction tasks.

    There was no noticeable trend in average fitness improvement over time, suggesting that the fitness function might 
    not be optimally aligned with the crash reproduction problem. This lack of convergence on a solution could be due to 
    the fitness function's inability to effectively differentiate solutions nearing the target fitness value.

    The outcomes might also indicate limitations in the simplified LGP (Linear Genetic Programming) implementation used 
    for this project. A more advanced LGP implementation or a Multi-Objective Genetic Programming (MOGP) approach might 
    yield better results. MOGP could optimize both the target fitness value and the solution's instruction count, 
    potentially leading to solutions that are both close to the desired fitness value and have fewer instructions, thus 
    more likely representing practical crash reproduction solutions.

    Given these insights, we conclude that a statistical analysis of the results is not necessary, as the current 
    findings do not provide a conclusive basis for such an analysis.

    A review of the implementation showcases how the LGP module employs the same evolutionary engine introduced 
    in the previous chapter, highlighting the framework's adaptability for different computational problems. This 
    unified approach simplifies the development process, allowing users to focus on problem-specific configurations 
    without the need to construct a new engine for GP tasks. The framework's design facilitates easy customization of 
    engine parameters such as genetic operators and fitness functions, ensuring solutions can be finely tuned to meet 
    particular requirements. Moreover, the inclusion of visual tools aids in the evaluation of genetic operators, 
    enhancing the analytical process. Moreover, introducing new genetic materials for complex solutions is made 
    straightforward by implementing the necessary interfaces (\texttt{Chromosome} and \texttt{Gene}), with the engine 
    handling the rest. This flexibility and user-friendly configuration exemplify the framework's capability to support 
    efficient development and application of evolutionary algorithms.

    The \textit{Tracer} tool, in particular, offers an extensive suite of functionalities for MCR problems, adaptable 
    to various optimization strategies and representations. Its flexibility and ease of use render it a valuable 
    asset for crash reproduction research. It's noteworthy that the implementations for test cases \textbf{TC1} and 
    \textbf{TC2} remain consistent, differing only in the parameters provided to \textit{Tracer}, further emphasizing 
    the framework's versatility.

    The implementation of the crash reproduction solution in other frameworks was not conducted due to its complexity. However, Bergel \& Slater's prior work~\autocite{bergelBeaconAutomatedTest2021} in \textit{Python} illustrates the potential for transferring the \textit{Tracer} tool to different programming environments. This suggests that adapting \textit{Tracer} to frameworks like \textit{DEAP} or \textit{Jenetics} could be straightforward, benefiting from the framework's modular structure and the simplicity of its genetic operator implementations. While the frameworks reviewed in this study don't inherently support LGP, they do offer capabilities for tree-based GP, grammatical evolution, and MOGP. These features could serve as a basis for extending the current work into other frameworks, marking an opportunity for further exploration and enhancement in the field.