\section{Genetic Algorithms}
\label{sec:genetic_algorithms}
  As we delve deeper into the intricate workings of the framework, it becomes 
  imperative to explore its foundational components.
  While the theoretical underpinnings of Genetic Algorithms (GAs) have been 
  exhaustively discussed in \vref{sec:bg:ga}, this section aims to shed light 
  on the actual implementation and design specifics of the GAs within our 
  framework.
  Our exploration will be segmented into three primary components: the 
  representation and structuring of the \emph{genetic material}, the core 
  mechanics of the \emph{evolution engine}, and the crucial role of the \emph{evolution listeners}.
  Each subsection will provide a granular understanding of how these elements 
  function and interconnect, ensuring that the GAs are optimally tailored to 
  meet the design goals of the framework.
  
  \subsection{Genetic Material}
  \label{sec:keen:ga:material}
    The genetic material is the core of the GAs, as it is the primary
    representation of the problem domain.
    As such, it is imperative that the genetic material is structured in a
    manner that is both efficient and flexible.

    To achieve this, we designed the genetic material to be 
    \enquote{gene-centric}.
    This means that the genetic material is represented directly or indirectly
    as a collection of genes.
    This design decision was made to provide a high degree of flexibility in
    the design of the APIs, e.g., we could use polymorphic behavior to
    represent the genetic material as a single gene, as a collection of genes 
    (chromosome), or even as a collection of collections of genes (genotype).

    \paragraph{Gene}
      The gene is the most basic unit of genetic material.
      It is the smallest unit of genetic material that can be manipulated by
      the framework.
      
      Each gene should be able to:
      \begin{enumerate*}
        \item \textit{mutate},
        \item \textit{reproduce}, and
        \item \textit{store} information.
      \end{enumerate*}
      In an effort to maintain a high degree of extensibility, we designed each
      gene to perform the least amount of work possible, delegating the
      majority of the work to the operators (refer to \vref{sec:keen:operators}).
      This design decision was made to ensure that the framework is able to
      support a wide variety of genetic material, e.g., binary, integer, real,
      and even custom genetic material.

      Given that we propose the following interface that each gene should
      implement:

      \begin{minted}{kotlin}
        interface Gene<DNA, G: Gene<DNA, G>> : GeneticMaterial<DNA, G>, 
                                               SelfReferential<G> {
            val dna: DNA
            fun mutate(): G = withDna(generator())
            fun generator(): DNA = dna
            fun withDna(dna: DNA): G
            override fun flatten() = listOf(dna)
        }
      \end{minted}

  \subsection{Evolution Engine}
  \label{sec:keen:ga:engine}
    \Blindtext
  \subsection{Evolution Listeners}
  \label{sec:keen:ga:listeners}
    \Blindtext
