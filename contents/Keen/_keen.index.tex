% Copyright (c) 2023 Ignacio Slater Muñoz All rights reserved.
% Use of this source code is governed by a BSD-style
% license that can be found in the LICENSE file.

\chapter{The Keen Framework}
\label{chap:keen}
    \subimport{./}{Introduction.tex}
    \subimport{./}{Architecture.tex}
    \subimport{./ga/}{_ga.index.tex}
    \subimport{./operators/}{_operators.index.tex}
    \subimport{./gp/}{_gp.index.tex}
    % \subimport{./parallelism/}{_parallelism.index.tex}  
    \subimport{./}{Extensibility.tex}

    \section{\textit{StraitJakt}}
        In the field of evolutionary computation, the complexity and non-deterministic nature of these processes 
        underscore the importance of robust data validation, both before and after execution. As frameworks expand in 
        scale, ensuring the integrity of operations becomes crucial. A common issue in genetic programming, for 
        instance, is the inadvertent generation of invalid trees. Such errors may arise from factors like the use of 
        improper operators or the unintended creation of cyclic dependencies.

        One challenge with implementing these necessary validations is the resulting code complexity. Intensive 
        validation checks can obscure the core logic, reducing the code's readability and maintainability. Moreover, in 
        scenarios where multiple validations are required for a single operation, the standard practice of throwing 
        exceptions for each failed check is inefficient. Consider the example of a chromosome with several invalid 
        genes; a more practical approach would be to collate and return a comprehensive list of errors rather than 
        issuing separate exceptions for each anomaly.

        To address these challenges, we developed \textit{StraitJakt}, a specialized library designed to streamline the 
        process of constraint validation within evolutionary algorithms. While \textit{StraitJakt} is tailored for 
        integration with the \textit{Keen} framework, its versatile design allows for application in other projects as 
        well. In \textit{Keen}, \textit{StraitJakt} plays a pivotal role in validating the numerous constraints 
        essential for the proper functioning of the algorithms.

        \textit{StraitJakt} was initially developed as part of the \textit{Keen} framework, but as its complexity grew,
        it was separated into its own project to maintain a clear separation of concerns.

        The library works by defining \textit{constraint blocks}, which are collections of validation checks. Each
        constraint block defines a scope that enables a special syntax for the validation checks within it. The syntax
        is designed to be as close as possible to the natural language used to describe the constraints. The syntax
        defined by the constraint blocks can be seen in \vref{lst:keen:strait-jakt-syntax}.

        \begin{code}{Constraint block syntax}{label={lst:keen:strait-jakt-syntax}}{kotlin}
            constraints {
                "The value of x must be greater than or equal to 10" {
                    // By default, the exception type is defined by the constraint
                    // In this case, BeAtLeast creates an IntConstraintException due to the type of x
                    x must BeAtLeast(10)
                }
                // Optionally, a custom exception constructor can be provided
                "The value of y must not be negative"(::CustomConstraintException) {
                    y mustNot BeNegative
                }
            }
        \end{code}

        Here, the \texttt{constraints} block defines a scope for the validation checks. Then, each constraint is defined
        by a string that describes the constraint and a lambda that contains the validation checks. The validation
        checks are defined using the \texttt{must} and \texttt{mustNot} keywords, which are followed by the constraint
        to be checked.
        
        Constraints are defined by a \texttt{Constraint} object, which is a class that implements the 
        \texttt{Constraint} interface. The \texttt{Constraint} interface defines two functions: \texttt{validator} and
        \texttt{generateException}. The \texttt{validator} function is used to check if the constraint is satisfied,
        and the \texttt{generateException} function is used to generate an exception if the constraint is not
        satisfied.

        Once defined, the block will be evaluated as follows:

        \begin{enumerate}
            \item A \texttt{Jakt.Scope} is created to enable the special syntax for the validation checks.
            \item For each constraint, a \texttt{Jakt.Scope.StringScope} is created to provide access to the
            \texttt{must} and \texttt{mustNot} keywords.
            \item If a validation check fails, an exception is generated and stored in the \texttt{Jakt.Scope}.
            \item Once all the constraints have been evaluated, if:
                \begin{enumerate}
                    \item No exceptions were generated, the code continues normally.
                    \item Exceptions were generated, a \texttt{CompositeException} is thrown containing all the
                        exceptions generated.
                \end{enumerate}
        \end{enumerate}

        \textit{StraitJakt} makes heavy use of operator overloading, extension functions, and trailing lambdas to
        provide a concise syntax for the validation checks. An implementation that does not use these features would
        look like the one shown in \vref{app:lst:strait-jakt-desugared}.
        
        The main purpose of \textit{StraitJakt} is to provide an expressive way of validating multiple constraints at
        once, even though in some cases \textit{Keen} uses it to validate a single constraint. This is done to maintain
        a consistent API for all the constraints, even though some of them only require a single validation check.

        A similar result could be achieved using the \textit{Arrow} library~\autocite{Arrow}, which provides an 
        \texttt{ensure} function that allows us to validate a single constraint. However, \textit{Arrow} introduces 
        non-standard\footnote{
            Non-standard in the context of the \textit{Kotlin} language.
        } functional programming concepts that may be unfamiliar to some users. Given that \textit{Keen} is focused on
        Evolutionary Algorithms, we tried to use a more standard approach to avoid introducing unnecessary complexity. 
        Moreover, we argue that the syntax provided is more expressive than the one provided by \textit{Arrow}\footnote{
            For the case of validating multiple pre-conditions and post-conditions.
        } since the
        scope of each constraint is defined by a string that describes the constraint.

        \begin{code}{
            Example of the use of \textit{Arrow} to validate multiple constraints, note that the code could be improved
            to avoid code duplication and reduce verbosity.
        }{label=lst:app:arrow}{kotlin}
            fun validateX(): Either<IllegalArgumentException, Unit> = either {
                ensure(x >= 10) { IllegalArgumentException("The value of x must be greater than or equal to 10") }
            }

            fun validateY(): Either<CustomConstraintException, Unit> = either {
                ensure(y >= 0) { CustomConstraintException("The value of y must not be negative") }
            }

            val exceptions = mutableListOf<Exception>()

            fold(
                { validateX() },
                { it: IllegalArgumentException -> exceptions.add(it) },
                {}
            )
            fold(
                { validateY() },
                { it: CustomConstraintException -> exceptions.add(it) },
                {}
            )
            if (exceptions.isNotEmpty()) {
                throw CompositeException(exceptions)
            }
        \end{code}

        An example of the use of \textit{StraitJakt} within \textit{Keen} can be seen in \vref{lst:app:strait-jakt}.
        
        As a final note, it is important to mention that \textit{StraitJakt} is heavily inspired by \textit{Kotest}'s
        String Spec DSL and the \texttt{should} and \texttt{shouldNot} keywords~\autocite{KotestKotesta}.
    