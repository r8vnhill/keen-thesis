% Copyright (c) 2023 Ignacio Slater Muñoz All rights reserved.
% Use of this source code is governed by a BSD-style
% license that can be found in the LICENSE file.

\chapter{The Keen Framework}
\label{chap:keen}
  \subimport{./}{Introduction.tex}
  \subimport{./}{Architecture.tex}
  \subimport{./ga/}{_ga.index.tex}
  \subimport{./operators/}{_operators.index.tex}
  \subimport{./gp/}{_gp.index.tex}
  % \subimport{./parallelism/}{_parallelism.index.tex}  
  \subimport{./}{Extensibility.tex}

  \section{Chapter Conclusion}
  \label{sec:keen:conclusion}
    Throughout this chapter, we've journeyed through the conceptual foundations 
    and architectural intricacies of the \textit{Keen} framework. As we've seen, 
    \textit{Keen} is not just another computational tool, but a flexible and 
    extensible platform poised to tackle the myriad challenges associated with 
    genetic algorithms and related computational techniques. Its design 
    philosophy, emphasizing modularity and extensibility, promises to make it a 
    valuable asset for both usage and research.

    But, understanding a framework's theory and architecture is just the 
    beginning. The true power and versatility of \textit{Keen} are best 
    experienced in action. In the subsequent chapters, we will shift our focus 
    from the abstract to the concrete. We'll explore how to harness 
    \textit{Keen}'s capabilities for real-world problems, diving deep into its 
    usage patterns and practical extensions.

    As we move forward, remember the essence of \textit{Keen}: a platform built 
    for evolution, by evolution.