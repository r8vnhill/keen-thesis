% Copyright (c) 2023 Ignacio Slater Muñoz All rights reserved.
% Use of this source code is governed by a BSD-style
% license that can be found in the LICENSE file.

\chapter{The Keen Framework}
\label{chap:keen}
    \subimport{./}{Introduction.tex}
    \subimport{./}{Architecture.tex}
    \subimport{./ga/}{_ga.index.tex}
    \subimport{./operators/}{_operators.index.tex}
    \subimport{./gp/}{_gp.index.tex}
    % \subimport{./parallelism/}{_parallelism.index.tex}  
    \subimport{./}{Extensibility.tex}

    \section{\textit{StraitJakt}}
        In the realm of evolutionary algorithms, the complexity and non-deterministic nature of these processes 
        underscore the importance of robust data validation, both before and after execution. As frameworks expand in 
        scale, ensuring the integrity of operations becomes crucial. A common issue in genetic programming, for 
        instance, is the inadvertent generation of invalid trees. Such errors may arise from factors like the use of 
        improper operators or the unintended creation of cyclic dependencies.

        One challenge with implementing these necessary validations is the resulting code complexity. Intensive 
        validation checks can obscure the core logic, reducing the code's readability and maintainability. Moreover, in 
        scenarios where multiple validations are required for a single operation, the standard practice of throwing 
        exceptions for each failed check is inefficient. Consider the example of a chromosome with several invalid 
        genes; a more practical approach would be to collate and return a comprehensive list of errors rather than 
        issuing separate exceptions for each anomaly.

        To address these challenges, we developed \textit{StraitJakt}, a specialized library designed to streamline the 
        process of constraint validation within evolutionary algorithms. While \textit{StraitJakt} is tailored for 
        integration with the \textit{Keen} framework, its versatile design allows for application in other projects as 
        well. In \textit{Keen}, \textit{StraitJakt} plays a pivotal role in validating the numerous constraints 
        essential for the proper functioning of the algorithms. 

        Although this work does not delve into the intricate details of \textit{StraitJakt}, its utility and impact 
        within the \textit{Keen} framework are notable. For an illustration of the types of validations 
        \textit{StraitJakt} performs, refer to the example provided in \vref{lst:app:strait-jakt}.

    \section{Conclusion}
    \label{sec:keen:conclusion}
        Throughout this chapter, we've journeyed through the conceptual foundations 
        and architectural intricacies of the \textit{Keen} framework. As we've seen, 
        \textit{Keen} is not just another computational tool, but a flexible and 
        extensible platform poised to tackle the myriad challenges associated with 
        genetic algorithms and related computational techniques. Its design 
        philosophy, emphasizing modularity and extensibility, promises to make it a 
        valuable asset for both usage and research.

        But, understanding a framework's theory and architecture is just the 
        beginning. The true power and versatility of \textit{Keen} are best 
        experienced in action. In the subsequent chapters, we will shift our focus 
        from the abstract to the concrete. We'll explore how to harness 
        \textit{Keen}'s capabilities for real-world problems, diving deep into its 
        usage patterns and practical extensions.

        As we move forward, remember the essence of \textit{Keen}: a platform built 
        for evolution, by evolution.