\section{Parallelism}
\label{sec:keen:parallel}
  Given \textit{Keen}'s focus on immutability, there is potential for
  parallelism. Given this, \textit{Keen} provides a number of constructs
  to allow for parallelism. 

  Parallelism is achieved through the use of coroutines. Coroutines are a 
  \textit{Kotlin} feature that enables the development of asynchronous, 
  non-blocking code in a more readable and maintainable, sequential style. 
  Unlike traditional threads which are managed by the operating system and can 
  be resource-intensive to start and switch between, coroutines are lightweight 
  and managed at the application level. They employ "suspending functions" that 
  can pause and resume execution without blocking, thus optimizing resource 
  usage. This allows \textit{Keen} to efficiently tap into the capabilities of 
  multi-core processors, simplifying parallel execution without the typical 
  intricacies of multi-threaded programming.
  
  To allow for an easy transition from sequential to parallel code, 
  \textit{Keen}'s proposes the concept of \textit{executors}. Executors are
  objects that manage the execution of a \enquote{single task} or a
  \enquote{batch of tasks}. This is formalized in the \texttt{KeenExecutor}
  interface, which is shown in \autoref{lst:keen:parallel:executor}. The
  
  \begin{code}{The \texttt{KeenExecutor} interface.}{label={lst:keen:parallel:executor}}{kotlin}
    interface KeenExecutor {
        interface Factory<I, R : KeenExecutor> {
            var creator: (I) -> R
        }
    }
  \end{code}

  The \texttt{KeenExecutor} interface defines a \texttt{Factory} interface
  that is used to create instances of the executor lazily. This is done to
  allow for the executor to be instantiated with the correct parameters at
  the time of execution.

  Currently, \textit{Keen} provides two subtypes of \texttt{KeenExecutor}:
  \texttt{EvaluationExecutor} and \texttt{ConstructorExecutor}. The former is
  used to evaluate the fitness of a population of individuals, while the latter
  is used to construct a population of individuals.
  Additionally, \textit{Keen} provides two implementations for each of these
  subtypes: \texttt{SequentialEvaluator} and \texttt{CoroutineEvaluator} for
  \texttt{EvaluationExecutor}, and \texttt{SequentialConstructor} and
  \texttt{CoroutineConstructor} for \texttt{ConstructorExecutor}. The
  \texttt{SequentialEvaluator} and \texttt{SequentialConstructor} are
  implementations that execute the tasks sequentially, while the
  \texttt{CoroutineEvaluator} and \texttt{CoroutineConstructor} are
  implementations that execute the tasks in parallel using coroutines.

  The following snippet shows the implementations of the 
  \texttt{ConstructorExecutor} interface:

  \begin{code}{The \texttt{ConstructorExecutor} interface.}{
    label={lst:keen:parallel:constructor}
  }{kotlin}
    interface ConstructorExecutor<T> : KeenExecutor {
        operator fun invoke(size: Int, init: () -> T): List<T>

        open class Factory<T> : KeenExecutor.Factory<Unit, ConstructorExecutor<T>> {
            override var creator: (Unit) -> ConstructorExecutor<T> = { SequentialConstructor() }
        }
    }

    class SequentialConstructor<T> : ConstructorExecutor<T> {
        override fun invoke(size: Int, init: () -> T): List<T> {
            return List(size) { init() }
        }
    }
    class CoroutineConstructor<T>(
        val dispatcher: CoroutineDispatcher = Dispatchers.Default
    ) : ConstructorExecutor<T> {
        override fun invoke(size: Int, init: () -> T): List<T> = runBlocking {
            val deferredList = List(size) {
                async(dispatcher) {
                    init()
                }
            }
            deferredList.awaitAll()
        }

        class Factory<T> : ConstructorExecutor.Factory<T>() {
            var dispatcher: CoroutineDispatcher = Dispatchers.Default

            override var creator: (Unit) -> ConstructorExecutor<T> =
                { CoroutineConstructor(dispatcher) }
        }
    }
  \end{code}
