\section{Extensibility}
\label{sec:keen:extensibility}

    \subsection{Structured Extensibility Through Interfaces}
        \textit{Keen} employs well-defined interfaces to encapsulate essential functionalities. These interfaces act as 
        templates, enabling developers to easily extend and modify the framework. The process is straightforward: to 
        enhance or customize capabilities, one simply implements the provided interfaces.

    \subsection{Factory Method Pattern in Genetic Construction}
        The Factory Method Pattern plays a pivotal role in \textit{Keen}, facilitating the dynamic creation of genetic 
        material. By decoupling the framework from specific implementations, this pattern provides the flexibility to 
        introduce varied genetic construction techniques seamlessly.

        Consider the \texttt{Chromosome.Factory} interface and the \texttt{Chromosome.AbstractFactory} abstract class
        that extends it. These provide a standardized mechanism for generating chromosome objects, ensuring
        easy adaptability to diverse and complex problem domains such as crash reproduction (refer to 
        \vref{chap:beacon})

        For illustration, consider the \texttt{SimpleChromosome}:

        \begin{code}{
        Implementation of the \textit{SimpleChromosome} using the 
        \texttt{Chromosome.Factory} interface.
        }{label=lst:keen:factory}{kotlin}
        data class SimpleChromosome(override val genes: List<SimpleGene>) :
            Chromosome<Int, SimpleGene> {
            // ... Other implementations ...

            class Factory(override var size: Int, private val geneFactory: () -> SimpleGene) :
                Chromosome.AbstractFactory<Int, SimpleGene>() {
                override fun make() = SimpleChromosome((0..<size).map { geneFactory() })
            }
        }
        \end{code}

        Using the \texttt{SimpleChromosome} is straightforward:

        \begin{code}{
        Utilization of the \textit{SimpleChromosome} within an evolutionary 
        engine setup.
        }{label=lst:keen:factory:usage}{kotlin}
        data class SimpleGene(val dna: Int) : Gene<Int, SimpleGene> {
            // ... Implementation details for SimpleGene ...
        }
                
        fun main() {
            val engine = evolutionEngine(
                ::fitnessFn,
                genotypeOf {
                    chromosomeOf {
                        // Random genes between 0 and 100
                        SimpleChromosome.Factory(10) { SimpleGene((0..100).random()) }
                    }
                }
            ) {
                // ... Other configurations ...
            }
            val result = engine.evolve()
            // ... Use the result ...
        }
        \end{code}

        We would like to mention that this is not a totally novel approach, as it is inspired by the work of 
        Wilhelmstötter's \textit{Jenetics} \autocite{wilhelmstotterJeneticsJavaGenetica}.

    \subsection{Observer Pattern for Evolution Monitoring}
        A key design pattern employed in \textit{Keen} is the Observer Pattern. This pattern is fundamental for 
        tracking the status of the evolutionary process. Its primary advantage lies in decoupling the core evolutionary 
        engine from the monitoring components. This separation facilitates the extension and customization of 
        monitoring functionalities without affecting the underlying evolutionary mechanisms.

        As discussed in \vref{sec:keen:ga:engine}, the \texttt{EvolutionListener} interface acts as a cornerstone for 
        this pattern. It offers a standardized approach to observe and respond to the evolutionary process. Developers 
        can implement this interface to devise custom listeners, which can then be integrated with the evolutionary 
        engine, enhancing the flexibility and adaptability of the monitoring process.

    \subsection{Modularity in Design}
        The architecture of \textit{Keen} is inherently modular, offering developers
        the flexibility to craft and integrate new algorithms with ease. This
        modularity is exemplified in the way genetic algorithms can be customized.
        Developers can, for instance, design a genetic algorithm that iteratively
        modifies the population until certain criteria are met, utilizing the
        existing functions that define the genetic algorithm's stages.

        This modular approach extends to the enhancement of evolutionary process notifications. For example, developers 
        can design a custom listener with a hook to receive an alert when the population achieves a predefined fitness 
        level. This ability to modify and extend the notification system underscores the framework's versatility in 
        adapting to varied evolutionary scenarios.

        Another, arguably more complex, usage example could be the implementation of Evolutionary Strategies. 
        Evolutionary Strategies are similar to Genetic Algorithms, but they use a different approach to the selection
        process. Without going into the details of the algorithm, the main difference between the two is that in
        Evolutionary Strategies, the operators used to alter the population --selection, mutation and crossover-- act 
        differently on the individuals, and the existence of additional parameters that affect the selection process.
        Notwithstanding these differences, the core of the algorithm is the same, and the implementation of the
        Evolutionary Strategy is straightforward. The following code snippet shows a possible implementation of a 
        simplified evolutionary strategy using Keen:

        \begin{code}{
        Implementation of an Evolutionary Strategy using Keen (assuming the existence of an evolution engine).
        }{label=lst:keen:es}{kotlin}
            var state = EvolutionState.empty()
            do {
                val initialPopulation = engine.startEvolution(state)
                val evaluatedPopulation = engine.evaluate(initialPopulation)
                // We assume this function selects individuals based on the strategy parameters
                val parents = evStratSelection(evaluatedPopulation)
                val nextPopulation = engine.alterOffspring(offspring)
                // We assume this function updates the parameters needed by the evolutionary strategy
                updateStrategyParameters()
                state = state.copy(population = nextPopulation)
            } while (engine.limits.none { it -> it(state) })
        \end{code}

        Note that most of the code is the same as the one used in the genetic algorithm, the only difference is the
        selection process and the update of the parameters. This is possible thanks to the modularity of the framework,
        which allows the developer to reuse the existing functions and classes to create new algorithms.
