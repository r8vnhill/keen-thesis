\section{Extensibility}
\label{sec:keen:extensibility}

  \textit{Keen} is architected with adaptability at its core, catering to a 
  wide array of genetic problems and computational scenarios. This adaptability 
  is principally achieved through the combination of intuitive interfaces and 
  the integration of the Factory Method Pattern.

  \subsection{Structured Extensibility Through Interfaces}
    \textit{Keen} employs well-defined interfaces to encapsulate essential 
    functionalities. These interfaces act as templates, enabling developers to 
    easily extend and modify the framework. The process is straightforward: to 
    enhance or customize capabilities, one simply implements the provided 
    interfaces.

  \subsection{Factory Method Pattern in Genetic Construction}
    The Factory Method Pattern plays a pivotal role in \textit{Keen}, 
    facilitating the dynamic creation of genetic material. By decoupling the 
    framework from specific implementations, this pattern provides the 
    flexibility to introduce varied genetic construction techniques seamlessly.

    Consider the \texttt{Chromosome.Factory} interface and the 
    \texttt{Chromosome.AbstractFactory}. These provide a standardized mechanism 
    for generating chromosome objects, ensuring easy adaptability to diverse 
    problem domains such as the Knapsack problem (refer to 
    \vref{chap:knapsack}).

    For illustration, consider the \texttt{SimpleChromosome}:

    \begin{code}{
      Implementation of the \textit{SimpleChromosome} using the 
      \texttt{Chromosome.Factory} interface.
    }{label=lst:keen:factory}{kotlin}
      class SimpleChromosome(override val genes: List<SimpleGene>) :
          Chromosome<Int, SimpleGene> {
          // ... Other implementations ...

          class Factory(override var size: Int, private val geneFactory: () -> SimpleGene) :
              Chromosome.AbstractFactory<Int, SimpleGene>() {
              override fun make() = SimpleChromosome((0 until size).map { geneFactory() })
          }
      }
    \end{code}

    Using the \texttt{SimpleChromosome} is straightforward:

    \begin{code}{
      Utilization of the \textit{SimpleChromosome} within an evolutionary 
      engine setup.
    }{label=lst:keen:factory:usage}{kotlin}
      data class SimpleGene(val dna: Int) : Gene<Int, SimpleGene> {
          // ... Implementation details for SimpleGene ...
      }
            
      fun main() {
          val engine = engine(
              ::fitnessFn,
              genotype {
                  chromosome {
                      SimpleChromosome.Factory(10) { SimpleGene((0..100).random()) } // Random genes between 0 and 100
                  }
              }
          ) {
              // ... Other configurations ...
          }
          val result = engine.evolve()
          // ... Use the result ...
      }
    \end{code}

  \subsection{Sustaining Future Evolution}
    Designed with a forward-looking perspective, \textit{Keen} leverages 
    interfaces and the Factory Method Pattern to ensure its sustained 
    evolution. This positions \textit{Keen} advantageously, ensuring it remains 
    relevant and adaptable to emerging challenges in genetic algorithms and 
    computations.
