
\section{Genetic Operators}
\label{sec:keen:operators}
  Genetic operators are the driving force behind the evolution process in any
  genetic algorithm or genetic programming paradigm.
  These operators stimulate change in the population and guide the search
  towards optimal solutions.
  This section illuminates the fundamental genetic operators incorporated within
  \textit{Keen}, categorized into \textit{Selection}, \textit{Mutation}, and
  \textit{Crossover}.

  \textit{Selection} operators guide the evolutionary process by determining
  which individuals from the population will contribute to the next generation.
  These mechanisms often favor individuals that exhibit superior fitness, 
  thereby promoting the inheritance of advantageous traits.

  \textit{Mutation} operators introduce randomness into the population by 
  modifying individual genes or gene sequences.
  This mechanism ensures genetic diversity, aiding the population to explore a 
  broader search space and escape local optima.

  \textit{Crossover}, also known as recombination, is another crucial genetic 
  operator.
  It combines the genetic material from a set of parent individuals to generate
  offspring, simulating the biological process of sexual reproduction.

  Each of these categories encompasses several techniques, providing
  \textit{Keen} with a versatile repertoire of genetic operators.
  The subsequent subsections provide a comprehensive discussion on each of these
  operators, elucidating their working principles and application contexts.

  \subsection{Word Guessing Problem}
  \label{sec:keen:operators:word_guessing}
    The \emph{Word Guessing Problem} is a simple example that illustrates the
    application of genetic operators.
    The problem statement is as follows: given a target word, the objective is
    to evolve a population of strings that match the target word.
    The fitness of each individual is determined by the number of characters
    that match the target word.
    The operators will be tested against a set of randomly generated target
    words, each with a length \(n\) between 1 and 100 characters.

  \subsection{Selection}
  \label{sec:keen:operators:selection}
    \subsubsection{Random Selector}
    \label{sec:keen:operators:selection:random}
      \Blindtext
    \subsubsection{Roulette Wheel Selector}
    \label{sec:keen:operators:selection:roulette_wheel}
      \Blindtext
    \subsubsection{Tournament Selector}
    \label{sec:keen:operators:selection:tournament}
      \Blindtext
  \subsection{Mutation}
  \label{sec:keen:operators:mutation}
    \subsubsection{Bit Flip Mutator}
    \label{sec:keen:operators:mutation:bit_flip}
      \Blindtext
    \subsubsection{Random Mutator}
    \label{sec:keen:operators:mutation:simple}
      \Blindtext
    \subsubsection{Swap Mutator}
    \label{sec:keen:operators:mutation:swap}
      \Blindtext
    \subsubsection{Inversion Mutator}
    \label{sec:keen:operators:mutation:inversion}
      \Blindtext
  \subsection{Crossover}
  \label{sec:keen:operators:crossover}
    \subsubsection{Combine Crossover}
    \label{sec:keen:operators:crossover:combine}
      \Blindtext
    \subsubsection{Mean Crossover}
    \label{sec:keen:operators:crossover:mean}
      \Blindtext
    \subsubsection{Ordered Crossover (OX)}
    \label{sec:keen:operators:crossover:ordered}
      \Blindtext
    \subsubsection{Partially Mapped Crossover (PMX)}
    \label{sec:keen:operators:crossover:partially_mapped}
      \Blindtext
    \subsubsection{Position Based Crossover (PBX)}
    \label{sec:keen:operators:crossover:position_based}
      \Blindtext
    \subsubsection{Single-Point Crossover}
    \label{sec:keen:operators:crossover:single_point}
      \Blindtext
