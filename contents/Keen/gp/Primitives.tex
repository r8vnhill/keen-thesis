% Copyright (c) 2023 Ignacio Slater Muñoz All rights reserved.
% Use of this source code is governed by a BSD-style
% license that can be found in the LICENSE file.

\subsection{Primitive Set}
\label{sec:keen:gp:primitives}
  The quintessential component of any GP implementation is the set of
  \textit{primitives} that define the language of the GP system. The
  \textit{Keen} framework's GP implementation is no exception, and its
  primitives are defined in the \texttt{prog} package.

  The \texttt{prog} package exposes three main components: \emph{reduceable}
  expressions, \emph{program} trees, and \emph{generator} functions.

  \subsubsection{Reduceable Expressions}
    Primitives in \textit{Keen} are represented by the \texttt{Reduceable}
    interface, which defines a single method, \texttt{invoke}, that accepts a
    list of arguments and returns a single value. The \texttt{invoke} method
    is analogous to a function call, and the \texttt{Reduceable} interface
    represents a function signature.

    The proposed \texttt{Reduceable} interface can be seen in 
    \vref{lst:sec:keen:gp:primitives:reduce}.

    \begin{code}{
      The \texttt{Reduceable} interface
    }{
      label=lst:sec:keen:gp:primitives:reduce
    }{kotlin}
      interface Reduceable<T> : Node<Reduceable<T>> {
          operator fun invoke(args: List<T>): T
      }      
    \end{code}

    Note that the \texttt{Reduceable} interface extends from 
    \texttt{Node<Reduceable<T>>}.
    This means that a \emph{reduceable} expression is also a node in an
    expression tree. This is a key feature of the \textit{Keen} framework's
    GP implementation, as it allows for the creation of arbitrarily complex
    expressions.

    \textit{Keen} then extends the \texttt{Reduceable} into two sets of
    primitives: \emph{terminals} and \emph{functions}. Terminals are
    reduceable expressions that represent leaf nodes in an expression tree.
    Functions are reduceable expressions that represent intermediate nodes in
    an expression tree.

    For the implementation of the primitives, we propose the creation of a
    \texttt{Terminal} interface, and the \texttt{Fun}\footnote{
      Other names were considered, such as \texttt{Function} and \texttt{Op}.
      However, these names were discarded to maintain consistency with the
      function declaration syntax of the Kotlin programming language; 
      \mintinline{kotlin}|fun name(args: List<T>): T|.
    } open class.
    The implementations could be as follows:

    \begin{code}{Prototype of the \texttt{Terminal} interface}{
      label=lst:sec:keen:gp:primitives:terminal
    }{kotlin}
      interface Terminal<T> : Reduceable<T>, Leaf<Reduceable<T>> {
          fun create(): Terminal<T>
          override val arity: Int
              get() = 0
      }
    \end{code}

    \begin{code}{Prototype of the \texttt{Fun} open class}{
      label=lst:sec:keen:gp:primitives:fun
    }{kotlin}
      open class Fun<T>(
          private val name: String,
          override val arity: Int,
          private val body: (List<T>) -> T
      ) : Reduceable<T>, Intermediate<Reduceable<T>> {
          override fun invoke(args: List<T>) = body(args)
      }
    \end{code}

    Note that both the \texttt{Terminal} interface and the \texttt{Fun} class
    are parameterized by a type \texttt{T}. This type represents the return
    type of the \texttt{invoke} method, and is used to enforce type safety
    when creating expressions.

    We argue that the usage of an open class for the \texttt{Fun} class is
    more appropriate than the usage of an interface. This is because a function
    is an expression with arbitrary behavior, and thus, creating an interface
    and extending it for each function would be unfeasible. Instead, we propose
    the usage of an open class, which allows for generic functions, as well as
    extending the class to provide more control over the behavior of the
    function.
    This will allow users to directly use the \texttt{Fun} class to create
    functions, as well as extending it to create more complex functions.
    For example, \vref{lst:sec:keen:gp:primitives:fun:add} shows how the
    addition operation could be implemented by using the \texttt{Fun} class and
    by extending it.
    % Mover a apendice
    \begin{code}{
      Example of using the \texttt{Fun} class to implement the addition 
      operation
    }{
      label=lst:sec:keen:gp:primitives:fun:add
    }{kotlin}
      // Using the Fun class
      var add = Fun<Double>("+", 2) { it[0] + it[1] }
      // Extending the Fun class
      class Add : Fun<Double>("+", 2, { it[0] + it[1] })
      var add = Add()
      // As an anonymous object
      var add = object : Fun<Double>("+", 2, { it[0] + it[1] }) {}
    \end{code}