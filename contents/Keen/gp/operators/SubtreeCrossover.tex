\subsubsection{Subtree Crossover}
\label{sec:keen:gp:op:cx:subtree}
  In the realm of Genetic Programming (GP), the subtree crossover operator
  emerges as a pivotal tool to foster genetic diversity and probe the vast 
  solution space. This operator is adept at melding genetic information from 
  two parent program trees, paving the way for outputs that may exhibit 
  enhanced fitness values.

  \begin{definition}[Subtree Crossover]
    The \textit{subtree crossover} is a genetic operator designed for 
    tree-structured chromosomes. For two given trees, it randomly chooses a 
    node or subtree in each. The subtrees anchored at these points are then 
    interchanged, birthing two new outputs. The operator can be mathematically 
    expressed as:

    \begin{equation}
      X_{subtree}: \mathbb{P} \times \{0,\, 1\} \times [0,\, 1] 
        \times [0,\, 1] \times [0,\, 1] \to \mathbb{P};\;
        (P,\, e,\, \rho_\mathbf{i},\, \rho_\mathbf{c},\, \rho_\mathbf{g}) 
        \mapsto X_{subtree}(P,\, e,\, \rho_\mathbf{i},\, \rho_\mathbf{c},\, \rho_\mathbf{g})
    \end{equation}

    Here's a brief rundown of the parameters:

    \begin{itemize}
      \item \(P\): A population of program trees.
      \item \(e\): A flag indicating if an individual can participate in 
        crossover more than once.
      \item \(\rho_\mathbf{i}\): Probability of an individual undergoing 
        crossover.
      \item \(\rho_\mathbf{c}\): Chance of initiating the subtree crossover.
      \item \(\rho_\mathbf{g}\): Likelihood of selecting a gene for the 
        crossover process.
    \end{itemize}
  \end{definition}

  \begin{remark}
    \textit{Keen}'s rendition of the subtree crossover currently equips all 
    tree nodes with equal selection probabilities. This strategy may evolve 
    in subsequent versions to encompass a more sophisticated selection 
    mechanism.
  \end{remark}

  To comprehend the subtree crossover's operation within \textit{Keen}, let's 
  walk through the code snippet below:

    \begin{code}{
      Illustration of Subtree Crossover in \textit{Keen}
    }{label=lst:keen:gp:cx:subtree}{kotlin}
      val (node1, node2) = (random node from input1, random node from input2)
      val slices = (gene1.dna.searchSubtree(node1), gene2.dna.searchSubtree(node2))
      val newTree1 = gene1.dna.replaceSubtree(slices.first, node2)
      val newTree2 = gene2.dna.replaceSubtree(slices.second, node1)
    \end{code}

  Delve deeper into this operator by consulting
  \vref{sec:bg:gp:variation:mutation}, where a vivid example of the subtree 
  crossover comes alive.

  \textit{Keen} is judiciously engineered to address potential challenges posed 
  by the subtree crossover, such as generating unmanageably large trees or 
  producing trees devoid of semantic significance for a given problem. Such 
  astute handling ensures the subtree crossover's unwavering contribution to 
  steering the evolutionary journey towards excellence.
