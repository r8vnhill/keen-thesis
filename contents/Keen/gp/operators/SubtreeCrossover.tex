\begin{definition}[Subtree Crossover]
    The \textit{subtree crossover} is a genetic operator designed for 
    tree-structured chromosomes. For two given trees, it randomly chooses a 
    node or subtree in each. The subtrees anchored at these points are then 
    interchanged, birthing two new outputs. The operator can be mathematically 
    expressed as:

    \begin{equation}
        X_{subtree}(P: \mathbb{P}, \rho_\mathbf{i}: [0,\, 1], \rho_\mathbf{c}: [0,\, 1], \rho_\mathbf{g}: [0,\, 1], e: \{0,\, 1\}) \to \mathbb{P}
    \end{equation}

    Here's a brief rundown of the parameters:

    \begin{itemize}
    \item \(P\): A population of program trees.
    \item \(\rho_\mathbf{i}\): Probability of an individual undergoing crossover.
    \item \(\rho_\mathbf{c}\): Chance of initiating the subtree crossover.
    \item \(\rho_\mathbf{g}\): Likelihood of selecting a gene for the crossover process.
    \item \(e\): A flag indicating if an individual can participate in crossover more than once.
    \end{itemize}
\end{definition}

\begin{remark}
    \textit{Keen}'s rendition of the subtree crossover currently equips all 
    tree nodes with equal selection probabilities. This strategy may evolve 
    in subsequent versions to encompass a more sophisticated selection 
    mechanism.
\end{remark}

To understand the subtree crossover's inner workings, let's consider the following pseudocode:

\begin{code}{
    Illustration of Subtree Crossover in \textit{Keen}
}{label=lst:keen:gp:cx:subtree}{kotlin}
    val (node1, node2) = (random node in input1, random node in input2)
    val slices = (gene1.value.searchSubtree(node1), gene2.value.searchSubtree(node2))
    val newTree1 = gene1.value.replaceSubtree(slices.first, node2)
    val newTree2 = gene2.value.replaceSubtree(slices.second, node1)
    \end{code}

An extensive explanation of this operator was provided in \vref{sec:bg:gp:variation:mutation}.
