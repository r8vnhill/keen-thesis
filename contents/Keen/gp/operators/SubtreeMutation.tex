\begin{definition}[Subtree Mutation]
    The \textit{subtree mutation} operator functions by picking a node from a program tree and replacing the subtree 
    from that node with a new one. This method maintains the overall structure of the program but adds new genetic 
    elements, potentially improving the program's performance.

    The mathematical representation of this operation is:

    \begin{equation}
        M_{subtree}: \mathbb{P} \times [0,\, 1] \times [0,\, 1] \times [0,\, 1] \rightarrow \mathbb{P};\;
        (P,\, \mu_\textbf{i},\, \mu_\textbf{c},\, \mu_\textbf{g}) 
            \mapsto M_{subtree}(P,\, \mu_\textbf{i},\, \mu_\textbf{c},\, \mu_\textbf{g})
    \end{equation}

    Explaining the parameters:

    \begin{itemize}
        \item \(P\): The population of program trees.
        \item \(\mu_\textbf{i}\): The chance of a tree undergoing mutation.
        \item \(\mu_\textbf{c}\): The likelihood of choosing a chromosome for mutation.
        \item \(\mu_\textbf{g}\): The probability of selecting a gene for mutation.
    \end{itemize}
\end{definition}
