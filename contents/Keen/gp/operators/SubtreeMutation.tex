\subsubsection{Subtree Mutation}
\label{sec:keen:op:mut:subtree}
  One key technique among mutation operators in GP is the Subtree Mutation. It 
  serves to adjust the genetic structure of program trees.

  \begin{definition}[Subtree Mutation]
    The \textit{subtree mutation} operator works by selecting a node from a 
    program tree. It then replaces the subtree originating from this node with 
    a new subtree. This process keeps the general shape of the program while 
    introducing fresh genetic elements, which might enhance the program's 
    effectiveness.

    To paint a mathematical portrait of this operation:

    \begin{equation}
      M_{subtree}: \mathbb{P} \times [0,\, 1] \to \mathbb{P};\;
        (P,\, \mu_\textbf{st}) \mapsto M_{subtree}(P,\, \mu_\textbf{st})
    \end{equation}

    Breaking down the parameters:

    \begin{itemize}
      \item \(P\): Embodies a population of program trees.
      \item \(\mu_\textbf{st}\): Represents the probability of a subtree 
        undergoing mutation.
    \end{itemize}
  \end{definition}

  Here's a more simplified version:

  In the \textit{Keen} framework, the Subtree Mutation process happens like 
  this:

  \begin{code}{
    Explaining Subtree Mutation in \textit{Keen}
  }{label=lst:keen:op:mut:subtree}{kotlin}
    val targetNode = choose random node from program tree
    val newSubtree = create a new subtree
    val mutatedTree = replace targetNode's subtree in program tree with newSubtree
  \end{code}

  In essence, the Subtree Mutation in \textit{Keen} adds new genetic components 
  without heavily changing the main structure of the tree.

  \begin{remark}
    The Subtree Mutation is great at making significant genetic changes. 
    However, it's important to make sure the new subtrees fit well with the 
    problem you're solving. Without this check, you might end up with trees 
    that don't make sense or are too complex.

    \textit{Keen} handles this by using generation methods mentioned in 
    \vref{sec:keen:gp:primitives:gen}.
  \end{remark}

  The beauty of the Subtree Mutation method lies in its balance. It keeps what 
  works while trying out new genetic possibilities.

  For a visual representation of the Subtree Mutation process, see 
  \vref{fig:keen:op:mut:subtree}.

  \begin{figure}[ht!]
    \centering
    \includegraphics[width=0.5\textwidth]{img/keen/STM.png}
    \caption{
      A visual guide to the Subtree Mutation process. The operator picks a node 
      and replaces its attached subtree with a new one, giving a changed but 
      still recognizable program tree.
    }
    \label{fig:keen:op:mut:subtree}
  \end{figure}
