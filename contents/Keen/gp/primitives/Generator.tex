% Copyright (c) 2023 Ignacio Slater Muñoz All rights reserved.
% Use of this source code is governed by a BSD-style
% license that can be found in the LICENSE file.

\subsubsection{Generation Methods}
\label{sec:keen:gp:primitives:gen}

  Every individual within the population should consistently be a valid program 
  throughout the evolution process. To guarantee the syntactic correctness of 
  all newly created or varied individuals, the GP algorithm must generate an 
  initial population of syntactically accurate individuals. The designated 
  algorithm for producing this initial population is termed the 
  \textit{generation method} or \textit{generator function}.

  Koza's influential work~\autocite{kozaGeneticProgrammingProgramming1992a} 
  introduced two pivotal algorithms for generating the initial population: the 
  \textit{full} and \textit{grow} methods.

  The \emph{full} method creates a tree with a specified maximum depth \(D\), 
  ensuring that all terminal nodes are located at depth \(D\).

  Contrastingly, the \emph{grow} method crafts a tree with a maximum depth of 
  \(D\) and a minimum depth of \(d\), positioning all terminal nodes at depths 
  ranging between \(d\) and \(D\). This method was elaborated upon in 
  \vref{sec:bg:gp:init}.

  It's noteworthy that the \emph{full} method can be viewed as a specific 
  instance of the \emph{grow} method when \(d = D\). Given this perspective, we 
  can further conceptualize the \emph{grow} method as a
  \enquote{depth-conditional tree generation} strategy. The detailed algorithm 
  for this is presented in \vref{lst:keen:gp:primitives:gen}.

  \begin{code}{Depth-conditional tree generation}{
    label=lst:keen:gp:primitives:gen
  }{kotlin}
    fun generateRecursive(
        intermediates: List<Intermediate>,
        leafs: List<Leaf>,
        depth: Int,
        maxHeight: Int,
        condition: (maxHeight: Int, depth: Int) -> Boolean,
        leafFactory: (Leaf) -> Tree,
        intermediateFactory: (Intermediate, List<Tree>) -> Tree,
    ) {
        val children = []
        val node = if (condition(maxHeight, depth)) {
            leafFactory(random node from leafs)
        } else {
            val intermediate = random node from intermediates
            repeat intermediate.arity times {
                children += generateRecursive(
                    intermediates,
                    leafs,
                    depth + 1,
                    maxHeight,
                    condition,
                    leafFactory,
                    intermediateFactory
                )
            }
            intermediateFactory(intermediate, children)
        }
        return node
    }
  \end{code}

  The \texttt{generateRecursive} function accepts a list of intermediate nodes, 
  a list of leaf nodes, a maximum height, a current depth, and a condition 
  function. The condition function receives the maximum height and the current 
  depth of a node, returning a boolean value. If the condition is met (true), a 
  leaf node is formed; if not (false), an intermediate node is created. This 
  function is invoked recursively to form the descendants of the current node, 
  halting the recursion when the condition is true or when the maximum height is 
  attained. Additionally, the \texttt{generateRecursive} function takes in two 
  other functions dedicated to the creation of both leaf and intermediate nodes, 
  enabling its adaptability for any tree type, including program trees.

  A salient feature of this algorithm is its bottom-up tree generation approach, 
  commencing with terminal nodes and culminating with the root node. This 
  strategy facilitates the production of immutable trees, obviating the 
  necessity for crafting temporary trees.

  The \texttt{generateRecursive} function underpins the \texttt{generate} 
  function, which simply invokes the recursive function, setting a maximum 
  height chosen randomly between \(d\) and \(D\).

  Using this method, the grow and full techniques can be illustrated as:

  \begin{code}{Grow and full methods}{
    label=lst:keen:gp:primitives:gen:growfull
  }{kotlin}
    fun <T> generateProgramGrowing(
        terminals: List<Terminal<T>>,
        functions: List<Fun<T>>,
        min: Int,
        max: Int,
    ) = generateProgram(terminals, functions, min, max) { h, d ->
        d == h || d >= min && random.nextDouble() < terminals.size / (terminals.size + functions.size)
    }

    fun <T> generateProgramFull(
        terminals: List<Terminal<T>>, functions: List<Fun<T>>, min: Int, max: Int,
    ) = generateProgram(terminals, functions, min, max) { h, d -> h == d }
    
    fun <T> generateProgram(
        terminals: List<Terminal<T>>,
        functions: List<Fun<T>>,
        min: Int,
        max: Int,
        condition: (Int, Int) -> Boolean,
    ) = Tree.generate(terminals to functions, min, max, condition, ::Program) { f, c ->
        Program(f, c)
    }
  \end{code}

  The \texttt{generateProgramGrowing} function obtains a list of terminal nodes, 
  a collection of intermediate nodes, a minimum and maximum height, and returns 
  a program tree. Its condition function is articulated based on the following 
  criteria:

  \begin{itemize}
    \item If the present depth mirrors the maximum height, the condition holds 
      true.
    \item Should the current depth equal or surpass the minimum height and a 
      random value between 0 and 1 is less than the proportion between the 
      quantity of terminal nodes and the number of intermediate nodes, the 
      condition is validated.
    \item In all other scenarios, the condition remains unfulfilled.
  \end{itemize}

  The \texttt{generateProgramFull} function mirrors the aforementioned one, with 
  the distinction that its condition is only validated when the current depth 
  matches the maximum height.

  Koza also suggested that population members could be derived using a 
  \textit{ramped half-and-half} technique. Here, the initial population is 
  formed utilizing both the \textit{full} and \textit{grow} strategies, with the 
  method for each individual being selected at random. To implement this, 
  \textit{Keen} offers the \texttt{generateProgramWith} function, receiving a 
  list of generation strategies, a compilation of terminals and functions, and a 
  minimum and maximum height. This function randomly chooses a generation method 
  to produce the program tree. By defining the function this way, integrating 
  new generation techniques into the library becomes seamless.

  Crucially, with this approach, the \textit{ramped half-and-half} technique 
  simply becomes a specialized version of the \\
  \texttt{generateProgramWith} function, wherein the list of generation 
  strategies is confined to the \textit{full} and \textit{grow} methodologies.
  