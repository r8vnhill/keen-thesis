\subsubsection{Reducible Expressions}
  Within the \textit{Keen} framework, \enquote{reducible expressions} serve as
  the building blocks for creating more complex mathematical or logical 
  constructs. Primitives in \textit{Keen} are captured using the 
  \texttt{Reducible} interface. This interface outlines a single method, 
  \texttt{invoke}, which accepts a list of arguments and yields a single 
  value. Drawing parallels, the \texttt{invoke} method resembles a function 
  call, and the \texttt{Reducible} interface stands as a function signature.

  The architecture of the proposed \texttt{Reducible} interface can be 
  appreciated in \vref{lst:sec:keen:gp:primitives:reduce}.

  \begin{code}{
    The \texttt{Reducible} interface
  }{
    label=lst:sec:keen:gp:primitives:reduce
  }{kotlin}
    interface Reducible<T> : Node<Reducible<T>>, SelfReferential<Reducible<T>> {
        override val contents: Reducible<T> get() = this
        operator fun invoke(environment: Environment<T>, args: List<T>): T
        operator fun <T> Reducible<T>.invoke(environment: Environment<T>, vararg args: T): T =
            invoke(environment, args.toList())
    }
  \end{code}

  An important aspect of the \texttt{Reducible} interface is the 
  \emph{environment}; a data structure that stores the values of the
  variables used in the expression.
  
  It's essential to underline that the \texttt{Reducible} interface is an 
  extension of \texttt{Node<Reducible<T>>}. This inherently implies that any 
  reducible expression is also a node within an expression tree. Such an 
  arrangement forms a cornerstone of \textit{Keen}'s Genetic Programming (GP) 
  approach as it lays the foundation for generating expressions of arbitrary 
  intricacy.

  Venturing forward, \textit{Keen} subdivides the \texttt{Reducible} into 
  two primitive subsets: \emph{terminals} and \emph{functions}. The former, 
  terminals, embody reducible expressions that act as leaf nodes in an 
  expression tree. 
  Conversely, functions serve as reducible expressions that denote 
  intermediate nodes in said tree.

  \begin{code}{Prototype of the \texttt{Terminal} interface}{
    label=lst:sec:keen:gp:primitives:terminal
  }{kotlin}
    interface Terminal<T> : Reducible<T>, Leaf<Reducible<T>> {
        fun create(): Terminal<T>
    }
  \end{code}

  \begin{code}{Prototype of the \texttt{Fun} open class}{
    label=lst:sec:keen:gp:primitives:fun
  }{kotlin}
    open class Fun<T>(
        val name: String, override val arity: Int, val body: (List<T>) -> T
    ) : Reducible<T>, Intermediate<Reducible<T>> {
        override fun invoke(environment: Environment<T>, args: List<T>): T { /* ... */ }
    }
  \end{code}

  Note that both the \texttt{Terminal} interface and the \texttt{Fun} class
  are parameterized by a type \texttt{T}. This type represents the return
  type of the \texttt{invoke} method, and is used to enforce type safety
  when creating expressions.

  A crucial aspect of our design strategy involves the employment of an open 
  class for the \texttt{Fun} class, a decision we consider more fitting than 
  resorting to an interface. Given that functions symbolize expressions with 
  an innate arbitrary behavior, architecting an interface and extending it 
  for each unique function emerges as an impractical venture. In contrast, an 
  open class stands out as a versatile choice, facilitating the crafting of 
  generic functions and paving the way for class extensions to enhance the 
  behavior of the function. This structural framework empowers users to 
  directly harness the \texttt{Fun} class for function creation or extend it 
  for more intricate functions, as exemplified in \vref{lst:app:keen_fun_add}.
  