% Copyright (c) 2023 Ignacio Slater Muñoz All rights reserved.
% Use of this source code is governed by a BSD-style
% license that can be found in the LICENSE file.

\subsubsection{Program trees}
  The \texttt{Reducible} interface and its subsets--terminals and 
  functions--are foundational components that aid in the construction of 
  program trees in \textit{Keen}. The framework represents these program 
  trees as immutable tree structures, a design decision rooted in advantages 
  such as thread safety, predictability, and reduced chances of accidental 
  data modification, in particular, making program trees immutable one can 
  assure that the arity of a node is always correct.

  In essence, a program tree is characterized by a reducible expression 
  situated at its root, which in turn references other programs as its 
  descendants. This design is vividly portrayed in the blueprint provided in 
  \vref{lst:sec:keen:gp:primitives:prog}.

  \begin{code}{Prototype of the \texttt{Program} class}{
    label=lst:sec:keen:gp:primitives:prog
  }{kotlin}
    class Program<V>(
        override val value: Reducible<V>,
        override val children: List<Program<V>>,
    ) : Tree<Reducible<V>, Program<V>> {
        /* ... */
        operator fun invoke(environment: Environment<V>, vararg args: V): V =
            value(environment, children.map { it(environment, *args) })
    }
  \end{code}

  The \texttt{Program} class employs a type \texttt{V} as a parameter, 
  symbolizing the return type of the \texttt{invoke} method intrinsic to the 
  \texttt{Reducible} interface. Such an approach ensures type-safety during the 
  construction of program trees. Notably, the \texttt{invoke} function adopts a 
  recursive mechanism. It calls upon the \texttt{invoke} method of the root 
  reducible expression, subsequently passing the results derived from the 
  recursive invocations of its children.

  Forming immutable trees, particularly in recursive architectures, can be a 
  challenging endeavor. To counteract this complexity, the \texttt{Tree} 
  interface introduces a method to instantiate trees by conducting a top down
  tree construction. The underlying algorithm, designed to transform a list of 
  nodes into trees, is presented in \vref{lst:sec:keen:gp:primitives:tree}.

  \begin{code}{Algorithm to create a tree from a list of nodes}{
    label=lst:sec:keen:gp:primitives:tree
  }{kotlin}
    val stack = []
    nodes.reversed().forEach { node ->
        val children = stack.take(node.arity)
        stack.removeAll(children)
        val node = createNode(node.value, children)
        stack += node
    }
    return stack.first()
  \end{code}

  This algorithm takes a list of nodes and creates a tree by traversing the
  list in reverse order. For each node, it takes the required number of
  children from the top of the stack, removes them from the stack, and
  creates a new node with the given value and children. Finally,
  it returns the top of the stack, which is the root of the tree.
