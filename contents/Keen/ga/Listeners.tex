\subsection{Evolution Listeners}
\label{sec:keen:ga:listeners}

  The \textit{Keen} framework introduces an \texttt{EvolutionListener} 
  interface designed with an \textit{observer pattern} to offer users insights 
  into the evolution process in real-time.
  This interface provides hooks for various events in the evolution cycle, 
  ranging from its commencement and culmination to the beginning and conclusion 
  of each phase, including generation, evaluation, and selection, among others.

  The primary objective of the \texttt{EvolutionListener} is to empower users 
  to oversee the evolution process, leveraging this data for diverse purposes. 
  Users can employ this interface to conduct a detailed analysis of the 
  evolution process, generate comprehensive reports based on its results, or 
  persistently store the best-performing individual of each generation.

  Within the \textit{Keen} framework, several concrete implementations of this 
  interface are available:

  \begin{itemize}
    \item \texttt{EvolutionPlotter}: Visualizes the best, worst, and average 
      fitness throughout the generations in a plot format.
    \item \texttt{EvolutionPrinter}: Outputs essential evolution data every 
      \(n\) generations, detailing aspects such as fitness metrics, the 
      top-performing individual, and the progression of generations.
    \item \texttt{EvolutionSummary}: Prints a comprehensive summary 
      post-evolution, encapsulating fitness statistics, leading individuals, 
      generation count, time metrics for each evolution phase, and the 
      evolution's total duration.
  \end{itemize}

  \begin{remark}
    The \emph{Evolution plotter} uses \textit{Lets-Plot Skia Frontend}~\autocite{JetBrainsLetsplotskiaSkia} to render
    the plots, which is a multi-platform plotting library for Kotlin, meaning that the plots can be rendered in any 
    platform supported by Kotlin Multi-Platform.
  \end{remark}

  The implementation of this interface is a novel feature of the \textit{Keen}
  framework, and it is not present in the frameworks that were used as a
  reference for this work.
