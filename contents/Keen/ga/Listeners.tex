\subsection{Evolution Listeners}
\label{sec:keen:ga:listeners}

  The \textit{Keen} framework introduces an \texttt{EvolutionListener} 
  interface designed with an \textit{observer pattern} to offer users insights 
  into the evolution process in real-time.
  This interface provides hooks for various events in the evolution cycle, 
  ranging from its commencement and culmination to the beginning and conclusion 
  of each phase, including generation, evaluation, and selection, among others.

  The primary objective of the \texttt{EvolutionListener} is to empower users 
  to oversee the evolution process, leveraging this data for diverse purposes. 
  Users can employ this interface to conduct a detailed analysis of the 
  evolution process, generate comprehensive reports based on its results, or 
  persistently store the best-performing individual of each generation.

  To facilitate these functions, the interface features the following methods:

  \begin{code}{
    Evolution Listener Interface (some variables and functions omitted for 
    brevity)
  }{label=lst:keen:ga:listeners:interface}{kotlin}
    interface EvolutionListener<DNA, G : Gene<DNA, G>> {
        fun onEvolutionEnded() = Unit
        fun onGenerationStarted(generation: Int, population: Population<DNA, G>) = Unit
        fun onGenerationFinished(population: Population<DNA, G>) = Unit
        fun onInitializationStarted() = Unit
        fun onInitializationFinished() = Unit
        fun onEvaluationStarted() = Unit
        fun onEvaluationFinished() = Unit
        fun onOffspringSelectionStarted() = Unit
        fun onOffspringSelectionFinished() = Unit
        fun onSurvivorSelectionStarted() = Unit
        fun onSurvivorSelectionFinished() = Unit
        fun onAlterationStarted() = Unit
        fun onAlterationFinished() = Unit
        fun onEvolutionStart() = Unit
        fun onEvolutionFinished() = Unit
    }
  \end{code}

  Within the \textit{Keen} framework, several concrete implementations of this 
  interface are available:

  \begin{itemize}
    \item \texttt{EvolutionPlotter}: Visualizes the best, worst, and average 
      fitness throughout the generations in a plot format.
    \item \texttt{EvolutionPrinter}: Outputs essential evolution data every 
      \(n\) generations, detailing aspects such as fitness metrics, the 
      top-performing individual, and the progression of generations.
    \item \texttt{EvolutionSummary}: Prints a comprehensive summary 
      post-evolution, encapsulating fitness statistics, leading individuals, 
      generation count, time metrics for each evolution phase, and the 
      evolution's total duration.
    \item \texttt{JsonSerializer}: Serializes and saves the population data of 
      every generation as a JSON file, complemented by pertinent evolution 
      details.
  \end{itemize}

  The implementation of this interface is a novel feature of the \textit{Keen}
  framework, and it is not present in the frameworks that were used as a
  reference for this work.

  \subsubsection{Records}
    The framework harnesses records to chronicle information pertaining to the 
    evolution process.

    \paragraph{Generation Record}
      The \texttt{GenerationRecord} class embodies data of an individual 
      generation, encapsulating each phase of the evolutionary journey within a 
      generation record.

      \begin{code}{
        Generation Record Class (some variables and functions omitted for 
        brevity)
      }{label=lst:keen:ga:listeners:generation-record}{kotlin}
        @Serializable
        data class GenerationRecord(val generation: Int) : AbstractTimedRecord() {
            val alteration = AlterationRecord()
            val evaluation = EvaluationRecord()
            val offspringSelection = SelectionRecord()
            val survivorSelection = SelectionRecord()
            val population = PopulationRecord()
            var steady: Int = ...
        }
      \end{code}
      
    \paragraph{Evolution Record}
      The \texttt{EvolutionRecord} class, while centered on the entire 
      evolutionary process, primarily delegates data storage responsibilities 
      to the \texttt{GenerationRecord} class.

      \begin{code}{
        Evolution Record Class (some variables and functions omitted for 
        brevity)
      }{label=lst:keen:ga:listeners:evolution-record}{kotlin}
        @Serializable
        data class EvolutionRecord<DNA, G : Gene<DNA, G>>(
            val generations: MutableList<GenerationRecord> = mutableListOf(),
        ) : AbstractTimedRecord() {
            val initialization = InitializationRecord()
        }
      \end{code}
      