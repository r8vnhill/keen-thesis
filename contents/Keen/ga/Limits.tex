\subsubsection{Termination Criteria}
  The algorithm employs a set of \texttt{Limit} objects to define its 
  termination criteria.
  These objects contain the logic to determine when the algorithm should halt, 
  thus permitting multiple criteria definitions without requiring 
  engine-specific awareness of their details.

  The structure of the \texttt{Limit} interface is shown in 
  \vref{lst:keen:ga:limit}.
  \begin{code}{Limit interface}{label={lst:keen:ga:limit}}{kotlin}      interface Limit {
    interface Limit<T, G> where G : Gene<T, G> {
        var engine: Evolver<T, G>?
        operator fun invoke(state: EvolutionState<T, G>): Boolean
    }
  \end{code}

  This interface is composed of two elements:
  \begin{enumerate*}
    \item The \texttt{engine} property, which is used to provide the limit with
      access to the engine's state, and
    \item The \texttt{invoke} operator, which is used to determine whether the
      algorithm should halt.
  \end{enumerate*}

  This design lets the \texttt{Limit} object function as an invocable entity, 
  as demonstrated in line 13 of \vref{lst:keen:ga:engine}.

  Furthermore, since the engine was defined in a way that does not store
  information about the evolution process, a novel approach is proposed, in
  which the \textit{Limit} registers itself as a listener of the engine.
  This way, the \textit{Limit} can store any information and perform any
  computation it requires, without the need for the engine to be aware of it.
  We claim this approach is more flexible and extensible than the traditional
  approach of having the engine store the information and perform the
  computation itself.

  Currently, the framework incorporates four predefined termination criteria:
  \begin{enumerate*}
  \item \emph{Generation limit}: halts after a designated number of generations.
  \item \emph{Steady generations}: halts after a certain count of 
  non-improving generations.
  \item \emph{Target fitness}: ceases once a defined fitness value is 
  attained.
  \item \emph{Listener limit}: halts when a given listener signals the
    termination of the algorithm. This class is open and thus can be extended 
    to incorporate any custom termination criteria.
  \end{enumerate*}
  