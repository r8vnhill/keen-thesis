\subsection{Evolution Engine}
\label{sec:keen:ga:engine}

  The \emph{evolution engine} serves as the framework's nucleus, orchestrating the
  evolution of the population.
  It encapsulates all functions essential for the evolutionary process.
  This engine integrates the fundamental structure of the evolutionary algorithms,
  invoking relevant functions from associated entities like the 
  \emph{genetic material} and \emph{genetic operators}.

  Specifically, the engine manages:

  \begin{enumerate}
    \item Population initialization;
    \item Population evaluation;
    \item Selection of individuals for the forthcoming generation;
    \item Genetic operator application on chosen individuals;
    \item Evaluation of the newly formed individuals;
    \item Supplanting the old population with the new;
    \item Iteration from step 3 until meeting termination criteria.
  \end{enumerate}

  Furthermore, the engine offers hooks for both pre and post-processing steps
  related to the population and individuals.
  These hooks facilitate user-driven customization of the evolution trajectory.
  Although this feature remains outside this work's scope, its availability merits
  acknowledgment.

  The engine's architecture is depicted in \vref{lst:keen:ga:engine}.

  \begin{code}*[Evolution engine structure][label={lst:keen:ga:engine}]{kotlin}
    class Engine<DNA, G : Gene<DNA, G>>(...) : Evolver<DNA, G> {
        override fun evolve(): EvolutionResult<DNA, G> {
            var evolution = EvolutionState.empty<DNA, G>()
            var result = EvolutionResult(optimizer, evolution.population, generation)
            while (limits.none { it(this) }) {
                result = evolve(evolution)
                evolution = result.next()
            }
            return result
        }
        fun evolve(start: EvolutionState<DNA, G>) = runBlocking {
            val interceptedStart = interceptor.before(start)
            val evolution = startEvolution(interceptedStart)
            val evaluatedPopulation = evaluate(evolution)
            val offspring = selectOffspring(evaluatedPopulation)
            val survivors = selectSurvivors(evaluatedPopulation)
            val alteredOffspring = alter(offspring, evolution)
            val nextPopulation = survivors + alteredOffspring.population
            val pop = evaluate(EvolutionState(nextPopulation, generation), true)
            evolutionResult = EvolutionResult(optimizer, pop, ++generation)
            val afterResult = interceptor.after(evolutionResult)
            afterResult
        }

        class Builder<DNA, G : Gene<DNA, G>>(
            private val fitnessFunction: (Genotype<DNA, G>) -> Double,
            private val genotype: Genotype.Factory<DNA, G>,
        ) {
            fun build() = Engine(...)
            ...
        }
    }
  \end{code}

  In the provided code, the function \texttt{runBlocking} harnesses 
  \textit{coroutines} to allow parallel computation.

  It's worth noting that two distinct \texttt{evolve} functions exist: 
  
  \begin{itemize}
    \item \texttt{evolve()} oversees the entire evolutionary procedure.
    \item \texttt{evolve(start: EvolutionState<DNA, G>)} orchestrates a singular
      iteration of the evolutionary process, commencing from a specified
      \texttt{EvolutionState}.
  \end{itemize}

  The supplementary \texttt{Builder} class streamlines the engine configuration 
  process.
  Although its utilization is recommended for engine instantiation, it remains 
  optional, ensuring flexibility for users.

  Finally, the \texttt{Evolver} interface is designed to accommodate varying 
  engine implementations, with the \texttt{Engine} class being the exclusive 
  existing implementation to date.

  \paragraph{Termination Criteria}
    The algorithm employs a set of \texttt{Limit} objects to define its 
    termination criteria.
    These objects contain the logic to determine when the algorithm should halt, 
    thus permitting multiple criteria definitions without requiring 
    engine-specific awareness of their details.

    The structure of the \texttt{Limit} interface is shown in 
    \vref{lst:keen:ga:limit}.

    \begin{code}[Limit interface][label={lst:keen:ga:limit}]{kotlin}
      interface Limit {
          operator fun invoke(engine: Engine<*, *>): Boolean
      }
    \end{code}

    Within this interface, the exclusive function is the \texttt{invoke} 
    operator.
    This design lets the \texttt{Limit} object function as an invokable entity, 
    as demonstrated in line 5 of \vref{lst:keen:ga:engine}.

    Currently, the framework incorporates four predefined termination criteria:
    \begin{enumerate*}
      \item \texttt{GenerationCount}: halts after a designated number of 
        generations.
      \item \texttt{SteadyGenerations}: halts after a certain count of 
        non-improving generations.
      \item \texttt{TargetFitness}: ceases once a defined fitness value is 
        attained.
      \item \texttt{Matcher}: stops when a predetermined predicate is 
        met.\footnote{
          Note that all criteria fundamentally use the \texttt{Matcher} class.
        }
    \end{enumerate*}
