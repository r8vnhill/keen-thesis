\section{Architecture}
\label{sec:architecture}
  The architectural blueprint of \textit{Keen} is divisible into five distinct
  modules: \textit{evolution engine}, \textit{genetic material}, 
  \textit{genetic operators}, \textit{utility}, and \textit{programs}.

  The \emph{utility} module, as implied by its name, houses utility classes and 
  functions serving other modules.
  We won't delve into its specifics in this document.

  The \emph{genetic material} module is composed of classes embodying the 
  characteristics of the evolving individuals, known as \texttt{Phenotype}s.
  Its primary components include the \texttt{Phenotype} and \texttt{Genotype} 
  classes, along with the \texttt{Chromosome} and \texttt{Gene} hierarchies.

  The \emph{genetic operators} module comprises classes representing the 
  genetic operators leveraged by the evolution engine for individuals' 
  evolution.
  It houses classes typifying crossover and mutation operators, further 
  categorized into \textbf{crossover}, \textbf{mutation}, and 
  \textbf{selection}.

  The \emph{evolution engine} module contains classes defining the evolution 
  engine and the requisite classes to facilitate the evolution process.

  A fifth module, \emph{programs}, depicts programs evolved under the 
  \textit{genetic programming} paradigm.
  As it's in its nascent development stages, this module's comprehensive 
  coverage will not be featured in this document.

  Most of the \textit{Keen} classes are parameterized by the type of 
  information they hold (for instance, an integer) and the type of gene that 
  composes the individuals of the population (for instance, an integer gene).
  This design offers immense flexibility, empowering users to define the 
  genetic material utilized by the evolution engine.
  Thanks to \textit{Kotlin}'s type inference system, most use cases allow the 
  omission of type parameters when the compiler can infer them, enhancing code 
  readability.

  Designed with immutability at its core, most of \textit{Keen}'s classes are 
  immutable.
  This design ensures robustness and predictability as users can be assured 
  that objects remain unchanged post-creation.
  This philosophy also enhances \textit{Keen}'s parallelization capabilities by 
  allowing shared memory use without necessitating synchronization mechanisms.

  \textit{Keen} offers factory methods for most classes, promoting concise, 
  readable code.
  This is accomplished using a language-oriented programming technique to 
  create a mini \textit{domain specific language} (DSL) that facilitates more 
  intuitive and readable object creation.
  While factory methods are \textit{Keen}'s preferred object creation mode, the 
  framework's design permits "manual" object creation if users need more 
  control over the creation process.
  An example of the DSL is shown below, where a \texttt{Genotype} is created
  for the \textit{Room Scheduling} problem, where each \texttt{Chromosome}
  represents a meeting and is composed of a single integer gene representing
  the meeting's assigned room.

  \begin{minted}{kotlin}
    genotype {
        repeat(meetings.size) {
            chromosome {
                ints { 
                    size = 1
                    range = meetings.indices.first to meetings.indices.last 
                }
            }
        }
    }
  \end{minted}
  
  The following sections offer a deeper exploration into these modules, 
  spotlighting key classes and their functions.
