
\section{Genetic Operators}
\label{sec:keen:operators}
  Genetic operators are the driving force behind the evolution process in any
  genetic algorithm or genetic programming paradigm.
  These operators stimulate change in the population and guide the search
  towards optimal solutions.
  

  This section explains the basic genetic operators in \textit{Keen}, including \textit{Selection}, \textit{Mutation}, and \textit{Crossover}.

  \textit{Selection} operators guide the evolutionary process by determining
  which individuals from the population will contribute to the next generation.
  These mechanisms often favor individuals that exhibit superior fitness, 
  thereby promoting the inheritance of advantageous traits.

  \textit{Mutation} operators introduce randomness into the population by 
  modifying individual genes or gene sequences.
  This mechanism ensures genetic diversity, aiding the population to explore a 
  broader search space and escape local optima.

  \textit{Crossover}, also known as recombination, is another crucial genetic 
  operator.
  It combines the genetic material from a set of parent individuals to generate
  offspring, simulating the biological process of sexual reproduction.

  Each of these categories encompasses several techniques, providing
  \textit{Keen} with a versatile repertoire of genetic operators.
  The subsequent subsections provide a comprehensive discussion on each of these
  operators, elucidating their working principles and application contexts.

  \subimport{selection/}{_selection.index.tex}
  \subimport{mutation/}{_mutation.index.tex}
  \subimport{crossover/}{_crossover.index.tex}
