
\section{Genetic Operators}
\label{sec:keen:operators}
  Genetic operators are the driving force behind the evolution process in any genetic algorithm or genetic programming 
  paradigm. These operators stimulate change in the population and guide the search towards optimal solutions.
  

  This section explains the basic genetic operators in \textit{Keen}, including \textit{Selection}, \textit{Mutation}, 
  and \textit{Crossover}.

  Each of these categories encompasses several techniques, providing \textit{Keen} with a versatile repertoire of
  genetic operators. The subsequent subsections provide a comprehensive discussion on each of these operators, 
  elucidating their working principles and application contexts.
    
  Before delving into the various genetic operators, we must take into consideration the special case where individuals
  are programs (specifically trees). Keen represents this individuals in a way that each gene is a program tree, this
  means that operators that work on the chromosome level will not affect the individuals. For this reason, Keen provides
  special operators for this case, which are described in \vref{sec:keen:gp:op}.

  \subimport{selection/}{_selection.index.tex}
  \subimport{mutation/}{_mutation.index.tex}
  \subimport{crossover/}{_crossover.index.tex}
