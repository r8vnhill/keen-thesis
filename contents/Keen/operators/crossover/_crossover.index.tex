\subsection{Crossover}
\label{sec:keen:op:cx}
    Crossover, also termed recombination, serves as an indispensable operator in a plethora of evolutionary computation 
    algorithms. Its primary function is to generate novel candidate solutions by amalgamating attributes from multiple 
    input solutions. This operation, inspired by genetic mechanisms observed in biological reproduction, is pivotal for 
    striking a balance between exploration and exploitation within an algorithm's search space. By fostering a diverse 
    set of solution structures and retaining beneficial traits, crossover steers the evolutionary process closer to 
    optimal or near-optimal solutions.

    Within the \textit{Keen} framework, a diverse array of crossover techniques has been conceptualized and developed, 
    each epitomizing a distinctive recombination approach. This section endeavors to furnish an exhaustive overview of 
    these methodologies, elucidating their mechanics, applications, and subtle intricacies.

    
    \subsubsection{Generalized Crossover Algorithm}
        A novel feature of \textit{Keen} is its innovative approach to crossover. Diverging from conventional 
        frameworks that typically restrict crossover to two-parent models, \textit{Keen} provides users the flexibility 
        to engage multiple parents in the crossover process. This multi-parent crossover approach aligns with 
        contemporary research findings~\autocite{tsutsuiMultiparentRecombinationSimplex1999,elsayedGANewMultiparent2011,arramNovelMultiparentOrder2019}, which advocate for the benefits of such strategies.

        To facilitate this advanced methodology, \textit{Keen} introduces a \textit{generalized crossover} algorithm. 
        This algorithm forms the basis for multi-parent crossover operations, offering a template for users to develop 
        custom crossover strategies.

        \begin{code}{
            Generalized crossover algorithm in \textit{Keen}, serving as a foundation for multi-parent crossover.
        }{
            label=code:keen:op:cx:generalized
        }{kotlin}
            val parents = Random.subsets(population, numParents, exclusivity)
            val recombined = []
            while (|recombined| < outputSize) {
                recombined += crossover(parents.random())
            }
        \end{code}

        This algorithm initiates by selecting parent groups from the population, governed by the \texttt{numParents} 
        parameter. The \texttt{exclusivity} parameter dictates whether parents can be chosen multiple times. 
        Subsequently, these selected parents are recombined to generate offspring, until the desired number of offspring 
        (\texttt{outputSize}) is reached. This algorithm is deliberately designed to be agnostic to the number of 
        parents, thus allowing for versatile implementation of various multi-parent crossover methods.

        The subset selection algorithm, an integral part of the generalized crossover process, is detailed in 
        \vref{code:keen:op:cx:subset}.

        \begin{code}{
            Subset selection algorithm utilized in the generalized crossover process.
        }{
            label=code:keen:op:cx:subset
        }{kotlin}
            fun <T> Random.subsets(
                elements: List<T>,
                size: Int,
                exclusive: Boolean,
            ): List<List<T>> {
                val subsets = []
                val remainingElements = elements.shuffled
                while (remainingElements is not empty) {
                    val subset = if (exclusive) {
                        remainingElements.take(size)
                    } else {
                        List(size) { it ->
                            if (it == 0) remainingElements.removeFirst()
                            else remainingElements.random().also { 
                                element -> remainingElements.remove(element) 
                            }
                        }
                    }
                    subsets += subset
                    remainingElements = remainingElements.drop(size)
                }
                return subsets
            }
        \end{code}

        In this algorithm, the elements are first shuffled to ensure randomness. Subsets of elements are then formed 
        based on the specified size. If \texttt{exclusive} is \texttt{true}, unique elements are chosen for each subset. 
        Otherwise, the first element of each subset is unique, while the rest are randomly selected and removed from the 
        remaining pool to avoid repetition. This approach guarantees that every element is included in at least one 
        subset, thereby ensuring comprehensive coverage and variety in the crossover process.

    \subsubsection{Crossover Strategies}
    \subimport{./}{Combine.tex}
    \subimport{./}{SinglePoint.tex}
    \subimport{./}{OX.tex}
    \subimport{./}{PMX.tex}
    \subimport{./}{PBX.tex}
  