\subsection{Crossover}
\label{sec:keen:operators:crossover}
  Crossover, also termed recombination, serves as an indispensable operator in 
  a plethora of evolutionary computation algorithms. Its primary function is to 
  generate novel candidate solutions by amalgamating attributes from multiple 
  input solutions. This operation, inspired by genetic mechanisms observed in 
  biological reproduction, is pivotal for striking a balance between 
  exploration and exploitation within an algorithm's search space. By fostering 
  a diverse set of solution structures and retaining beneficial traits, 
  crossover steers the evolutionary process closer to optimal or near-optimal 
  solutions.

  Within the \textit{Keen} framework, a diverse array of crossover techniques 
  has been conceptualized and developed, each epitomizing a distinctive 
  recombination approach. This section endeavors to furnish an exhaustive 
  overview of these methodologies, elucidating their mechanics, applications, 
  and subtle intricacies.

  What distinguishes \textit{Keen} is its adaptable approach to crossover. Many 
  frameworks typically limit a crossover to two parents, but \textit{Keen} 
  offers users the flexibility to determine the number of parents they wish to 
  involve. This adaptability necessitated a reevaluation of terminology. In 
  many contexts, the term \enquote{parents} is naturally associated with a 
  pair, mirroring biological reproduction. To avoid potential confusion, 
  especially in scenarios with more than two parents, we opted for the term 
  \textit{inputs} to replace \enquote{parents} and \textit{outputs} for 
  \enquote{offspring}. Research supports this broader view of crossover, with 
  studies advocating for multi-parent crossover approaches, including those by 
  Tsutsui and Yamamura~\autocite{tsutsuiMultiparentRecombinationSimplex1999}, 
  Elsayed et al.\autocite{elsayedGANewMultiparent2011}, and Arram and 
  Ayob\autocite{arramNovelMultiparentOrder2019}. Although \textit{Keen} does 
  not feature multi-parent crossovers at present, its architecture paves the 
  way for potential inclusion in the future.
  
  \subsubsection{Combine Crossover}
  \label{sec:keen:operators:crossover:combine}
    \Blindtext
  \subsubsection{Mean Crossover}
  \label{sec:keen:operators:crossover:mean}
    \Blindtext
  \subsubsection{Ordered Crossover (OX)}
  \label{sec:keen:operators:crossover:ordered}
    \Blindtext
  \subsubsection{Partially Mapped Crossover (PMX)}
  \label{sec:keen:operators:crossover:partially_mapped}
    \Blindtext
  \subsubsection{Position Based Crossover (PBX)}
  \label{sec:keen:operators:crossover:position_based}
    \Blindtext
  \subsubsection{Single-Point Crossover}
  \label{sec:keen:operators:crossover:single_point}
    \Blindtext