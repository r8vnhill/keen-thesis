\subsubsection{Combine Crossover}
\label{sec:keen:op:cx:combine}

    A practical and effective strategy for executing crossover in genetic algorithms is through what we termed as 
    \textit{combine crossover}. This method involves applying a function that performs element-wise transformations on 
    parent inputs to produce offspring. In the \textit{Keen} framework, this is encapsulated within the 
    \texttt{CombineCrossover} class.\footnote{
        \textit{Jenetics} also provides a similar operator, known as the \textit{Combine Alterer}, however, it is 
        restricted to two parents. We also deem the name \textit{Combine Crossover} to be more intuitive and 
        descriptive since it explicitly mentions the crossover aspect of the operator.
    }

    \begin{definition}[\(n\)-Combine Crossover]
        The \textit{\(n\)-combine crossover} operator is defined as a function that accepts \(n\) inputs and applies a 
        designated combiner function to these inputs, yielding a single combined offspring. Formally, the combine 
        crossover can be expressed as:

        \begin{equation}
            X_{n\text{-comb}} :\: 
                \mathbb{P}_T \times (T^n \rightarrow T) \times [0,\, 1] \times [0,\, 1] \times [0,\, 1] 
                    \times \{0,\, 1\} \rightarrow \mathbb{P}_T;\;
                (P,\, f,\, \rho_i,\, \rho_\mathbf{c},\, \rho_g,\, e)
                    \mapsto X_{n\text{-comb}}(P,\, f,\, \rho_i,\, \rho_\mathbf{c},\, \rho_g,\, e)
        \end{equation}

        where:

        \begin{itemize}
            \item \(P\) represents the population of individuals of type \(T\).
            \item \(f\) is the combiner function, taking \(n\) inputs to produce an output.
            \item \(\rho_i\) is the probability of applying the crossover to an individual in the population.
            \item \(\rho_\mathbf{c}\) is the probability of the combiner function affecting a chromosome.
            \item \(\rho_g\) is the gene-level application probability of the function.
            \item \(e\) is a boolean value that determines whether the same individual can be used more than once as a 
                parent.
        \end{itemize}
    \end{definition}

    \begin{remark}
        The core principle of the combine crossover lies in the multi-parent crossover concept, wherein the arity of the 
        combiner function aligns with the number of input parents.
    \end{remark}

    Notably, the \textit{Combine Crossover} operator in \textit{Keen} is designed as an \textbf{open class}. This design 
    choice allows for extensive customization, enabling developers to extend and tailor the operator according to 
    specific needs. An illustrative example is the \texttt{AverageCrossover} operator, which calculates the arithmetic 
    mean of the parent inputs, demonstrating the versatility and adaptability of the \textit{Combine Crossover} in 
    various genetic algorithm scenarios.
