\subsubsection{Combine Crossover}
\label{sec:keen:op:cx:combine}

  One of the straightforward strategies for executing crossover in genetic 
  algorithms is to utilize a function that transforms numerical inputs into numerical outputs. Termed the \textit{combine crossover}, this approach 
  processes multiple inputs through a designated function to generate outputs. 
  In \textit{Keen}, this is realized through the \texttt{CombineCrossover} 
  class.

  \begin{definition}[Combine Crossover]
    The \textit{combine crossover} operator ingests a set of numerical inputs 
    and employs a specific function to them, producing combined offspring. More 
    formally, a combine crossover is represented as:

    \begin{equation}
      X_\mathrm{combine} :\: 
        \mathbb{P}_\mathbb{R} \times (\mathbb{R}^n \rightarrow \mathbb{R})
          \times [0,\, 1] \times [0,\, 1] \times [0,\, 1] 
            \rightarrow \mathbb{P}_\mathbb{R};\;
        (P,\, f,\, \rho_i,\, \rho_\mathbf{c},\, \rho_g)
          \mapsto X_\mathrm{combine}(P,\, f,\, \rho_i,\, \rho_\mathbf{c},\, 
            \rho_g)
    \end{equation}

    with:

    \begin{itemize}
      \item \(P\) denoting a population of numerical entities.
      \item \(f\) representing a combiner function that processes a set of 
        numerical inputs to yield a singular numerical outcome.
      \item \(\rho_i\) symbolizing the likelihood of the crossover's application
        on an individual of the population.
      \item \(\rho_\mathbf{c}\) symbolizing the likelihood of the function's 
        application on a chromosome.
      \item \(\rho_g\) indicating the probability of the function influencing a 
        gene.
    \end{itemize}
  \end{definition}

  \begin{remark}
    The essence of combine crossover is rooted in the multi-parent crossover 
    paradigm, where the number of input parents matches the arity of the 
    employed combiner function.
  \end{remark}

  Significantly, the \texttt{CombineCrossover} operator in \textit{Keen} is 
  architected as an \textbf{open class}, empowering developers to extend it and 
  craft bespoke combiner operators. A case in point is the 
  \texttt{AverageCrossover} operator, which computes the arithmetic mean of the 
  inputs.
