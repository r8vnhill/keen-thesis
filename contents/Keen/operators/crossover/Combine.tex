\begin{definition}[\(n\)-Combine Crossover]
    The \textit{\(n\)-combine crossover} operator is defined as a function that accepts \(n\) inputs and applies a 
    designated combiner function to these inputs, yielding a single combined offspring. Formally, the combine 
    crossover can be expressed as:

    \begin{equation}
        X_{n\text{-comb}}(P: \mathbb{P}_T,\, f: T^n \rightarrow T,\, \rho_i: [0,\, 1],\, \rho_\mathbf{c}: [0,\, 1],\, 
            \rho_g: [0,\, 1],\, e: \{0,\, 1\})\to \mathbb{P}_T
    \end{equation}

    where:

    \begin{itemize}
        \item \(P\) represents the population of individuals of type \(T\).
        \item \(f\) is the combiner function, taking \(n\) inputs to produce an output.
        \item \(\rho_i\) is the probability of applying the crossover to an individual in the population.
        \item \(\rho_\mathbf{c}\) is the probability of the combiner function affecting a chromosome.
        \item \(\rho_g\) is the gene-level application probability of the function.
        \item \(e\) is a boolean value that determines whether the same individual can be used more than once as a 
            parent.
    \end{itemize}
\end{definition}

\begin{remark}
    The core principle of the combine crossover lies in the multi-parent crossover concept, wherein the arity of the 
    combiner function aligns with the number of input parents.
\end{remark}

Notably, the \textit{Combine Crossover} operator in \textit{Keen} is designed as an \textbf{open class}. This design 
choice allows for extensive customization, enabling developers to extend and tailor the operator according to 
specific needs. An illustrative example is the \texttt{AverageCrossover} operator, which calculates the arithmetic 
mean of the parent inputs, demonstrating the versatility and adaptability of the \textit{Combine Crossover} in 
various genetic algorithm scenarios.
