\begin{definition}[Single-Point Crossover]
    The \textit{single-point crossover} operator takes a pair of input chromosomes and randomly selects a crossover point 
    on them. It then exchanges the subsequences of genes situated after this point between the two input chromosomes to 
    produce two outputs. More formally, a single-point crossover is represented as:

    \begin{equation}
        X_\mathrm{sp}(P: \mathbb{P},\, \rho_i: [0,\, 1],\, \rho_\mathbf{c}: [0,\, 1],\, e: \{0,\, 1\}) \to \mathbb{P}
    \end{equation}

    where:

    \begin{itemize}
        \item \(P\) represents a population of binary chromosomes.
        \item \(\rho_i\) symbolizes the likelihood of applying the crossover to 
            an individual.
        \item \(\rho_\mathbf{c}\) symbolizes the likelihood of applying the 
            crossover to a chromosome.
        \item \(e\) is a boolean value that determines whether the same individual can be used as both parents.
    \end{itemize}
\end{definition}

\begin{remark}
    The simplicity of single-point crossover makes it a popular choice in many EC applications. However, it might not 
    always ensure adequate exploration of the search space, especially in problems where gene positions have strong 
    interactions. In these scenarios, permutation crossover methods like \textit{ordered crossover} and 
    \textit{partially mapped crossover} might be more effective.
\end{remark}

For better clarity on the single-point crossover operation, see the graphical 
representation in \vref{fig:bg:ga:var:cx:single_point}, which showcases the 
random selection of a crossover point and the subsequent exchange of gene 
subsequences.

Note: There exists a generalized version of single-point crossover named 
\textit{multi-point crossover}. This method selects multiple crossover points 
on the input chromosomes and exchanges the subsequences of genes situated 
after these points between the two input chromosomes to produce two 
outputs. However, this method is currently not supported in \textit{Keen}.
