% Copyright (c) 2023 Ignacio Slater Muñoz All rights reserved.
% Use of this source code is governed by a BSD-style
% license that can be found in the LICENSE file.

\subsubsection{Swap Mutator}
\label{sec:keen:op:mut:swap}
  The \texttt{SwapMutator} is a mutation operator designed to enhance 
  population diversity by interchanging the positions of two genes in a 
  chromosome. Unlike conventional mutation techniques that modify the intrinsic 
  value of a gene, the \texttt{SwapMutator} focuses on the positional 
  rearrangement of genes within a chromosome.

  \begin{definition}[Swap Mutator]
  \label{def:keen:op:mut:swap}
    The \emph{swap mutator}, denoted by

    \begin{equation}
      M_\mathrm{swap} :\: \mathbb{P} \times [0,\, 1] \times [0,\, 1] 
        \times [0,\, 1] \to \mathbb{P};\;
      (P,\, \mu,\, \rho_\mathbf{c},\, \rho_g) 
        \mapsto M_\mathrm{swap}(P,\, \mu,\, \rho_\mathbf{c},\, \rho_g)
    \end{equation}

    is an operator that, with given probabilities, chooses a chromosome from a 
    population and swaps the positions of two genes within it. In the equation 
    above, \(P\) represents the population, \(\mu\) denotes the probability of 
    mutation, \(\rho_\mathbf{c}\) quantifies the likelihood of chromosome 
    selection, and \(\rho_g\) determines the gene selection probability.

    Its versatility lies in its applicability to any chromosome, regardless of 
    its specific genetic representation.
  \end{definition}

  The mechanism of the \texttt{SwapMutator} is elucidated below:

  \begin{code}{
    Schematic representation of the \texttt{SwapMutator}. This representation simplifies the mutator's workings, particularly considering the assumption of chromosome mutability. In contrast, \textit{Keen} adopts an immutable approach.
  }{
    label=lst:keen:op:mut:swap
  }{kotlin}
    var genesToSwap = []
    for (gene in chromosome) {
        if (random.nextDouble() < ~\(\rho_g\)~) {
            genesToSwap += gene
        }
    }
    for (gene in genesToSwap) {
        chromosome.swap(gene.index, random.int(0..chromosome.size))
    }
  \end{code}

  \begin{remark}
    This swap mutator implementation permits the possibility of selecting the 
    same gene multiple times, leading to potential multiple swaps involving the 
    same gene. While increasing the population's diversity, this approach also 
    reflects \textit{Keen's} design choice. Implementers seeking to negate this 
    behavior can design a tailored mutator.
  \end{remark}

  One of the prominent use-cases of the swap mutator is in combinatorial 
  optimization tasks, where gene positioning holds significance. An exemplar 
  problem is the \textit{Travelling Salesman Problem} (TSP), aiming to 
  ascertain the shortest possible route that visits each city precisely once, 
  subsequently returning to the starting city. Given that both the order of 
  visiting and the distances between cities matter, the swap mutator becomes 
  invaluable in exploring potential solutions without altering the cities' 
  identity. Comprehensive studies elucidating the swap mutator's efficacy in 
  TSP are beyond the scope of this thesis. However, interested readers can 
  refer to Bergel's work~\autocite{bergelAgileArtificialIntelligence2020} and 
  Abdoun et al.~\autocite{abdounAnalyzingPerformanceMutation2012}.
