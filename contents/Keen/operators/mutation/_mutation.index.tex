\subsection{Mutation}
\label{sec:keen:operators:mutation}
  Mutation, a cornerstone operator in evolutionary computation, plays a critical 
  role in introducing genetic variance.
  By sporadically modifying parts of a solution representation (often called a 
  chromosome or individual), mutation ensures the presence of a rich and diverse
  genetic material, which is essential for the robustness and adaptability of 
  evolved solutions.
  While mutation is commonly associated with genetic algorithms, its 
  significance spans across a multitude of evolutionary computation techniques, 
  from genetic programming to evolutionary strategies and beyond.
  The method of introducing these mutations can vary significantly, tailored to 
  the nature of the solution space and the specific requirements of the problem 
  at hand.
  This section delves into various mutation strategies, elucidating their 
  characteristics and applications:

  Recalling what we saw in \vref{sec:bg:ga:var:mut}, a mutation operator can be
  defined as a function \(M :\: \mathbb{P} \times \mathbb{R} \times \cdots \to 
  \mathbb{P}\).
  This operator picks individuals from the population with a probability, 
  \(\mu\), and modifies part of its genetic material.

  \textit{Keen}'s approach is similar to the one used for the selection
  operators.
  The following interface is proposed:

  \begin{code}{
    Mutator interface (some segments of the code are omited for brevity.)
   }{}{kotlin}
    interface Mutator<DNA, G: Gene<DNA, G>> : Alterer<DNA, G> {
        override fun invoke(
            population: Population<DNA, G>,
            generation: Int,
        ): AltererResult<DNA, G> 
        fun mutatePhenotype(
            phenotype: Phenotype<DNA, G>,
        )
        fun mutateGenotype(
            genotype: Genotype<DNA, G>,
        ): MutatorResult<DNA, G, Genotype<DNA, G>>
        fun mutateChromosome(
            chromosome: Chromosome<DNA, G>,
        ): MutatorResult<DNA, G, Chromosome<DNA, G>> 
        fun mutateGene(
            gene: G,
        ): MutatorResult<DNA, G, G>
    }
  \end{code}



\subsubsection{Random Mutator}
\label{sec:keen:operators:mutation:simple}
\Blindtext

\subsubsection{Bit Flip Mutator}
\label{sec:keen:operators:mutation:bit_flip}
\Blindtext

\subsubsection{Swap Mutator}
\label{sec:keen:operators:mutation:swap}
\Blindtext

\subsubsection{Inversion Mutator}
\label{sec:keen:operators:mutation:inversion}
\Blindtext
