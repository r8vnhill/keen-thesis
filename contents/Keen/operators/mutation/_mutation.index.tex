\subsection{Mutation}
\label{sec:keen:op:mut}
  Mutation is a crucial operator in evolutionary computation, acting as the main source of genetic variation.
  This operation sporadically alters parts of a solution, commonly referred to 
  as a chromosome or individual. Such alterations ensure that the genetic 
  material remains diverse and rich—qualities vital for producing robust and 
  adaptable solutions. Even though mutation is often linked with genetic 
  algorithms, its importance is felt throughout various evolutionary 
  computation methods, from genetic programming to evolutionary strategies.

  The methodology of mutation differs depending on the solution space's nature 
  and the problem's unique demands. This section will examine various mutation strategies, highlighting their unique characteristics and applications.

  From our previous discussion in \vref{sec:bg:ga:var:mut}, a mutation operator 
  can be articulated as a function \(M :\: \mathbb{P} \times [0,\, 1] \times 
  \cdots \to \mathbb{P}\). This operator selects individuals from a population 
  based on a probability, \(\mu\), and then alters a portion of its genetic 
  composition.

  The approach adopted by \textit{Keen} mirrors that of selection operators. 
  The proposed interface is as follows:

  \begin{code}{
    Mutator interface (Note: For clarity, certain parts of the code have been omitted.)
  }{}{kotlin}
    interface Mutator<T, G> where G : Gene<T, G> {
        val individualRate: Double
        val chromosomeRate: Double
        operator fun invoke(
            state: EvolutionState<T, G>, outputSize: Int
        ): EvolutionState<T, G>
        fun mutateIndividual(individual: Individual<T, G>): Individual<T, G>
        fun mutateChromosome(chromosome: Chromosome<T, G>): Chromosome<T, G>
    }
  \end{code}

  This design contemplates mutations at different stages of the solution 
  blueprint, focusing primarily on the chromosome level. It provides default
  \texttt{inkove} and \texttt{mutateIndividual} methods (omitted from the code 
  for the sake of brevity), directing the mutation task to the 
  \texttt{mutateChromosome} method. This design choice facilitates ease for
  users: they only need to implement the \texttt{mutateChromosome} method when
  devising a new mutator, but still retain the freedom to override other methods 
  if necessary.

  \subimport{./}{Random.tex}
  \subimport{./}{BitFlip.tex}
  \subimport{./}{Swap.tex}
  \subimport{./}{Inversion.tex}
