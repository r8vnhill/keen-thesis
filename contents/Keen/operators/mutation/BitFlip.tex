% Copyright (c) 2023 Ignacio Slater Muñoz All rights reserved.
% Use of this source code is governed by a BSD-style
% license that can be found in the LICENSE file.

\subsubsection{Bit Flip Mutator}
\label{sec:keen:op:mut:bit_flip}

  The \texttt{BitFlipMutator} stands out as one of the quintessential mutation 
  operators specifically tailored for binary chromosomes. As the name suggests, 
  its operation revolves around inverting individual bits within a chromosome, 
  providing a simple yet effective means of introducing variability in the 
  genetic representation.

  At its core, the \texttt{BitFlipMutator} iterates over each bit of a binary 
  chromosome. With a predetermined mutation probability \( \mu \), it decides 
  whether to flip a given bit from 0 to 1 or vice versa. Such a straightforward 
  approach ensures that the fundamental structure of the chromosome remains 
  intact, while still allowing for diverse offspring.

  \begin{definition}[Bit Flip Mutator]
    The \emph{bit flip mutator} is a mutation operator tailored for binary 
    genetic representations. It can be mathematically formulated as:

    \begin{equation}
      M_{0/1} :\: \mathbb{P} \times [0,\, 1] \times [0,\, 1] \to \mathbb{P};\;
      (P,\, \mu,\, \rho_\mathbf{c},\, \rho_g) 
        \mapsto M_{0/1}(P,\, \mu,\, \rho_\mathbf{c},\, \rho_g)
    \end{equation}

    In this equation, \(P\) symbolizes the population, and \(\mu\) is the 
    predefined probability determining the likelihood of a given individual
    undergoing mutation, \(\rho_\mathbf{c}\) is the probability of a chromosome
    being selected for mutation, and \(\rho_g\) is the probability of a gene
    being selected for mutation.
  \end{definition}

  Note that this operator is similar to applying a \texttt{RandomMutator} to a 
  binary chromosome with a probability of \(\mu\). However, when a given
  gene is selected for mutation, the \texttt{BitFlipMutator} always flips the
  gene's value, whereas the \texttt{RandomMutator} randomly assigns a new value
  to the gene. Take for example a binary chromosome with the following genetic
  representation: 011\textcolor{red}{0}1001. If the \texttt{BitFlipMutator}
  selects the fourth gene for mutation (denoted by the red color), it will
  always produce the following result: 011\textcolor{red}{1}1001. On the other
  hand, the \texttt{RandomMutator} will randomly assign a new value to the
  gene, which might result in the following: 011\textcolor{red}{1}1001 or
  011\textcolor{red}{0}1001.
