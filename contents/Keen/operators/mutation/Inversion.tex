% Copyright (c) 2023 Ignacio Slater Muñoz All rights reserved.
% Use of this source code is governed by a BSD-style
% license that can be found in the LICENSE file.

\subsubsection{Inversion Mutator}
\label{sec:keen:op:mut:inversion}
  The class \texttt{InversionMutator} embodies the inversion mutation operator. 
  This operator, upon activation, selects a subset of genes from a chromosome 
  and inverts their sequence. Analogous to the \texttt{SwapMutator}, it 
  primarily targets the gene ordering.

  \begin{definition}[Inversion Mutator]
  \label{def:keen:op:mut:inversion}
    The inversion mutator works by selecting a random subset of genes from a 
    chromosome and subsequently inverting their sequence. Formally, the 
    operator can be represented as:

    \begin{equation}
      M_\mathrm{inv} :\: 
        \mathbb{P} \times [0,\, 1] \times [0,\, 1] \times [0,\, 1] 
          \to \mathbb{P};\;
      (P,\, \mu,\, \rho_\mathbf{c},\, \rho_{ib}) 
        \mapsto M_\mathrm{inv}(P,\, \mu,\, \rho_\mathbf{c},\, \rho_{ib})
    \end{equation}

    Where:
    \begin{itemize}
        \item \(P\) represents the population.
        \item \(\mu\) denotes the mutation probability.
        \item \(\rho_\mathbf{c}\) is the chromosome-wise mutation rate.
        \item \(\rho_{ib}\) is the \textit{inversion boundary} probability. 
          This probability is used to determine the starting and ending index 
          of the inversion.
    \end{itemize}
  \end{definition}

  The operator's function can be outlined as follows:

  \begin{code}{
      Pseudo-code depicting the inversion mutation process. For illustrative
      purposes, we consider the chromosome as mutable, even if it isn't in the
      \textit{Keen} framework.
    }{
      label=code:keen:op:mut:inversion
    }{kotlin}
      var start = 0
      var end = chromosome.size - 1
      for (i in chromosome.indices) {
          if (random.double(0..1) < ~$\rho_{ib}$~) {
              start = i
              break
          }
      }
      for (i in chromosome.indices) {
          if (random.double(0..1) > ~$\rho_{ib}$~) {
              end = i
              break
          }
      }
      chromosome.invert(start, end)
  \end{code}

  Two pivotal steps are present in this operation: 
  \begin{enumerate*}
    \item The determination of the starting index, and
    \item The identification of the ending index.
  \end{enumerate*}

  The starting index is discerned by sequentially scanning the chromosome and 
  selecting a gene based on the gene-wise mutation rate. The ending index is 
  determined in a similar manner, initiating from the previously established 
  start index. This methodology ensures only one iteration over the chromosome, 
  optimizing the time complexity of the operation. The inversion is 
  subsequently applied to the selected gene subset.

  Much like the \texttt{SwapMutator}, the inversion mutator exhibits prominence 
  in permutation-centric problems, with the TSP being a notable example. Abdoun 
  et al. provide a comprehensive analysis of the inversion operator's efficacy, 
  especially in the context of the TSP~\autocite{abdounAnalyzingPerformanceMutation2012}.

  The content is clear, but there are some minor improvements that can be made to further enhance clarity and maintain formality:

  \begin{remark}
    A vital consideration must be made when interpreting the inversion operator 
    within the \textit{Keen} framework. Specifically, the gene-wise mutation 
    rate, \(\rho_{ib}\), signifies the probability of selecting a particular gene 
    for the starting or ending index. It should not be misconstrued as a metric 
    indicating the percentage of genes selected for inversion.
  \end{remark}
