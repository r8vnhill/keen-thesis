\subsubsection{Random Mutator}
\label{sec:keen:op:mut:simple}

  The \texttt{RandomMutator} represents a fundamental mutation operator that
  introduces random alterations to the genes within a chromosome. Its versatility 
  lies in its capacity to function with any gene type, making it a universally 
  adaptable mutation mechanism.

  To underpin this mutator's generic nature, each gene is equipped with a
  \texttt{mutate} function. This function, when invoked, produces a randomly 
  mutated gene of the same type as the origin. By default, the function generates 
  a new gene mirroring the initial gene's value. This design choice provides 
  users with the flexibility to determine if a specific mutation operator should 
  employ this function. As we'll observe, certain mutation operators opt not to 
  utilize it.

  \begin{definition}[Random Mutator]
    The \emph{random mutator} serves as a rudimentary mutation operator, 
    undertaking random alterations to the genes of a given chromosome. 
    Formally, it is articulated as:

    \begin{equation}
      M_\mathrm{random} :\: \mathbb{P} \times [0,\, 1] \times [0,\, 1] 
        \times [0,\, 1] \to \mathbb{P};\;
      (P,\, \mu,\, \rho_\textbf{c}, \rho_g) 
        \mapsto M_\mathrm{random}(P,\, \mu,\, \rho_\textbf{c}, \rho_g)
    \end{equation}

    Here, \(P\) symbolizes the population, \(\mu\) represents the likelihood of 
    an individual undergoing mutation, \(\rho_\textbf{c}\) is the probability
    of a chromosome being selected for mutation, and \(\rho_g\) is the
    probability of a gene being selected for mutation.
  \end{definition}

  % Insert discussion or analysis of the results here

  % \begin{table}[H]
  %   % Insert table content here
  %   \caption{
  %     The table encapsulates an assessment of the \texttt{RandomMutator}'s 
  %     performance within the WGP, traversing an array of word lengths and 
  %     mutation probabilities. The experimental setup employed a single-point 
  %     crossover characterized by a 0.2 crossover probability and was 
  %     complemented by a tournament selection approach, configured with a 
  %     tournament size of 3.
  %   }
  %   \label{tab:keen:op:mut:random}
  % \end{table}
