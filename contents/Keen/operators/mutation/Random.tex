\begin{definition}[Random Mutator]
  The \emph{random mutator} serves as a rudimentary mutation operator, undertaking random alterations to the genes of a 
  given chromosome. Formally, it is articulated as:

  \begin{equation}
    M_\mathrm{random}(P: \mathbb{P},\, \mu: [0,\, 1],\, \rho_\textbf{c}: [0,\, 1],\, \rho_g: [0,\, 1]) \to \mathbb{P}
  \end{equation}

  Here, \(P\) symbolizes the population, \(\mu\) represents the likelihood of an individual undergoing mutation, 
  \(\rho_\textbf{c}\) is the probability of a chromosome being selected for mutation, and \(\rho_g\) is the probability 
  of a gene being selected for mutation.
\end{definition}

\begin{remark}
  To underpin this mutator's generic nature, each gene is equipped with a \texttt{mutate} function. This function, when invoked, produces a randomly mutated gene of the same type as the original. By default, the function generates a new 
  gene mirroring the initial gene's value. This design choice provides users with the flexibility to determine if a 
  specific mutation operator should employ this function. As we'll observe, certain mutation operators opt not to 
  utilize it.
\end{remark}
