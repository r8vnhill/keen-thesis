\subsubsection{Random Selector}
\label{sec:keen:operators:selection:random}
  The \texttt{RandomSelector} class acts as a selection strategy, assigning 
  equal selection probabilities to every individual within a population.
  Its primary utility is to serve as a baseline for comparison with other 
  selection methods.

  \begin{definition}[Random Selector]
  \label{def:keen:operators:selection:random}
    The \emph{random selector} operates under a uniform approach, ensuring every 
    individual in the population receives the same selection probability. 
    Formally, the \texttt{RandomSelector} is defined as:

    \begin{equation}
      \Sigma_{\mathrm{random}} : 
        \mathbb{P} \times \mathbb{N} \to \mathbb{P};\; 
      (P,\, n) \mapsto \Sigma_{\mathrm{random}}(P,\, n)  
    \end{equation}

    Each individual is then selected based on a uniform probability 
    distribution:

    \begin{equation}
      \rho_i = \frac{1}{|P|}
    \end{equation}
  \end{definition}

  Although the \texttt{RandomSelector} is not frequently mentioned in 
  literature, its straightforward nature provides a valuable touchstone for 
  assessment.
  For illustrative purposes, its performance against the WGP is detailed in 
  \vref{tab:keen:operators:selection:random}.

  \subimport{./}{tab_keen_operators_selection_random.tex}

  As anticipated, the random selector's performance is subpar since it does not 
  consider the fitness of individuals.
  While included in the \textit{Keen} framework for comparative analysis, its 
  application in production environments is discouraged.
