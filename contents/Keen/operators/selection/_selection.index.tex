\subsection{Selection}
\label{sec:keen:operators:selection}
  Building upon the concepts presented in \vref{def:selection_operator}, the 
  \emph{selector operator} is mathematically represented as \(\Sigma : \mathbb
  {P} \times \mathbb{N} \times \cdots \to \mathbb{P}\).
  At its core, the selector operator processes a given population 
  \(\mathbb{P}\), selects \(n\) individuals based on specific criteria, and 
  outputs a subset of the initial population.
  To ensure adaptability, the design challenge centers around establishing a 
  versatile interface to cater to diverse selection strategies.

  A promising approach consists of introducing an interface encompassing a 
  method that requires the population and desired number of individuals as 
  inputs.
  We can incorporate any supplementary parameters as properties of a specific 
  selector object.
  This conceptualization can be defined as:

  \begin{code}{Selector interface}{
    label=lst:keen:operators:selection:interface
  }{kotlin}
    interface Selector<DNA, G : Gene<DNA, G>> {
      operator fun invoke(
          population: Population<DNA, G>,
          count: Int,
          optimizer: PhenotypeOptimizer<DNA, G>
      ): Population<DNA, G>
    }
  \end{code}

  In this model, the \texttt{Selector} interface gets parameterized according 
  to the \texttt{DNA} and \texttt{Gene} types.
  The incorporated \texttt{invoke} method explicitly mentions the population, 
  the individual count, and an instance of the \texttt{PhenotypeOptimizer} 
  class.
  Thanks to Kotlin's capacity for operator overloading, this framework allows 
  direct function invocation through the selector object, depicted as \texttt
  {selector(...)}.\footnote{Syntactic sugar for \texttt{selector.invoke(...)}.}

  We introduce an additional parameter, \texttt{PhenotypeOptimizer}, which 
  signifies the optimization strategy employed to facilitate selection, be it 
  maximization or minimization.\footnote{
    Both these strategies - maximization and minimization - are integral 
    components of the \textit{Utility} module in \textit{Keen}.
    Additionally, the \texttt{PhenotypeOptimizer} interface can be modified to 
    support bespoke optimization techniques.
  }

  \begin{remark}
    To add depth to our architecture, the \texttt{Selector} interface 
    seamlessly integrates with the \texttt{AbstractSelector} abstract class.
    This integration ensures foundational functionalities for all selector 
    types, while also allowing developers to design specialized selectors by 
    extending this class.
  \end{remark}

  \subimport{./}{Random.tex}
  \subimport{./}{Roulette.tex}
    
  \subsubsection{Tournament Selector}
  \label{sec:keen:operators:selection:tournament}
    \Blindtext
