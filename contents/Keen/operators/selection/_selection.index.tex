\subsection{Selection}
\label{sec:keen:operators:selection}
  Building upon the concepts presented in \vref{def:selection_operator}, the \emph{selector operator} is mathematically 
  represented as \(\Sigma : \mathbb{P} \times \mathbb{N} \times \cdots \to \mathbb{P}\).
  At its core, the selector operator processes a given population \(P\), selects \(n\) individuals based on specific 
  criteria, and outputs a subset of the initial population. To ensure adaptability, the design challenge centers around 
  establishing a versatile interface to cater to diverse selection strategies.

  A promising approach consists of introducing an interface encompassing a method that requires the population and 
  desired number of individuals as inputs. We can incorporate any supplementary parameters as properties of a specific 
  selector object. This conceptualization can be defined as:

  \begin{code}{Selector interface}{
    label=lst:keen:operators:selection:interface
  }{kotlin}
    interface Selector<T, G> where G : Gene<T, G> {
        operator fun invoke(
            state: EvolutionState<T, G>, outputSize: Int
        ): EvolutionState<T, G>
    }
  \end{code}

  In this model, the \texttt{Selector} interface gets parameterized according to the \textit{value} and \textit{gene} 
  types. The incorporated \texttt{invoke} method explicitly mentions the state and output size parameters, which are 
  essential for the selection process. Thanks to operator overloading, this framework allows direct function invocation 
  through the selector object, depicted as \texttt{selector(...)}.\footnote{Syntactic sugar for 
  \texttt{selector.invoke(...)}.}

  \subimport{./random/}{Random.tex}
  \subimport{./roulette/}{Roulette.tex}
  \subimport{./tournament/}{TournamentSelector.tex}
