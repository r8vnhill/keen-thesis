\subsubsection{Tournament Selector}
\label{sec:keen:operators:selection:tournament}
  The \texttt{TournamentSelector} is a selection operator that employs the 
  widely-recognized tournament selection strategy found in evolutionary 
  computation.
  At its core, this strategy involves randomly selecting a subset of 
  individuals from the population and evaluating them based on their fitness. 
  From this subset, the individual with the highest fitness is deemed the 
  winner and becomes eligible for participation in the formation of the 
  subsequent generation.
  The \enquote{tournament} process is iteratively carried out until the 
  required number of eligible individuals is chosen.

  \begin{definition}[Tournament Selector]
  \label{def:keen:op:select:tournament}
    The \emph{tournament selector} embodies the tournament selection approach 
    and is formally defined as:

    \begin{equation}
      \Sigma_\mathrm{tournament} :\: 
        \mathbb{P} \times \mathbb{N} \times \mathbb{N} \to \mathbb{P};\;
      (P,\, n,\, ka) \mapsto \Sigma_\mathrm{tournament}(P,\, n,\, k)
    \end{equation}

    Herein, \(P\) represents the population, \(n\) stands for the count of 
    individuals to be deemed eligible, and \(k\) pertains to the tournament or 
    sample size.

    The methodological sequence for selection is as follows:

    \begin{enumerate}
      \item A random subset of \(k\) individuals is culled from the population.
      \item From this subset, the individual bearing the most superior fitness 
        score is singled out.
      \item Steps 1 and 2 are reiterated until the set quota of eligible 
        individuals is reached.
    \end{enumerate}
  \end{definition}

  \begin{remark}
    The selection dynamic is profoundly influenced by the tournament size, 
    \(k\).
    Opting for a larger \(k\) leans towards elitism, predominantly favoring 
    individuals with superior fitness scores.
    In contrast, a reduced \(k\) champions diversity, helping to deter the 
    premature convergence of the genetic algorithm.
  \end{remark}

  \begin{table}[ht!]
    % Table goes here
    \caption{
      Displayed above is an evaluation of the tournament selector's efficacy within the Word Generation Problem (WGP) spanning varied word lengths and tournament dimensions. The methodology involved the utilization of a single-point crossover with a fixed 0.2 crossover probability, coupled with a 0.06 mutation rate. The chosen population featured 500 members with an evolutionary epoch threshold set at 1000 generations. The results presented are an averaged outcome from three distinct experimental runs.
    }
    \label{tab:keen:op:select:tournament}
  \end{table}

  % Discussion or analysis of the results can be inserted here

  For an exhaustive delve into the intricacies and theoretical frameworks 
  surrounding the tournament selection modality, the reader is directed to 
  Blickle et al.'s seminal 
  paper~\autocite{blickleMathematicalAnalysisTournament1995}.
