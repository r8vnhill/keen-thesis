\subsubsection{Tournament Selector}
\label{sec:keen:operators:selection:tournament}
  The \emph{Tournament Selector} is a selection operator that employs the 
  widely-recognized tournament selection strategy found in evolutionary 
  computation.
  At its core, this strategy involves randomly selecting a subset of 
  individuals from the population and evaluating them based on their fitness. 
  From this subset, the individual with the highest fitness is deemed the 
  winner and becomes eligible for participation in the formation of the 
  subsequent generation.
  The \enquote{tournament} process is iteratively carried out until the 
  required number of eligible individuals is chosen.

  \begin{definition}[Tournament Selector]
    \label{def:keen:op:select:tournament}
      The \emph{tournament selector} uses a tournament selection method and is defined as:
  
      \begin{equation}
        \Sigma_\mathrm{tournament} :\: 
          \mathbb{P} \times \mathbb{N} \times \mathbb{N} \to \mathbb{P};\;
        (P,\, n,\, k) \mapsto \Sigma_\mathrm{tournament}(P,\, n,\, k)
      \end{equation}
  
      In this definition, \(P\) is the population, \(n\) is the number of individuals to be selected, and \(k\) is the size of each tournament.
  
      The selection process works as follows:
  
      \begin{enumerate}
        \item Select a random group of \(k\) individuals from the population.
        \item Choose the individual with the highest fitness score from this group.
        \item Repeat steps 1 and 2 until \(n\) individuals have been selected.
      \end{enumerate}
    \end{definition}

  \begin{remark}
    The selection dynamic is profoundly influenced by the tournament size, 
    \(k\).
    Opting for a larger \(k\) leans towards elitism, predominantly favoring 
    individuals with superior fitness scores.
    In contrast, a smaller \(k\) value promotes diversity, which helps prevent the genetic algorithm from converging too early.
  \end{remark}

  For an exhaustive delve into the intricacies and theoretical frameworks 
  surrounding the tournament selection modality, the reader is directed to 
  Blickle et al.'s seminal 
  paper~\autocite{blickleMathematicalAnalysisTournament1995}.
