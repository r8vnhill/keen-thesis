\subsubsection{Random Selector}
\label{sec:keen:operators:selection:random}
  The class \texttt{RandomSelector} serves as a probability selector, acting 
  as a selection strategy that assigns equal probabilities to every 
  individual within a population.
  
  \begin{definition}[Random Selector]
  \label{def:keen:operators:selection:random}
    The \emph{random selector} is characterized by its uniform approach, where every individual in the population is assigned the same selection probability.
    Formally, the \texttt{RandomSelector} is articulated as:
    
    \begin{equation}
      \Sigma_{\mathrm{random}} : 
        \mathbb{P} \times \mathbb{N} \to \mathbb{P};\; 
      (P,\, n) \mapsto \Sigma_{\mathrm{random}}(P,\, n)  
    \end{equation}
    
    Here, each individual is selected following a uniform probability 
    distribution:
    
    \begin{equation}
      \rho_i = \frac{1}{|P|}
    \end{equation}
  \end{definition}
  
  The primary aim of this selector is to serve as a reference point when 
  comparing with other selection strategies.
  Although this type of selector doesn't prominently appear in the 
  literature, its simplicity offers a valuable benchmark for assessment.
  
  As an illustrative point, when tested against the WGP, the results were as 
  outlined in \vref{tab:keen:operators:selection:random}.
