\subsubsection{Roulette Wheel Selector}
\label{sec:keen:op:select:roulette}
  The \texttt{RouletteWheelSelector}, commonly known as \enquote{fitness 
  proportionate selection}, is a widely-used selection method in GAs.
  It allocates each individual in a population a segment of a roulette wheel, 
  proportional to its fitness.
  Thus, individuals with higher fitness have a greater chance of selection.

  \begin{definition}[Roulette Wheel Selector]
    The \emph{roulette wheel selector} ensures selection chances proportional to 
    an individual's fitness.
    It's defined as:

    \begin{equation}
      \Sigma_\mathrm{roulette} :\:
        \mathbb{P} \times \mathbb{N} \times \{0, 1\};\;
      (P,\, n,\, b) \mapsto \Sigma_\mathrm{roulette}(P,\, n,\, b)
    \end{equation}

    where \(P\) represents the population, \(n\) is the individuals' count to 
    select, and \(b\) is a boolean for preceding selection with a sort.

    Let \(\phi_i\) be the fitness of the \(i^{th}\) individual.
    Suppose we have a function \(t : \mathbb{R}^n \to \mathbb{R}^n\) that 
    transforms the fitness into a value that assigns greater probabilities to 
    the fittest individuals according to the ranking strategy.

    The \textit{transformed fitness}, meaning the fitness adjusted to the
    current ranking strategy, is defined as: \(\Phi' = t(\Phi)\).
    Consider \(\Phi'_\mathrm{min} = \min\left\{\Phi_i,\, 0\right\}\) the minimum
    fitness in the population, and \(\phi_i' =  
    \Phi'_i - \min\left\{\Phi'_\mathrm{min},\, 0\right\}\) its \textit{adjusted 
    fitness}.
    The selection probability \(p_i\) for the \(i^{th}\) individual is:

    \begin{equation}
      \rho_i = \frac{\phi_i'}{\sum_{j=1}^{|P|} \phi_j'}
    \end{equation}

    Selection is based on a random number between 0 and 1, selecting \(i\) if 
    the summed probabilities up to \(i\) surpass this number.
  \end{definition}

  \begin{remark}
    The fitness adjustment, \(\phi_i'\), ensures non-negative probabilities. 
    However, it makes the least fit individual unselectable, with a zero 
    probability.
  \end{remark}

  This approach, while ensuring a selection chance for almost every individual, can face challenges. If the fitness 
  values are too close to each other, the selection probability will be too similar, leading to a lack of diversity in 
  the selected individuals. This issue is further discussed in \vref{sec:fn_opt:results}.
