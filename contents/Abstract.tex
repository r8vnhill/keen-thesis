% \setabstractname { Summary }
\begin{abstract}
    This thesis explores Evolutionary Computation (EC) within Artificial Intelligence, particularly focusing on its application in software engineering and scientific computing. It introduces a Kotlin-based EC framework, highlighting its modularity and ability to integrate various EC algorithms. The framework's efficacy is demonstrated through case studies on Genetic Algorithms and Linear Genetic Programming, emphasizing its ability to solve complex computational problems.

    The development is grounded in a detailed review of existing EC frameworks, with an emphasis on their functionalities and architectural details. Additionally, the thesis explores the theoretical dimensions of EC, providing a comprehensive analysis of its fundamental concepts and methodologies. This study enriches the understanding of EC and contributes to academic discussions, establishing a firm foundation for future research and applications.

    Central to the thesis is the hypothesis that Kotlin, with its expressive syntax and cross-platform capabilities, can enhance evolutionary algorithms. This hypothesis is examined through various research questions, focusing on the essentials for an efficient genetic algorithm framework, leveraging Kotlin's unique features, and applying the framework to real-world problems.

    The structure of the thesis is methodically laid out as follows:
    \begin{enumerate}
    \item \textbf{Background}: Establishes the theoretical foundation for the central concepts.
    \item \textbf{State of the Art}: Reviews current advancements and frameworks in EC.
    \item \textbf{Framework Design and Implementation}: Details the architecture and implementation of the Kotlin-based framework.
    \item \textbf{Function Optimization Case Study}: Tests the framework with real-world function optimization problems.
    \item \textbf{Crash Reproduction Problem Case Study}: Demonstrates the framework's application in another domain.
    \item \textbf{Conclusion and Future Work}: Summarizes the work and proposes future research directions.
    \end{enumerate}

    This structure ensures a comprehensive understanding of the research's objectives, methodology, and outcomes, providing a coherent narrative throughout the thesis.

\end{abstract}